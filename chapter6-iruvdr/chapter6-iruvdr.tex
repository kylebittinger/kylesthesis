\documentclass[12pt]{mitthesis} 
\usepackage{kylesthesis}
\begin{document}

\tableofcontents
\clearpage

\subsubsection*{NOTES}
\clearpage

\chapter{IR-UV double resonance LIF/SEELEM spectroscopy of the
  $3^34^1$ and $3^36^1$ \emph{ungerade} vibrational sublevels of $S_1$
  acetylene}

\section{Introduction}

\POINT{Repeat case for studying the non-symmetric bending modes of
  acetylene.}

\POINT{Argument for mode 6: brings molecule to half-linear geometry,
  which may increase coupling.}

\POINT{State goal: look at n.s. bending motions without DD or Coriolis
  coupling.} By selecting vibrational levels with only 
\emph{This spring, Adam and Hans were mapping out the non-symmetric
  bending polyads of acetylene in IR-UV double resonance experiments.
  During this process, they encountered a number of vibrational
  subbands with telltale signs of singlet-triplet coupling: long
  lifetimes, fractionation, and strong quantum beats.  With help from
  Hans, Wilton and I were able to study a few of these bands with the
  SEELEM detector, illuminating the metastable eigenstates connected
  to these allowed transitions.}


Several recent \emph{ab initio} calculations have established a
non-planar equilibrium geometry for the $T_3$ state of acetylene.  The
most recent claculations by Bryan Wong [\emph{JCP} 126, 184307 (2007)]
give a torsional angle of $105^\circ$, which is well in line with the
earlier results of Ventura \emph{et al} [\emph{JCP} 118, 1702 (2003)].
Since the $T_3$ state is known to play a role in coupling the $S_1$
state to the bath of metastable $T_{1,2}$ states, we wish to
investigate vibrational motions that may promote coupling between
$S_1$ and $T_3$.  In particular, we are interested in the torsional
mode $\nu_4'$, which twists the molecule out of a planar geometry.

However, the two non-symmetric bending modes $\nu_4'$ and $\nu_6'$ are
strongly coupled by Darling-Dennison and a-type Coriolis interactions.
Our recent submission to \emph{JPC A} contains an overview of these
effects as they relate to singlet-triplet coupling.  Anthony Merer and
the Singlets understand these couplings very well and are now in the
process of writing what is sure to be the classic paper on this topic.

We wanted to ivestigate subbands where these effects are not present.
To avoid the Darling-Dennison resonance, we chose a polyad with one
quantum of bend.  To escape \emph{a}-type Coriolis coupling, we chose
to look at the $K=0$ levels. We also wanted a lot of signal, so on
Hans' suggestion we turned the laser to the $3^3 B^1$ polyad, where
both $K=0$ subbands show signs of strong singlet-triplet mixing.

This particular polyad has been studied before. The most relevant
paper is Nami and Soji's study of Zeeman quantum beats in the $3^3 B^1
K=1$ subbands [\emph{CPL} 348, 53 (2001)].  They observe strong
splitting and Zeeman quantum beats in the $3^3 6^1$ band, but not in
the band involving torsion.

Our results follow.  The first two spectra give an overview of both
bands.  We use overlapping transitions of $J'=1-6$ in the Q-branch of
the intermediate state to take spectra that \emph{look} like
one-photon spectra.  Unlike Nami and Soji's observations in the $K=1$
levels, we observe splittings and strong SEELEM signal in the $3^3
4^1$ band when $K=0$.

The major advantage of working in double resonance is of course the
ability to simplify the spectrum.  This is especially beneficial for
SEELEM experiments, because the manifold of metastable states
borrowing intensity from one rotational level often overlaps with that
of the next.  We examined each rotational level in turn for $3^3 6^1$,
collecting SEELEM data for each rotational level of the bright state
without overlap from neighboring levels.  Since the spin-orbit
operator is diagonal in $J$, we get this quantum number for free when
we examine one rotational transition at a time.

The SEELEM signals for these transitions are the strongest we have
ever recorded for acetylene.  Even the ``weak'' intensity regions are
full of well-resolved lines, as illustrated in a close-up of our data
from $J'=4$.

Too top it all off, our choice of $K=0$ bands allowed us to observe
the rotationless level of the bright state, coupled to $J'=0$ and
$K=0,1$ levels in the triplet manifold.  In $3^3 6^1$, we couldn't use
a P or R-branch transition in the intermediate state to single out
this level, but we were able to use polarization to record a SEELEM
spectrum without the $J'=0$ levels.


\section{Experiment}

\POINT{Reference previous chapter for most experimental details.}

\POINT{Double resonance notes from AHS, also in his paper.}

\POINT{Say what intermediate states were used for double resonance.}
Transitions ending in the \emph{e}-symmetry branches of $S_1$ $3^36^1$
and $3^34^1$ were recorded using the $J=1-5$ levels of the $S_0$
$\nu_3+\nu_4$ Q branch at 3897.16 \rcm as intermediate states.
Transitions ending in the \emph{f}-symmetry branches were recorded
using various P and R branch transitions.

\POINT{Laser resolution $\approx$ 0.094 \rcm; laser step size
  $\approx$ 0.047 \rcm }

\POINT{Describe polarization method used to record the $J'=0$
  spectrum.  (See p.11 in 4/2007--8/2007 notebook.}

\section{Observations}

\POINT{Present survey spectra for both bands and individual lines for
  $3^36^1$.}

\POINT{Present polarization spectrum for rotationless level of
  $3^36^1$.}

\section{Selection rules for spin-orbit coupling}

\POINT{Discuss spin-orbit selection rules for $\Delta K$, and
  reference tables.  The theory is from Stevens and Brand as well as
  Hougen.}  Spin-orbit selection rules for $N$ and $K$ in polyatomic
molecules are given by Stevens and Brand, but have been reformulated
by other authors \cite{stevens73, howard78, dupre84}.  These follow a
subset of the general rules for singlet$\sim$triplet transitions in
polyatomic molecules originally given by Hougen \cite{hougen64}.



\TODO{Show character tables and axis labels for \emph{trans}
  acetylene.  Discuss the correspondence between the ($a,b,c$) axis
  system and the ($x,y,z$).  See p.9, 18--19 in 4/2007--8/2007 notebook.}

\begin{table}
  \centering
  \begin{tabular}{llllrl}
    Vib.
    & $^{v}\Gamma_S$ & $^{ev}\Gamma_S$ & $\Gamma_\sigma$ 
    & $\Delta K$ & $^{v}\Gamma_T$ \\
    \toprule

%     $2^1 3^1 4^2$ 
%     & $a_g$ & $^{1}A_u$ & $B_g$ & $0$ & \textcolor{red}{$a_g$} \\
%     & & & $A_g, B_g$ & $\pm 1$ & $a_g, b_g$ \\

%     $2^1 3^1 6^2$
%     & $a_g$ & $^{1}A_u$ & $B_g$ & $0$ & \textcolor{red}{$a_g$} \\
%     & & & $A_g, B_g$ & $\pm 1$ & $a_g, b_g$ \\

%     $2^1 3^1 4^1 6^1$ 
%     & $b_g$ & $^{1}B_u$ & $B_g$ & $0$ & \textcolor{red}{$b_g$} \\
%     &      &                 & $A_g, B_g$ & $\pm1$     & $a_g, b_g$ \\

    $3^3 4^1$ 
    & $a_u$ & $^{1}A_g$ & $B_g$ & $0$ & \textcolor{red}{$a_u$} \\
    & & & $A_g, B_g$ & $\pm1$ & $a_u, b_u$ \\[10pt]

    $3^3 6^1$ 
    & $b_u$ & $^{1}B_g$ & $B_g$ & $0$ & \textcolor{red}{$b_u$} \\
    & & & $A_g, B_g$ & $\pm1$ & $a_u, b_u$ \\

  \end{tabular}\\[5mm]
  
  $C_{2h}$ symmetry, $^{e}\Gamma_T =$ $^{3}B_u$\\[1cm]

  \begin{tabular}{llllrl}
    Vib.
    & $^{v}\Gamma_S$ & $^{ev}\Gamma_S$ & $\Gamma_\sigma$ & $\Delta K$ & $^{v}\Gamma_T$ \\
    \toprule
%     $2^1 3^1 4^2$
%     & $a$ & $^{1}A$ & $B$ & $0$ & \textcolor{red}{$a$} \\
%     & & & $A, B$ & $\pm 1$ & $a, b$ \\

%     $2^1 3^1 6^2$
%     & $a$ & $^{1}A$ & $B$ & $0$ & \textcolor{red}{$a$} \\
%     & & & $A, B$ & $\pm 1$ & $a, b$ \\
    
%     $2^1 3^1 4^1 6^1$
%     & $b$ & $^{1}B$ & $B$ & $0$ &\textcolor{red}{$b$} \\
%     & & & $A, B$ & $\pm1$ & $a, b$ \\

    $3^3 4^1$ 
    & $a$ & $^{1}A$ & $B$ & $0$ & \textcolor{red}{$a$} \\
    & & & $A, B$ & $\pm1$ & $a, b$ \\[10pt]

    $3^3 6^1$ 
    & $b$ & $^{1}B$ & $B$ & $0$ & \textcolor{red}{$b$} \\
    & & & $A, B$ & $\pm1$ & $a, b$ \\

  \end{tabular}\\[5mm]

  $C_{2}$ symmetry, $^{e}\Gamma_T =$ $^{3}B$
\end{table}

\POINT{Show intensity factors for each coupling component and
  forbidden transitions from $3^3B^1$ to $T_3$ levels.  (See p.138,143 of
  1/2007--3/2007 notebook.)}

\TODO{Figure: Rosetta stone for singlet-triplet coupling.  (See p.127
  of 1/2007--3/2007 notebook.)}

\TODO{Use Bryan's program to compute overlap integrals for the two bands.}

\POINT{Show number of modes available in each symmetry as a function
  of energy.}

\section{Analysis: Intensity distributions}

\POINT{Present energy levels and reduced term value plot for $3^36^1$.
  (See p.97 of 4/2007--8/2007 notebook.)} 

\section{Analysis: Parsimonious trees}

\POINT{Use $\Delta_3$ statistic, $\Sigma^2$ statistic, or Fourier
  transform methods? (See p.65 in 11/2007--1/2008 notebook.)}

\section{Analysis: Gated fluorescence and ``smooth transition''
  between SEELEM and LIF spectra}

\section{LIF/SEELEM intensity ratio analysis}

\section{Conclusion}

\bibliography{master}
\bibliographystyle{plain}
\end{document}