\documentclass[12pt]{article}

\usepackage{amsmath, amssymb, fullpage}
\usepackage[pdftex]{graphicx}

\setlength{\parindent}{0pt}
\setlength{\parskip}{2ex}

\pagestyle{empty}

\begin{document}
142 Pine St.\\
Cambridge, MA 02139

\today

Jeffrey S. Saltzman, Ph.D.\\
Senior Director\\
Applied Computer Science and Mathematics\\
Merck \& Co., Inc.\\
West Point, PA 19486

% Do not address the letter 'Dear Sir or Madam'.
Dear Dr. Saltzman,

% Tell why you are writing, name position in which you are interested
I am writing in regard to the Ph.D.-level position in Applied Computer
Science \& Mathematics Information Technology (INF003308).
% How did you hear about this opening?
Stefan Zajic, a colleague of mine from M.I.T. and a pharmacokineticist
at your company, brought this position to my attention.
% Introduce self and state when you will complete program
I am a Ph.D. candidate in the Chemistry Department at M.I.T., and I
expect to complete my studies in June 2008.
% Lead into next paragraph
I believe my computational skills and research experience as a
spectroscopist make me an excellent candidate for this position.

A cornerstone of my graduate research is the development of simple
mathematical models to describe the complex and tangled spectra of
molecules.
% Mention one or two qualifications you think would be of greatest 
% interest to the employer.
One example of this is my extension of a spectral
deconvolution method to determine the energy of a mediating
``doorway'' triplet state from a high-resolution spectrum, allowing the
separation of global properties from the statistical properties of the
system.
%  The properties of
% such a doorway are crucial parameters for a model of the molecular
% interactions, while the remainder of the system behaves in a
% statistical manner.

Building such models has allowed me to apply my interest in
programming to chemical problems.  
To simulate large ensembles of interacting energy levels, I translated 
\textsc{Fortran} routines involving the statistics of random matrices 
to Python.
I wrote much of my thesis work in Python, following standards for 
documentation, unit testing, and source code management.
In addition, my reputation in the department for text processing in Perl 
led to a talk for M.I.T. computational chemists called \emph{Perl in the 
Lab} in 2006.
% Familiarity with experimental techniques?
% My chemistry research experience includes scattering theory
% (used in PET imaging), as well as training in NMR and mass
% spectroscopy.
% Tell why you are particularly interested in the company, type of work 
% or location.
% If you have related experience, or specialized training, point it out.
% Refer the reader to the enclosed resume, which will give additional
% information concerning your background and interests.

I wish to bring my knowledge of chemistry and my computational skills
to Merck \& Co., Inc., in order to work alongside other leading
scientists with the mission of improving human health.
% in order to help other leading scientists on their
% mission to improve human health.
% My desire now is to use my abstract knowledge of chemistry to help
% other scientists hoping to make a difference in peoples' health.  I
% want to worck for merck
One of the things that impresses me about the ACSM department is the
opportunity to collaborate with a variety of people on challenging and
stimulating problems.
% Something about the location
% I would most enjoy working at the West Point location,
% because I have family in and around Harrisburg, PA.
% Close by stating your desire for an interview.
I would like to arrange an interview for this position at your
convenience.
% Give info
Please contact me by email (kyle2@mit.edu) or phone (617-794-3807) to
schedule an interview.  Thank you for your time and consideration.

Sincerely,

\vspace{5mm}

Kyle Bittinger


\end{document}