\documentclass[12pt]{mitthesis}
\usepackage{lgrind, braket, amsmath, amssymb, bbm, booktabs, color}
\usepackage[pdftex]{graphicx}
\usepackage[version=3]{mhchem}
\pagestyle{plain}

\newcommand{\TODO} [1]{\textcolor{magenta}{\textbf{TODO:} #1}}
\newcommand{\POINT}[1]{\textcolor{magenta}{\emph{#1}}}

\hyphenation{acetylene}
\hyphenation{Hamiltonian}

\begin{document}
\tableofcontents
\clearpage

\chapter{Introduction}

\section{The importance of studying triplet states and intersystem
  crossing}

\POINT{Introduce triplet states and motivate research. Metastable
  energy carriers, easy to populate collisionally, hard to study.}

\POINT{Introduce ISC as the dynamical manifestation of mixing between
  states of different multiplicities \cite{kommandeur87, robinson67,
    tramer05}.}

\POINT{Introduce ISC parameter $\gamma = \langle H_{sd} \rangle /
  \rho_d$}

\TODO{Go through talks, proposals and gather motivation points.}

\TODO{Overview of acetylene spectrum \cite{watson82}.}

\section{The low-lying triplet electronic states of acetylene:
  molecular orbital theory and \emph{ab initio} calculations}

\POINT{Summarize the electronic structure of the acetylene valence
  triplet states.  Show molecular orbitals and energy level diagrams.}
Figure \ref{fig:mol-orbitals} shows the principal molecular orbital
configurations for the 4 lowest excited electronic states of
acetylene \cite{yamaguchi93}.

\POINT{Discuss the change in geometry, cis and trans wells.}  The
$a_g^*$ molecular orbital is stabilized with increasing \ce{CCH} bond
angle.  Balanced against the destabilization of filled
$\sigma$-bonding orbitals, this effect creates potential energy wells
in bent geometry for the low-lying electronic states of acetylene.
The valence excited states of acetylene considered in this work,
namely $S_1$, $T_1$, $T_2$, and $T_3$, each have minima in both the
\emph{cis} and\emph{trans} geometries.

\POINT{Discuss the forbiddenness of the singlet transition and how it
  is made allowed only in the \emph{trans} well.}  Since the
$\tilde{A}^1A_u \leftarrow \tilde{X}^1\Sigma_g$ transition is
localized in the \emph{trans} well of acetylene, the process of
intersystem crossing between singlet and triplet excited states
proceeds locally on this side of the barrier.  For this reason, we
will no longer explicitly write the geometry when referring to the
valence electronic states of acetylene; the \emph{trans} geometry is
implied unless otherwise stated.

\begin{figure}
  \caption{Principal molecular orbital configurations for the valence
    excited states of \emph{trans}-bent acetylene.  The four
    lowest-lying electronic states are listed in energy order above
    the diagram.  An electronic transition from $S_1$ to $T_3$
    involves a change in one antibonding spin orbital, from
    $a_g^*(\downharpoonright)$ to $b_g^*(\upharpoonright)$.}
  \label{fig:mol-orbitals}

  \centering
  \setlength{\unitlength}{1cm}
  \begin{picture}(8,8)
    \put(0,1){$b_u$}
    \put(0,2){$a_u$}
    \put(0,4){$a_g^*$}
    \put(0,5){$b_g^*$}


    \put(1,1){\line(1,0){1}}
    \put(1,2){\line(1,0){1}}
    \put(1,4){\line(1,0){1}}
    \put(1,5){\line(1,0){1}}
    \put(1.333,0.666){\vector(0, 1){.666}}
    \put(1.333,1.666){\vector(0, 1){.666}}
    \put(1.666,2.333){\vector(0,-1){.666}}
    \put(1.666,3.666){\vector(0, 1){.666}}
    \put(1.333,7){$T_1$}
    \put(1.333,6){$B_u$}

    \put(3,1){\line(1,0){1}}
    \put(3,2){\line(1,0){1}}
    \put(3,4){\line(1,0){1}}
    \put(3,5){\line(1,0){1}}
    \put(3.333,0.666){\vector(0, 1){.666}}
    \put(3.333,1.666){\vector(0, 1){.666}}
    \put(3.666,1.333){\vector(0,-1){.666}}
    \put(3.666,3.666){\vector(0, 1){.666}}
    \put(3.333,7){$T_2$}
    \put(3.333,6){$A_u$}


    \put(5,1){\line(1,0){1}}
    \put(5,2){\line(1,0){1}}
    \put(5,4){\line(1,0){1}}
    \put(5,5){\line(1,0){1}}
    \put(5.333,0.666){\vector(0, 1){.666}}
    \put(5.333,1.666){\vector(0, 1){.666}}
    \put(5.666,1.333){\vector(0,-1){.666}}
    \put(5.666,4.666){\vector(0, 1){.666}}
    \put(5.333,7){$T_3$}
    \put(5.333,6){$B_u$}


    \put(7,1){\line(1,0){1}}
    \put(7,2){\line(1,0){1}}
    \put(7,4){\line(1,0){1}}
    \put(7,5){\line(1,0){1}}
    \put(7.333,0.666){\vector(0, 1){.666}}
    \put(7.333,1.666){\vector(0, 1){.666}}
    \put(7.666,1.333){\vector(0,-1){.666}}
    \put(7.666,4.333){\vector(0,-1){.666}}
    \put(7.333,7){$S_1$}
    \put(7.333,6){$A_u$}

  \end{picture}
\end{figure}

\POINT{Discuss spin-orbit coupling with S1 state.  Show the spin-orbit
  operator and one-electron matrix elements among the triplet states.
  The $\ell_+s_-$ operator requires a change in spin-orbital.
  (Examples of Slater determinants: p.59 of 11/2007--1/2008 notebook
  [\ce{CO2}])}

\POINT{Many ab initio studies of acetylene triplet states
  \cite{demoulin75, lischka86, cui96, cui97, malsch98, dallos02,
    ventura03, thom07}.  Give a summary of the most recent theoretical
  results.}  The electronic energy of $T_3$ has been determined in
several \emph{ab initio} calculations.  Malsch and coworkers found an
energy of 5.46eV, almost 2200cm$^{-1}$ higher than their calculated
energy for $S_1$ \cite{malsch98}.  The most recent and sophistocated
determination of the $T_3$ energy was carried out by Thom and
coworkers \cite{thom07}.  After explicitly diabatizing the $T_2$ and
$T_3$ surfaces, they obtained an energy difference $T_e(S_1)-T_e(T_3)$
of 270cm$^{-1}$, placing the $T_3$ state at approximately the same
energy as, and perhaps slightly lower than, the $S_1$ state.  It is in
light of these most recent calculations that we continue our
discussion.

\POINT{Almost no direct experimental evidence concerning triplet
  states.  No spectra for trans-bent triplet states, although spectra
  have been recorded for the cis-well.}

\POINT{Discuss propensity/selection rules for triplet-triplet
  transitions.  No allowed T-T electronic transitions among the
  valence states in the trans well.}

\POINT{Discuss density of triplet vibrational levels as a function of
  energy.  Discussion of Ryan and Bryan's paper here \cite{thom07}?}

\POINT{Show the vibrational modes of \emph{trans} acetylene and their
  symmetries.  Explain selection rules for transitions from the ground
  state.  (See p.33 of1/2007--3/2007 notebook.)}


\section{A heirichy of electronic coupling in acetylene: evidence from
  Zeeman anticrossing experiments}

In a Zeeman anticrossing experiment, the laser is locked on an optical
transition in the molecule (we'll call this ``the singlet level'').
As the magnetic field is tuned, triplet-mixed levels tune through
resonance with the singlet.  Relative to the singlet level at
$E=0$, the energy of a mixed triplet level is (linear Zeeman effect)
\begin{equation}
E = g \, M_s \, \mu_B \, (B-B_0),
\end{equation}
where $g$ is the mixed-triplet level's $g$-factor, $M_s$ is the
lab-fixed projection of spin, $B_0$ is the field required to bring the
level's energy to 0, and $\mu_B$ is the Bohr magneton (constant).  An
excited electron which is not coupled to the molecular axis (Hund's
case \emph{b}) has a $g$-factor of $2.0023$.

When the singlet level and one of the weakly interacting background
levels are at exactly the same energy, they become 50:50 mixed.  When
this happens under the experimental conditions, half the molecules
decay by collision instead of fluorescence, and the fluorescence
intensity decreases by half.  The \emph{width} of the resonance is
proportional to the matrix element between the bright state and the
interacting level:
\begin{equation}
\Delta B = \frac{4\,V}{g\,\mu_B}
\end{equation}

What were the results and conclusions of these experiments?

\subsection{Paper 1: Anomalous behavior of the anticrossing density}

Dupr\'{e}, Jost, Lombardi, Green Abramson, and Field.
\emph{Chem. Phys.}  \textbf{152}, 293 (1991).\\
Measured anticrossing spectra of $\nu_3=1-3$ rotationless levels,
field strength $0-8T$.

\begin{description}
\item[Observation: Rapid increase in density of mixed-triplet levels
  with $\nu_3$ excitation.]  Observed density of states is larger than
  number of $T_{1,2}$ levels, therefore the triplet levels must mix
  with and contaminate the denser manifold of $S_0$ levels.

  However, increasing the $T_{1,2} \sim S_0$ coupling by itself does
  not make more mixed-triplet levels accessible to cross the $S_1$
  level.  The $S_1$ level can only undergo triplet-mediated mixing
  with $S_0$ levels over an energy range on the order of the $S_1 \sim
  T$ matrix element.

  Therefore, coupling between $S_1$ and a subset of the triplet levels
  $T$ must increase to make the dense manifold of contaminated
  $T_{1,2} \sim S_0$ levels observable in the experiment.

\item[Observation: Rapid increase in anticrossing linewidth.]  Since
  the linewidth is proportional to the $S_1 \sim T$ matrix element,
  this supports the conclusion that the $S_1 \sim T$ coupling is
  increasing with excitation in $\nu_3$.

\item[Excluded Mechanism: Direct $S_1 \sim S_0$ internal conversion.]
  No mechanism for sudden increase in vibrational overlap integrals
  between relatively unmixed $S_1$ levels and essentially fully mixed
  $S_0$ levels over this energy range.

\item[Excluded Mechanism: $S_0$ dissociation limit increases total
  density of states.]  The increase in state density would be
  accompanied by a decrease in vibrational overlap integrals.

\item[Proposed Mechanism: $S_1 \sim T_3$ curve crossing.]  The
  observed $S_1 \sim T$ coupling could be induced by a sparse
  collection $T_3$ levels, mostly distant.  Question proposed by
  authors: too many broad anticrossings in $3\nu_3$ spectrum?

\item[Proposed Mechanism: $T_2$ linear isomerization barrier.]  The
  vibrational overlap between $T_2$ levels in the \emph{cis}-well and
  $S_1$ levels in the \emph{trans}-well would increase in the region
  of the linear turning point.
\end{description}

\subsection{Paper 2: Characterization of a large singlet-triplet
  coupling}

Dupr\'{e} and Green. \emph{Chem. Phys. Lett.} \textbf{212}, 555
(1993).\\
Measured LIF spectrum of rotationless $3\nu_3$ level at 0T and near 7.075T

Ryan provides a useful summary of this paper in his article ``Studies
of Intersystem Crossing Dynamics in Acetylene,'' \emph{JCP}
\textbf{126}, 184307 (2007).  

A $T_3$ level with energy $\Delta E = \pm 6.67$ cm$^{-1}$ interacts
with the $J=K=0$ level of $3\nu_3$.  This strong interaction is very
well characterized by the Zeeman anticrossing data.  Many weaker
interactions with nominal $S_0$ levels can be seen perturbing the
strong $S_1 \sim T_3$ anticrossing.  The zero-field $S_1 \sim T_3$
matrix element is $0.29$ cm$^{-1}$ if $K_{T_3}=0$ or $0.58$ if
$K_{T_3}=1$.

\subsection{Paper 3: Quantum beat spectroscopic studies}

Dupr\'{e}, Green, and Field.  \emph{Chem. Phys.} \textbf{196}, 211
(1995).\\
A number ($\sim 60$) of anticrossings from $\nu_3=0-2$ are analyzed
individually.

\begin{description}
\item[Observation: Large variation in measured $g$ factors.]
  Mixing between triplet states with the selection rule $\Delta M_S =
  \pm 1$ causes the $g$ factors to vary from the expected value of
  $2$.  The triplet$\sim$triplet interactions must be strong relative
  to the mediated interactions between $S_1$ and $T_{1,2}$.

  If the $S_0 \sim T_{1,2}$ were strong, the $g$ factors would all be
  close to zero.

\item[Observation: All quantum beats change with field strength.]
  Direct ``internal conversion'' interactions between $S_1$ and $S_0$
  would give rise to quantum beats which would not tune with magnetic
  field strength ($M_S=0$ for both states).

\item[Proposed order of interaction strengths: $T_1 \sim T_2 \gg S_1
  \sim T_{1,2} \gg S_0 \sim T_{1,2} \gg S_1 \sim S_0$.] The first
  inequality is supported by the large variation in measured $g$
  factors.  The second inequality allows the dense number of $S_0$
  levels to be sampled in the experiment, as observed in paper \#1.
  The third inequality states that the internal conversion process
  from $S_1$ to $S_0$ is triplet-mediated, as supported by the second
  observation in this paper.

  The ordering of interaction strengths not only proceeds from the
  observations, but also makes sense according to our chemical
  intuition.
\end{description}

\subsection{Paper 4: Study by Fourier transform}

Dupr\'{e}. \emph{Chem. Phys.} \textbf{196}, 239 (1995).\\
The product $\rho_{vib} \cdot \langle V \rangle$ is derived from the
Fourier xform of anticrossing spectra for $\nu_3=0-4$.

\begin{description}
\item[Observation: The product $\rho_{vib} \cdot \langle V \rangle$
  increases exponentially from $0\nu_3$ to $4\nu_3$.]  See figure 1
  for a great example of how the spectra change from $0\nu_3$ to
  $4\nu_3$ (ignore frame E, which is the singlet perturber in
  $3\nu_3$).  The density of states does not increase appreciably over
  this energy range, so the effect must arise from an increasing
  matrix element.
\end{description}

\section{Doorway-mediated intersystem crossing in acetylene: evidence
  from laser-induced fluorescence and quantum beats}

\POINT{Summarize and explain the results of the Zeeman anticrossing
  experiments.} 

Main ideas:
\begin{enumerate}

\item Strong triplet-triplet interactions dominate the intramolecular
  dynamics.  Evidence: the local Land\'{e} g-factor varies from
  $0.3-2.0$.

\item The $T_1 \sim T_2$ coupling strength increases only slightly over
  the energy range of $0\nu_3'-2\nu_3'$.  Evidence: an observed
  decrease in the average Land\'{e} g-factor with energy.

\item Couplings between the $S_1$ and $S_0$ electronic states are
  induced by mixed triplet levels.  Evidence: \TODO{Find evidence for
    this.}

\item There is a rapid increase in $S_1 \sim T_3$ coupling in the range of
  $0\nu_3'-2\nu_3'$. \TODO{Cite evidence.}

\end{enumerate}
These observations led the authors to propose a heirarchy of of
interactions between the electronic states of acetylene:
\begin{equation}
  T_1 \sim T_2 \gg S_1 \sim T \gg S_0 \sim T \gg S_1 \sim S_0
\end{equation}

\POINT{Talk about the ZQB experiments of Tsuchiya.}


\POINT{Discuss observation of triplet perturbations by HB/AHS/AM in
  their investigation of bending polyads.}


\section{Detection of laser-excited metastables by surface electron
  ejection}

\POINT{Introduce method and cite papers by Chesnovsky.}

\POINT{Talk about earlier SEELEM experiments on acetylene.}


\section{Summary}

In Chapter 2, we develop several frameworks for thinking about
doorway-mediated intersystem crossing.  Using these frameworks, the
relative statistical properties of LIF and SEELEM intensity
distributions are derived for several doorway-coupling model
Hamiltonians.  Moving beyond statistical properties, Chapter 3
demonstrates that a doorway-coupling model Hamiltonian can be
generated directly from spectral data by applying the standard LKL
deconvolution technique in a recursive manner.

Chapter 4 presents a new and extended analysis of the well-studied but
incompletely characterized $3\nu'_3$ vibrational level of $S_1$
acetylene, highlighting experimental evidence for a new model of
doorway-mediated coupling.  The new model proposed for $3\nu'_3$
includes two doorway states: one distant doorway state with a large
matrix element controls the overall coupling strength, while another
local doorway with a smaller matrix element is responsible for
splittings at low values of the rotational quantum number $J$.
Several other vibrational levels of $S_1$ acetylene are investigated
in Chapter 5.  The complex interplay between bending vibrations and
Singlet$\sim$Triplet coupling is apparent from the drastically
different LIF/SEELEM spectra obtained.  We discuss the ubiquitous role
of $T_3$ in mediating $S_1 \sim T_{1,2}$ coupling in light of these
results.  Chapter 6 presents the results of IR-UV double resonance
SEELEM experiments, which reveal similar Singlet-Triplet coupling
characteristics for the torsion and antisymmetric in-plane bending
motions.

We depart from our study of the acetylene molecule in Chapter 7, where
a new method is considered for the manufacture of metastable molecules
in a molecular beam.  The basis of this method is two-photon optical
pumping of metastable atoms followed by collisional excitation
transfer.  Chapter 8 concludes and discusses the prospects for future
research.

\bibliography{master}
\bibliographystyle{plain}
\end{document}