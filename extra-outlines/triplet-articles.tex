\documentclass[12pt]{mitthesis}
\usepackage[pdftex]{graphicx}
\begin{document}

\newcommand{\mytitle}[1]{\textbf{#1}}
\newcommand{\AtoX}{$
  \tilde{A} \: ^1\!A_u 
  \leftarrow 
  \tilde{X} \: ^1\Sigma_g^+
  $}
\section*{Triplet Project Articles}
\begin{enumerate}
\item \mytitle{Deconvolution of spectral data to produce a
    doorway-coupling model Hamiltonian} (adaptation of Kyle's Chapter
  3)

\item \mytitle{COMMUNICATION: Signatures of doorway-mediated
    intersystem crossing in delayed, incoherent fluorescence
    measurements.}  Describe the technique of using delayed
  fluorescence to locate and assign $T_3$ doorway levels. (adaptation
  of Kyle's Chapter 4, Section 2)

\item \mytitle{FEATURE: SEELEM/LIF spectroscopy of acetylene: Spectral
    signatures of energetically distant doorway levels}  Show off
  spectra of the $2^13^2$, $2^23^1$, $3^24^2$, and $3^3$ $K$=2
  sublevels, and discuss the location and assignment of $T_3$
  doorways. (adaptation of remaining material in Kyle's Chapter 4)

\item \mytitle{IR-UV double resonance SEELEM spectroscopy of the
    $3^3B^1$ \emph{ungerade} bending polyad of $S_1$ acetylene.}
  Compare singlet$\sim$triplet interactions in $3^36^1$ and $3^34^1$
  $K$=0.  (adaptation of Kyle's Chapter 6, not written yet)

\item \mytitle{COMMUNICATION: Direct observation of acetylene
    $T_{1,2}$ state density in the SEELEM spectrum of $3^36^1$
    $J$=$K$=0.} (not written yet)

\item \mytitle{Evidence for a singlet$\sim$triplet dynamical ``double
    doorway'' in acetylene \AtoX\ $V^3_0K^1_0$} (not written yet)

\end{enumerate}
\end{document}