\documentclass[12pt, letter]{article}

\usepackage{amsmath, amssymb, fullpage}
\usepackage[pdftex]{graphicx}
\usepackage{multicol}
\usepackage{fancyhdr}
\usepackage{lastpage}

\renewcommand{\labelitemi}{}
\setlength{\parindent}{0pt}
\setlength{\parskip}{1ex}
\hyphenation{Ionization}

\newcommand{\hellomynameis}[1]{{\large \textbf{#1}}}
\newcommand{\cvsection}[1]{
  \vspace{3mm}

  \textsc{#1}

  \vspace{3mm}
}

\pagestyle{fancy}
\setlength{\headheight}{15pt}
\setlength{\headsep}{30pt}
\setlength{\textheight}{598pt}
\setlength{\footskip}{0pt}
\lhead{Kyle Bittinger {\em Curriculum Vitae}}
\chead{}
\rhead{kyle2@mit.edu}
\lfoot{}
\cfoot{}
\rfoot{\thepage\ of \pageref{LastPage}}
\renewcommand{\headrulewidth}{0.5pt}
\renewcommand{\footrulewidth}{0pt}

\fancypagestyle{plain}{
\setlength{\headheight}{0pt}
\setlength{\headsep}{0pt}
\setlength{\footskip}{46pt}
%\setlength{\textheight}{610pt}
\fancyhf{}
\fancyfoot[R]{\thepage\ of \pageref{LastPage}}
\renewcommand{\headrulewidth}{0pt}
\renewcommand{\footrulewidth}{0pt}}




\begin{document}

\thispagestyle{plain}

\begin{center}
\emph{Curriculum Vitae}\\[1mm]
\hellomynameis{Kyle L. Bittinger}\\[1mm]
kyle2@mit.edu
\end{center}

\begin{multicols}{2}
  Office: \\
  Massachusetts Institute of Technology \\
  77 Massachusetts Ave. Room 6-022 \\
  Cambridge, MA 02139 \\
  (617) 252-1975

  Home: \\
  142 Pine St. \\
  Cambridge, MA 02139 \\
  Cell: (617) 794-3807
\end{multicols}

\begin{tabular}{@{}l@{\extracolsep{1cm}}p{13.2cm}}
  \textsc{Current}
  & \textbf{Massachusetts Institute of Technology}, Cambridge, MA\\
  \textsc{Research}
  & Graduate Research Assistant\\
  \textsc{Experience}
  & Research Advisor: Prof. Robert W. Field\\
  \textsc{2001-2008} 
  & Department of Chemistry / George Harrison Spectroscopy Laboratory\\
  \\
  & Developed mechanistic models to describe
  the rich information encoded in high-resolution molecular spectra.
  Recorded spectra of the acetylene molecule using a variety of experimental
  techniques, such as laser-induced fluorescence, double resonance, and
  surface electron ejection by laser-excited metastables (SEELEM).
  Interpreted simultaneously recorded datasets in terms of hierarchical
  interactions between short-lived (singlet) and long-lived (triplet) 
  excited electronic states.
  Investigated these mechanisms as they relate to bending motions in the 
  molecule.
  Enhanced and extended the method of Lawrance-Knight spectral
  deconvolution to deduce the energy and matrix element of a mediating
  ``doorway'' triplet state from the experimental data.
  Modeled molecular systems involving statistical ensembles of triplet 
  states using the NumPy and SciPy libraries for Python.
  \\

  \\

\textsc{Education}
 & \textbf{Massachusetts Institute of Technology}, Cambridge, MA\\
 & Department of Chemistry \\
 & Ph.D. Candidate, Sixth Year \\
 & Expected graduation date: June 2008 \\
 & Thesis Advisor: Prof. Robert W. Field \\
 & Thesis Title: Metastable States of Small Molecules \\
 & \\
 & \textbf{University of Pittsburgh}, Pittsburgh, PA \\
 & B.S., \emph{Magna Cum Laude}, 2001 \\
 & Major: Chemistry (Bioscience option)\\
 & Minor: Mathematics\\
 & Senior Thesis: Angle-Energy Resolved Scattering in the\\
 & \hspace{5mm}Penning Ionization of $H_2$ by $He^*$\\
\end{tabular}



\begin{tabular}{@{}l@{\extracolsep{1cm}}p{13.2cm}}
  \textsc{Previous}
  & \textbf{University of Pittsburgh}, Pittsburgh, PA\\
  \textsc{Research}
  & Undergraduate Research Assistant\\
  \textsc{Experience}
  & Department of Chemistry\\
  \textsc{2000-2001}
  & Research Supervisor: Prof. Peter E. Siska\\
  \\
  & 
  Measured the angular distribution of $H_2^+$ formed in the reaction of 
  $H_2$ with metastable $He^*$ atoms, using mass spectrometry. 
  Interpreted the results in the framework of classical scattering theory. 
  Modified and ran \textsc{Fortran} programs to simulate the experiment 
  and predict ion distributions.\\
\end{tabular}

\cvsection{Honors \& Awards}

\begin{tabular}{@{}l@{\extracolsep{7mm}}l}
2001-2002 & DuPont Graduate Fellow, Department of Chemistry, M.I.T.\\
2001 & Mary Louise Theodore Award\\
     & Department of Chemistry, University of Pittsburgh\\
2000 & Carrie T. Holland Scholarship\\
     & Department of Chemistry, University of Pittsburgh\\
\end{tabular}

\cvsection{Publications \& Presentations}

\textbf{Bittinger KL}, Virgo WL, Field RW. Double doorway in the
$V^3_0K^1_0$ vibrational subband of acetylene revealed by Enhanced
Lehmann-Lawrance-Knight deconvolution. \emph{in preparation}. 

\textbf{Bittinger KL}, Virgo WL, Field RW. IR-UV double resonance
SEELEM spectroscopy of the $3^3 B^1$ polyad of acetylene. \emph{in
  preparation}.

Virgo WL, \textbf{Bittinger KL}, Steeves AH, Field RW. Contrasting
dynamical behavior of two vibrational levels of the $S_1$ $2^1 3^1
B^2$ polyad of acetylene. \emph{Journal of Physical Chemistry A}
\textbf{111}, 12534 (2007).

\textbf{Bittinger KL}, Virgo WL, Field RW. Singlet-triplet coupling
and the non-symmetric bending modes of acetylene $^1\tilde{A}_u$.
\emph{68th International Symposium on Molecular Spectroscopy} in
Columbus, OH, June 18-23, 2007.

\textbf{Bittinger KL}. Perl in the Lab. \emph{Atomistic Modeling And
Simulation Seminar: Show and Tell Session}, M.I.T., November 8, 2006.

Gulati K, Longley EG, Dorko MJ, \textbf{Bittinger KL}, Siska
PE. Angle-energy distributions of Penning ions in crossed molecular
beams IV. $He^* (2 ^1S, 2 ^3S) + H_2 \rightarrow He + H_2^+ + e^-$.
\emph{Journal of Chemical Physics} \textbf{120}, 8485 (2004).

\clearpage

\cvsection{Skills - Analysis \& Programming}

Extensive experience in Python (NumPy/SciPy), Perl, \textsc{Matlab}, LabView,\\
\hspace*{10mm}Linux/UNIX environments\\
Familiarity with Ruby, \textsc{Fortran}, Haskell, Javascript, Scheme\\
Pattern recognition techniques, including extended cross correlation\\
Bayesian parameter estimation and model selection

% \textsc{Experimental}\\
% Experience with pulsed and CW laser systems, as well as standard
% optics techniques.  Spectroscopic techniques including double
% resonance, two-photon excitation, photoacoustic spectroscopy,
% detection of metastables, and dispersed fluorescence.  High vacuum and
% molecular beam techniques, quadrupole and time-of-flight mass
% spectrometry.

\cvsection{Leadership Experience}

\vspace{2mm}

\begin{tabular}{@{}l@{\extracolsep{1cm}}p{13.7cm}}
2002-2007 
 & Mentored five undergraduate researchers on experimental and computational 
 research projects. Encouraged development of semi-independent, student-led 
 projects.\\
\\

2001
 & Worked with a team of teaching assistants in a laboratory class. 
 Created standards and guidelines for evaluating student performance.
 Coordinated grading and scheduling with lab supervisors and team 
 members.
 \\
\\

2000, 2003
 & Served as Engineering Director at WMBR (Cambridge) and WPTS (Pittsburgh). 
 Oversaw technical projects and improvements. Organized response among the 
 engineering team to equipment failures.\\
\\

1999-2001
 & Taught organic chemistry laboratory for two years as an undergraduate. 
 Supervised and evaluated students with and without assistance.
 Explained concepts in a classroom setting, in addition to individual 
 instruction.\\

\end{tabular}

\cvsection{Other Interests}

Audio engineering and recording (I am a long-time live engineer at
WMBR in Cambridge, and have published several equipment reviews in
\emph{Tape Op} magazine), amateur electronics, playing music with
others, organic farm work

% \cvsection{References}

% \begin{multicols}{2}
% Prof. Robert W. Field\\
% Massachusetts Institute of Technology\\
% Department of Chemistry\\
% 77 Massachusetts Ave. Room 6-215\\

% Dr. Stephen Coy\\
% Sionex Corp.\\
% 8-A Preston Ct.\\
% Bedford, MA\\
% \end{multicols}

\vspace{10mm}

References available upon request.

\end{document}