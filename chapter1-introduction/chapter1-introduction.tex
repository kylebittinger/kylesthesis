\documentclass[12pt]{mitthesis}
\usepackage{lgrind, braket, amsmath, amssymb, bbm, booktabs, color}
\usepackage[pdftex]{graphicx}
\usepackage[version=3]{mhchem}
\pagestyle{plain}

\newcommand{\TODO} [1]{\textcolor{magenta}{\textbf{TODO:} #1}}
\newcommand{\POINT}[1]{\textcolor{magenta}{\emph{#1}}}

\hyphenation{acetylene}
\hyphenation{Hamiltonian}

\begin{document}
\tableofcontents
\clearpage

\chapter{Introduction}

\section{The importance of studying triplet states and intersystem
  crossing}

\POINT{Introduce triplet states and motivate research. Metastable
  energy carriers, easy to populate collisionally, hard to study.}

\POINT{Introduce ISC as the dynamical manifestation of mixing between
  states of different multiplicities. Make the point that energy
  enters and also leaves the triplet manifold through interaction with
  the radiation field.}

\POINT{Introduce ISC parameter $\gamma = \langle H_{sd} \rangle /
  \rho_d$}

\TODO{Go through talks, proposals and gather motivation points.}

\section{The low-lying triplet electronic states of acetylene}

\POINT{Summarize the electronic structure of the acetylene valence
  triplet states.  Show molecular orbitals and energy level diagrams.}

\begin{figure}
  \caption{Principal molecular orbital configurations for the valence
    excited states of \emph{trans}-bent acetylene.  The four
    lowest-lying electronic states are listed in energy order above
    the diagram.  An electronic transition from $S_1$ to $T_3$
    involves a change in one antibonding spin orbital, from
    $a_g^*(\downharpoonright)$ to $b_g^*(\upharpoonright)$.}
  \label{fig:mol-rbitals}

  \centering
  \setlength{\unitlength}{1cm}
  \begin{picture}(8,8)
    \put(0,1){$b_u$}
    \put(0,2){$a_u$}
    \put(0,4){$a_g^*$}
    \put(0,5){$b_g^*$}


    \put(1,1){\line(1,0){1}}
    \put(1,2){\line(1,0){1}}
    \put(1,4){\line(1,0){1}}
    \put(1,5){\line(1,0){1}}
    \put(1.333,0.666){\vector(0, 1){.666}}
    \put(1.333,1.666){\vector(0, 1){.666}}
    \put(1.666,2.333){\vector(0,-1){.666}}
    \put(1.666,3.666){\vector(0, 1){.666}}
    \put(1.333,7){$T_1$}
    \put(1.333,6){$B_u$}

    \put(3,1){\line(1,0){1}}
    \put(3,2){\line(1,0){1}}
    \put(3,4){\line(1,0){1}}
    \put(3,5){\line(1,0){1}}
    \put(3.333,0.666){\vector(0, 1){.666}}
    \put(3.333,1.666){\vector(0, 1){.666}}
    \put(3.666,1.333){\vector(0,-1){.666}}
    \put(3.666,3.666){\vector(0, 1){.666}}
    \put(3.333,7){$T_2$}
    \put(3.333,6){$A_u$}


    \put(5,1){\line(1,0){1}}
    \put(5,2){\line(1,0){1}}
    \put(5,4){\line(1,0){1}}
    \put(5,5){\line(1,0){1}}
    \put(5.333,0.666){\vector(0, 1){.666}}
    \put(5.333,1.666){\vector(0, 1){.666}}
    \put(5.666,1.333){\vector(0,-1){.666}}
    \put(5.666,4.666){\vector(0, 1){.666}}
    \put(5.333,7){$T_3$}
    \put(5.333,6){$B_u$}


    \put(7,1){\line(1,0){1}}
    \put(7,2){\line(1,0){1}}
    \put(7,4){\line(1,0){1}}
    \put(7,5){\line(1,0){1}}
    \put(7.333,0.666){\vector(0, 1){.666}}
    \put(7.333,1.666){\vector(0, 1){.666}}
    \put(7.666,1.333){\vector(0,-1){.666}}
    \put(7.666,4.333){\vector(0,-1){.666}}
    \put(7.333,7){$S_1$}
    \put(7.333,6){$A_u$}

  \end{picture}
\end{figure}

\POINT{Discuss spin-orbit coupling with trans-bent S1 state.  Show the
  spin-orbit operator and one-electron matrix elements among the
  triplet states.  The $\ell_+s_-$ operator requires a change in
  spin-orbital.  (Examples of Slater determinants: p.59 of
  11/2007--1/2008 notebook [\ce{CO2}])}

\POINT{Many ab initio studies of acetylene triplet states.  Give a
  summary of the most recent theoretical results.}

\POINT{Almost no direct experimental evidence concerning triplet
  states.  No spectra for trans-bent triplet states, although spectra
  have been recorded for the cis-well.}

\POINT{Discuss propensity/selection rules for triplet-triplet
  transitions.  No allowed T-T electronic transitions among the
  valence states in the trans well.}

\POINT{Discuss density of triplet vibrational levels as a function of
  energy.  Discussion of Ryan and Bryan's paper here?}

\section{Previous experiments involving ISC in acetylene}

\POINT{Summarize and explain the results of the Zeeman anticrossing
  experiments.} 

Main ideas:
\begin{enumerate}

\item Strong triplet-triplet interactions dominate the intramolecular
  dynamics.  Evidence: the local Land\'{e} g-factor varies from
  $0.3-2.0$.

\item The $T_1 \sim T_2$ coupling strength increases only slightly over
  the energy range of $0\nu_3'-2\nu_3'$.  Evidence: an observed
  decrease in the average Land\'{e} g-factor with energy.

\item Couplings between the $S_1$ and $S_0$ electronic states are
  induced by mixed triplet levels.  Evidence: \TODO{Find evidence for
    this.}

\item There is a rapid increase in $S_1 \sim T_3$ coupling in the range of
  $0\nu_3'-2\nu_3'$. \TODO{Cite evidence.}

\end{enumerate}
These observations led the authors to propose a heirarchy of of
interactions between the electronic states of acetylene:
\begin{equation}
  T_1 \sim T_2 \gg S_1 \sim T \gg S_0 \sim T \gg S_1 \sim S_0
\end{equation}

\POINT{Talk about the ZQB experiments of Tsuchiya.}

\POINT{Talk about earlier SEELEM experiments on acetylene.}

\POINT{Discuss observation of triplet perturbations by HB/AHS/AM in
  their investigation of bending polyads.}

\section{Summary}

In Chapter 2, we develop several frameworks for thinking about
doorway-mediated intersystem crossing.  Using these frameworks, the
relative statistical properties of LIF and SEELEM intensity
distributions are derived for several doorway-coupling model
Hamiltonians.  Moving beyond statistical properties, Chapter 3
demonstrates that a doorway-coupling model Hamiltonian can be
generated directly from spectral data by applying the standard LKL
deconvolution technique in a recursive manner.  Chapter 4 presents a
new and extended analysis of the famous $3\nu'_3$ vibrational level of
$S_1$ acetylene, highlighting experimental evidence for a dynamical
(local $+$ distant) double doorway model.  Several other vibrational
levels of $S_1$ acetylene are investigated in Chapter 5.  The complex
interplay between bending vibrations and Singlet$\sim$Triplet coupling
is apparent from the drastically different LIF/SEELEM spectra
obtained.  We discuss the ubiquitous role of $T_3$ in mediating $S_1
\sim T_{1,2}$ coupling in light of these results.  Chapter 6 presents
the results of IR-UV double resonance SEELEM experiments, which reveal
similar Singlet-Triplet coupling characteristics for the torsion and
antisymmetric in-plane bending motions.  We depart from our study of
the acetylene molecule in Chapter 7, where a new method is considered
for the manufacture of metastable molecules in a molecular beam.  The
basis of this method is two-photon optical pumping of metastable atoms
followed by collisional excitation transfer.  Chapter 8 concludes and
discusses the prospects for future research.

\end{document}