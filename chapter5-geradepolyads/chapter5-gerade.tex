\documentclass[12pt,draft]{mitthesis} 

\usepackage{lgrind, braket, amsmath,
  amssymb, bbm, booktabs, subfig, color} 

\usepackage[pdftex]{graphicx}
\usepackage[version=3]{mhchem}

\newcommand{\TODO} [1]{\textcolor{magenta}{\textbf{TODO:} #1}}
\newcommand{\NOTE} [1]{\textcolor{magenta}{[\emph{#1}]}}
\newcommand{\POINT}[1]{\textcolor{magenta}{\textbf{POINT:} #1}}

\newcommand{\rcm}{cm$^{-1}$}
\newcommand{\bigspace}{$
  \;
  $}

\newcommand{\astate}{$
  \tilde{A} \: ^1\!A_u
  $}
\newcommand{\AtoX}{$
  \tilde{A} \: ^1\!A_u 
  \leftarrow 
  \tilde{X} \: ^1\Sigma_g^+
  $}
\newcommand{\StoS}{$
  S_1 \leftarrow S_0
  $}
\newcommand{\microsec}{$\mu$s}

\hyphenation{acetylene}
\hyphenation{Hamiltonian}

\begin{document}

\tableofcontents
\clearpage

\subsubsection*{NOTES}

\clearpage

\setcounter{chapter}{3}
\chapter{SEELEM/LIF spectroscopy of acetylene: Spectral signatures of
  energetically distant doorway levels}

\section{Introduction}

The dynamics of singlet$\sim$triplet coupling in the \AtoX \bigspace
(\StoS) electronic spectrum of acetylene, \ce{C2H2}, have been
extensively studied.  This body of work establishes a heirarchical
model for electronic coupling, which leads to the effects of
intersystem crossing and internal conversion: $S_1 \rightarrow T_3
\rightarrow T_{1,2} \rightarrow S_0$.  Top-tier interactions between
the relatively sparse levels of the $S_1$ and $T_3$ electronic states
are of particular interest because (1) they control coupling to the
remaining electronic states of the molecule and (2) they are primarily
determined by vibrational overlap factors and energy denominators, two
sensitive consequents of the molecule's electronic structure.

% Singlet~triplet coupling in the S_1 state of trans-acetylene is
% mediated by the relatively sparse (0.1 per cm-1) levels of the third
% triplet state, T_3.  When a mediating T_3 level is energetically
% distant from an interacting singlet level, traces of its influence
% appear in the patterns of coupling between the singlet level and the
% dense, local manifold of T_1,2 levels (10 per cm-1).  The method of
% SEELEM spectroscopy is tuned to detect molecules in such second-tier
% eigenstates, being most sensitive to states with about 0.1\% S_1
% character.  Conversely, LIF spectroscopy is sensitive to molecules in
% states with large amounts of S_1 character.

Surface Electron Ejection by Laser-Excited Metastables \cite{sneh89a,
  sneh89b, sneh91, humphrey97} (SEELEM) and LIF spectroscopy are
complementary techniques, which, when used together, can provide
information about $S_1 \sim T_3$ interactions \cite{humphrey97,
  mishra04, altunata00, altunata02}.  \TODO{Finish adapting from
  paper.}  LIF detection is limited to short-lived
($\tau_{radiative}<$ 10 \microsec), strongly fluorescing states, while
SEELEM detection is sensitive only to long-lived ($\tau >$ 300
\microsec) states with vertical electronic excitation above a
threshold energy set by the work function of the metal used as the
SEELEM detector surface. Therefore, SEELEM and LIF detection channels
observe mutually exclusive sets of eigenstates that arise from
spin-orbit mixed $S_1$ and $T_{3,2,1}$ zero-order basis states.
Because SEELEM detection is sensitive to the electronic character of
the molecule, a comparison of simultaneously recorded SEELEM and LIF
spectra reveals features of electronic structure and photochemical
pathways that are invisible via traditional, single-channel
spectroscopic probes such as LIF alone, REMPI, phosphorescence, or
phosphor surface \cite{shi98, campos01, burton72}.

% Information gained from comparison of acetylene SEELEM
% and LIF spectra can yield a mechanistic description of
% singlet$\sim$triplet interaction, and holds promise for describing the
% structure and dynamics of other small polyatomic species \cite{jung07,
%   allen07}.
% SEELEM spectroscopy has
% been used in this effort as a tool to examine $S_1 \sim T_3$ coupling,
% by allowing detection of the local manifold of predominantly $T_{1,2}$
% eigenstates in the region surrounding an \StoS\ transition.

The \emph{trans}-bending mode of $S_1$ acetylene, $\nu_3$, is known to
be an important promoter of singlet$\sim$triplet coupling.  In Zeeman
anticrossing experiments, Dupr\'{e} and coworkers observed a rapid
increase in the anticrossing density, as well as the product
$\rho_{\text{vib}} \braket{H_{st}}$, with energy in $\nu_3$
\cite{dupre91, dupre95b}.  They also observed an single large
singlet$\sim$triplet anticrossing in the $3 \nu_3$ $K_a=0$ level,
%with a zero-field matrix element of 0.29 \rcm,
which was in turn perturbed by many smaller couplings \cite{dupre93}.
These observations led the authors to propose that the
singlet$\sim$triplet dynamics of acetylene in the \astate\ state are
best described by a doorway-mediated mechanism, in which particular
triplet vibrational levels mediate intersystem crossing between the
initially excited $S_1$ bright state and the dense manifold of highly
mixed, dark $T_{1,2}$ states.  Subsequent theoretical and experimental
work confirmed that the coupling doorways were vibrational levels of
the $T_3$ electronic state \cite{vacek96, sherrill96, humphrey97,
  altunata00}.

The $S_1$ $3 \nu_3$ $K_a$=1 level has been heavily studied due to a
local $S_1 \sim T_3$ level crossing at $J \approx 3$.  Spectroscopic
patterns in SEELEM arising from the effects of the local $T_3$
perturber in have been discussed by several authors \cite{humphrey97,
  altunata00, altunata01, mishra04}.  Using vibrational overlap
integrals gained from \emph{ab initio} calculations of the $T_3$
electronic surface, Thom and coauthors were able to exclude all but
several candidate $T_3$ levels as the $3\nu_3$ local perturber
\cite{thom07}.
% \POINT{Overlap between $S_1$ and $T_3$ levels predicted by Bryan and
%   Ryan.  (See p.40 of 1/2007--3/2007 notebook.)}
However, recent observations of an increase in average $T_{1,2}$
electronic character at higher values of $J$ suggest that the local
$T_3$ perturber is not the sole, and perhaps not the primary, doorway
for $3 \nu_3$ \cite{degroot07}.

That an energetically distant doorway, which would be required to have
a correspondingly larger spin-orbit matrix element, may play a role in
coupling $S_1$ $3 \nu_3$ to the $T_{1,2}$ manifold is not altogether
surprising, given the energy region in question.  \emph{Ab initio}
calculations are in agreement that an electronic surface crossing
between the $S_1$ and $T_3$ states is energetically nearby
\cite{ventura03, thom07}.  Such a surface crossing would allow for
strong interactions with several $T_3$ levels in the same energy
region.  It should also play a role in promoting singlet$\sim$triplet
coupling within other $S_1$ levels in the same energy region.  The
$4\nu_3$ level, 1000 \rcm\ higher in energy, is also strongly coupled
to the $T_{1,2}$ manifold, although no obvious local $T_3$ doorway has
been observed \cite{drabbels93, ochi91}.

The energies of the $3 \nu_3$ and $4\nu_3$ levels can therefore be
taken as conservative lower and upper bounds for such an $S_1 \sim
T_3$ ``active'' coupling region.  We turn our attention to other
vibrational levels of $S_1$, which, while in this critical energy
region, are not near-degenerate with a mediating $T_3$ level at low
$J$.  The levels chosen for study all lie within 500 \rcm\ of $3
\nu_3$ $K_a$=1, and involve excitation in the Franck-Condon active
modes of $S_1$ acetlyene, $\nu_2$ (C-C stretch) and $\nu_3$
(\emph{trans} bend), with one exception.

In the absence of a local $T_3$ perturber, coupling between $S_1$
levels and the local manifold of $T_{1,2}$ states is expected to be
mediated by energetically distant $T_3$ levels.  Evidence for
energetically distant, mediating $T_3$ levels is obtained by comparing
simultaneously recorded LIF and SEELEM spectra.  Additionally, a new
technique for comparison is developed that is sensitive to the
presence of distant $T_3$ doorway levels.  This new technique takes
into account shifting intensity patterns in the frequency-domain LIF
spectrum as a function of delay time, due to the time development of
an incoherent ensemble of eigenstates with different fluorescence
lifetimes.  When used on a segment of spectrum that includes just one
singlet rovibronic transition, the technique reveals a
coupling-dependent shift in the statistical properties of the spectrum
with delay time.  Comparison with other transitions in a rotational
series can reveal local perturbations caused by weakly coupled $T_3$
levels.  When used on a spectrum containing several rovibronic
transitions in series, the technique can give insight into the matrix
element and relative energy of a $T_3$ doorway level.

\section{COMMUNICATION: Patterns of spectral intensity in delayed,
  incoherent fluorescence measurements}

In Chapter 2, we introduced the concept of a SEELEM intensity
distribution associated with a singlet bright state.  Here, we extend
the idea of dark state intensity distributions to fluorescence
measurements by considering intensities in the \emph{incoherent,
  frequency domain} LIF spectrum at a time window $t+dt$ after
excitation of the molecule.  The SEELEM spectrum is shown to be an
extreme, limiting case of delayed fluorescence.  Furthermore, we
demonstrate that molecular properties such as coupling matrix element
and the energy denominator for a mediating doorway state are revealed
by changes in statistical properties of the spectrum as it changes
from sub-\microsec\ gated LIF, into delayed LIF, and finally to
SEELEM.

\TODO{Adapt following from delayedfluorescence report.}  For a pure
singlet bright state $\ket{s}$, the time-dependent fluorescence
intensity is
\begin{equation}
  I_s(t) = \frac{1}{\tau_s} \;
           \exp \left[
             -\frac{t}{ \tau_s} 
           \right],
\end{equation}
normalized such that $\int_0^{\infty} I_s(t) \, dt = 1$.

We consider the case where $\ket{s}$ is directly coupled with a set of
$N$ triplet dark states to create a set of $N$+1 mixed states
$\lbrace\ket{m}\rbrace$.  Each state $\ket{m}$ has some bright state
amplitude $a_m$.  In the case of direct or doorway-mediated coupling,
the Hamiltonian contains only one pathway from each dark state to the
bright state (see Chapter 3).  As a result, the mixing amlitudes may
be taken to be real and positive by convention.  If the lifetime of a
pure triplet dark state is much longer than $\tau_s$, the lifetime of
a mixed eigenstate $\ket{m}$ having bright state character
$\alpha_m^2$ is
\begin{equation}
  \label{eq:tau-m}
  \tau_m = \tau_s / a_m^2,
\end{equation}
and its time-dependent fluorescence intensity is
\begin{equation}
  \label{eq:int-m}
  I_m(t) = \frac{a_m^4}{\tau_s} \;
           \exp \left[
             -\frac{a_m^2 \, t}{\tau_s} 
           \right].
\end{equation}
The total integrated fluorescence intensity for a mixed state is
$\int_0^{\infty} I_m(t) \, dt = a_m^2$, relative to unit intensity for
a pure bright state.

The normalization condition for bright state character $\sum_m a_m^2 =
1$ leads to the relation
\begin{equation}
  \tau_s^{-1} = \sum_m \tau_m^{-1};
\end{equation}
in this way, $\tau_s$ can be derived from the set of mixed state
lifetimes $\lbrace \tau_m \rbrace$.

\subsection{SEELEM detection as an extreme limit of delayed fluorescence}

We examine the fluorescence intensity after a time delay $t_c$,
which we cast in units of the bright state lifetime:
\begin{equation}
  R_c = t_c / \tau_s.
\end{equation}
At a chosen $R_c$, the fluorescence intensity equation has a single
peak according to some value of bright state character.  Within the
range $0 \le a_m^2 \le 1$, the fluorescence intensity equation is at a
maximum when (Chapter 2, equation 31)
\begin{equation}
  \label{eq:am-max}
  a_m^2 = \frac{2}{R_c}.
\end{equation}

\begin{figure}
  \caption{The intensity of a mixed eigenstate at time $R_c =
    t/\tau_s$, plotted as a function of bright
    state character.}
  \label{fig:int-at-rc}
  \centering
  \includegraphics[width=7.5in,angle=90]{intensity-at-rc.png}
\end{figure}

Figure \ref{fig:int-at-rc} shows the dependence of fluorescence
intensity on bright state character for several values of $R_c$.  At
early fluorescence times ($\sim \tau_s$), the fluorescence intensity
is greatest for states with large amounts of bright state character
($2/R_c > 1$).  At a delay of $5 \tau_s$, the fluorescence intensity
equation discriminates against states with large amounts of bright
state character, since molecules in such states have already
fluoresced with high probability.  States with small amounts of bright
state character are also discriminated against -- molecules in these
states have a low probability of fluorescing at the time under
consideration.  The fluorescence intensity is thus ``tuned'' to a
particular value of bright state character at every time delay $R_c$,
according to Equation \ref{eq:am-max}.

Note that the intensity expression used here does not account for
molecules leaving the field of view of the fluorescence detection
optics.  We consider only the distribution of bright state character
\emph{among molecules remaining in the field of view at time $R_c$}.
However, since there is no momentum transfer from the excitation
photon to the molecule, the rate of molecules exiting the field of
view is independent of bright state character.  Therefore,
simple reasoning according to the above intensity formulas will be
correct as long as we discuss fluorescence intensities only in terms
of ratios.

To foster a more concrete discussion, consider our experiments on
intersystem crossing in acetylene.  For the $S_1$ electronic state,
$\tau_s=270$ ns.  In our apparatus, the field-of-view of the
fluorescence detection optics is several mm, which amounts to a
maximum viewing time of 3 $\mu$s, about $10\tau_s$, for molecules in
the molecular beam with a velocity of $10^3$ m/s.  At times later than
$10\tau_s$, there are simply no molecules left to observe in the
fluorescence field of view.  This places an upper limit on the value
of $R_c$ that can be examined in our fluorescence experiments.  Figure
\ref{fig:int-at-rc} shows the intensity equation for the limit of
fluorescence detection, $10\tau_s$.

The SEELEM detector used in our experiments detects metastable
molecules after a 309 $\mu$s flight time.  If we set aside some
particularly interesting aspects of SEELEM detectivity and consider
only its detection sensitivity to bright state character, we arrive at
the following equation for SEELEM detection probability (Chapter 2,
equation 28):
\begin{equation}
  \label{eq:seelem-prob-s}
  P_{SEELEM}^{(s)} \propto a_m^4 \; \exp \left( -R_c \, a_m^2 \right).
\end{equation}
(This is a good approximation to its overall detection sensitivity,
including $T_3$, in the weak coupling limit.)  The SEELEM detection
probability equation above has the same functional form as the
Equation \ref{eq:int-m}.  Thus, the SEELEM detector may be viewed in
this limited sense as an extreme discriminator for states with small
amounts of bright state character.

Acetylene molecules in our apparatus are detected after an average
flight time of 309 $\mu$s, yielding an $R_c$ value of $1144$ for the
SEELEM detector.  This corresponds to a maximum detection probability
for states with $0.17\%$ bright state character.

\subsection{Delayed fluorescence of a two-level system}

Our next step is to examine the changing characteristics of a
fluorescence intensity distribution as we increase the time delay
$R_c$.  We model the interaction between a bright state and an
ensemble of dark states by generalizing from a two-level system.
Consider a Hamiltonian with only one dark state.  Define the two mixed
states $\ket{m=1,2}$ as
\begin{equation}
  \begin{split}
    \ket{1} &=  (1 - \alpha^2)^{1/2} \ket{s} + \alpha \ket{t}\\
    \ket{2} &= -\alpha \ket{s} + (1 - \alpha^2)^{1/2} \ket{t}.
  \end{split}
\end{equation}
The fluorescence intensity of both states, relative to a pure bright
state, is
\begin{equation}
  \begin{split}
    I_1 &= \frac{(1 - \alpha^2)^2}{\tau_s} \; \exp 
          \left[
            - R_c \, (1 - \alpha^2)
          \right]\\
    I_2 &= \frac{\alpha^4}{\tau_s} \; \exp 
          \left[
            - R_c \, \alpha^2
          \right],
  \end{split}
\end{equation}
and the ratio of the fluorescence intensities changes with time as
\begin{equation}
  I_{12} = I_1 / I_2 = 
  \left(
    \frac{(1 - \alpha^2)}{\alpha^2}
  \right)^2
  \exp
  \left[
    - R_c \, (1 - 2\alpha^2)
  \right].
\end{equation}
The fluorescence intensity ratio may be used to write an equation for
the dependence of average intensity on $R_c$.
\begin{equation}
  \label{eq:ratio}
  \braket{I_{LIF}} = 
  \frac{\Delta E_{12}}{2} \,
  \left(
    \frac{I_{12}-1}{I_{12}+1}
  \right)
\end{equation}
Figure \ref{fig:ratio} shows the dependence of center-of-gravity on
the intensity ratio $I_{12}$.

\begin{figure}
  \caption{Center of gravity as a function of the intensity ratio
    $I_1$:$I_2$ for a two-state system.  At $t=0$, the initial center
    of gravity is a function of the mixing amplitude $\alpha$.  The
    center of gravity then shifts with the changing intensity ratio as
    delay time is increased, arriving at a limiting value of $-\Delta
    E_{12}/2$.}
  \label{fig:ratio}
  \centering
  \includegraphics[width=6in]{cog-from-ratio.png}
\end{figure}

The initial intensity ratio at time $R_c=0$ is
\begin{equation}
  I_{12} = 
  \left(
    \frac{(1 - \alpha^2)}{\alpha^2}
  \right)^2.
\end{equation}
From this, the initial center of gravity can be found according to
Equation \ref{eq:ratio} or Figure \ref{fig:ratio}.  At large values of
$R_c$, the intensity ratio decreases to zero as $I_1 \ll I_2$, if the
states are not 50:50 mixed.  As the ratio decreases to zero, the
center of gravity approaches $-\Delta E_{12} / 2$.  The development of
intensity ratios and the consequent shift in center of gravity are
shown in Figure \ref{fig:cog-devel}.  One interesting aspect of the
center of gravity shift shown in the figure is the rapid onset of the
shift at small mixing coefficients.

\begin{figure}
  \caption{(Top) Time development of the relative intensity ratio for
    a two state system. (Bottom) Time development of the center of
    gravity for a two state system.}
  \label{fig:cog-devel}
  \centering
  \includegraphics[width=6in]{ratio-development.png}
  \includegraphics[width=6in]{cog-development.png}
\end{figure}

\subsection{Delayed fluorescence of a series of rotational
  transitions}

Information about the relative energy of a mediating level can be
recovered by studying the statistical properties of delayed
fluorescence as a function of rotational quantum number.  For a
near-symmetric prolate top in Hund's case ($b$), the selection rules
for spin-orbit coupling are \cite{stevens73}
\begin{equation}
  \begin{split}
    \Delta J &= 0 \\
    \Delta N &= 0, \pm 1\\
    \Delta K_a &= 0, \pm 1.
  \end{split}
\end{equation}
The $\Delta N$ selection rule gives rise to three rotational
components of a dark triplet state $\ket{\ell}$ coupled to a singlet
bright state $\ket{s;N}$ : $F_1$, $\ket{\ell;N-1}$; $F_2$,
$\ket{\ell;N}$; and $F_3$, $\ket{\ell;N+1}$.  The relative energies of
the three components are, in terms of the singlet rotational quantum
number $J$,
\begin{equation}
  \label{eq:components}
  \Delta E_{s\ell}(J) = (E_{s} - E_{\ell}) +
  \begin{cases}
    \Delta B_{s\ell}J(J+1) + 2B_{\ell}J           
    & F_1 \text{ component}\\
    \Delta B_{s\ell}J(J+1)                      
    & F_2 \text{ component}\\
    \Delta B_{s\ell}J(J+1) - 2B_{\ell}J - 2B_{\ell} 
    & F_3 \text{ component}.\\
  \end{cases}
\end{equation}
In the approximation that $B_s \approx B_{\ell}$, the relative
energies of the $F_{1,2,3}$ components change linearly when plotted
against $J$.  The relative energy of the $F_2$ component has a slope
of approximately zero, while the $F_1$ and $F_3$ components have a
slope of approximately $\pm2B$.  These relative energy differences are
presented in Figure \ref{fig:components}.

\begin{figure}
  \caption{ Reduced term value plot of the three singlet$\sim$triplet
    roational components permitted by the spin-orbit operator in
    Hund's case ($b$).  The $F_1$ component corresponds to  }
  \label{fig:components}
  \centering
  \includegraphics[width=6in]{f-components.png}
\end{figure}

Based on the existing assignment of a local $T_3$ perturber in $S_1$
$3 \nu_3$ $K_a$=1, as well as $T_3$ geometries prediced by \emph{ab
  initio} calculations, we expect the product $\Delta B_{s\ell}J(J+1)$
to remain small compared to $2BJ$ for any combination of $S_1$ and
$T_3$ vibrational levels \cite{mishra04, ventura03, thom07}.
\POINT{The figure is a good approximation to what we expect from the
  molecule.}

To a very good approximation the total first-order spin-orbit matrix
element between two rovibrational states is given by the product of
three factors: an electronic spin-orbit matrix element, a vibrational
overlap factor, and a rotational factor arising from angular momentum
coupling rules.  The rotational factors are given for the general case
of polyatomic molecules by Stevens and Brand \cite{stevens73}.  Plots
of the spin-orbit rotational factors are presented for singlet levels
having $K$=0 (Figure \ref{fig:rotational-factors-0}), $K$=1 (Figure
\ref{fig:rotational-factors-1}), and $K$=2 (Figure
\ref{fig:rotational-factors-2}).  With the exception of the $\Delta N
= \Delta K = 0$ components, the rotational factors quickly approach
their asymptotic limits, usually between 0.1 and 0.6.  Even at low
values of $J$, their variation with rotational quantum number is
always less than a factor of 2.  \POINT{These variations are trumped
  by energy denominator and vibrational overlap effects.}

\begin{figure}
  \caption{Rotational factors for spin-orbit coupling in polyatomic
    molecules, $K_s$=0}
  \label{fig:rotational-factors-0}
  \centering
  \includegraphics[width=6in]{rotational_factors_k0.png}
\end{figure}

\begin{figure}
  \caption{Rotational factors for spin-orbit coupling in polyatomic
    molecules, $K_s$=1}
  \label{fig:rotational-factors-1}
  \centering
  \includegraphics[width=6in]{rotational_factors_k1.png}
\end{figure}

\begin{figure}
  \caption{Rotational factors for spin-orbit coupling in polyatomic
    molecules, $K_s$=2}
  \label{fig:rotational-factors-2}
  \centering
  \includegraphics[width=6in]{rotational_factors_k2.png}
\end{figure}



The effects of widely varying vibrational overlap factors will divide
$S_1 \sim T_3$ perturbations into two classes: those where
$H_{st}/\Delta E_{st} \ll 2B$, and those where $H_{st} / \Delta
E_{st}$ is on the order of $2B$.  Perturbations in the first class
will appear only once in the spectrum.  However, once a triplet level
with small vibrational overlap is found at a particular $J$, its
energy above or below the perturbed singlet level may be found
according to Equation \ref{eq:components}.

Perturbations falling into the second class will affect the
singlet$\sim$triplet coupling at several values of $J$; these are the
distant $T_3$ doorways with which we are most concerned.  \TODO{Plot
  mixing coefficients of distant doorways as a fcn of rotational
  quantum number.  Trend will be: weakly coupled states turn on/off
  suddenly, strongly coupled states are visible for several values of
  $N$.}

\section{Experiment}

\TODO{Adapt this from paper.}  SEELEM is a versatile and sensitive
technique for investigating ``dark'' (weakly-fluorescing) metastable
molecules produced via laser excitation.1-6 In the SEELEM experiment,
a molecular beam of acetylene is excited by a $\sim$5 ns pulsed laser
into spin-rotation-vibration eigenstates of metastable electronic
states via weak, nominally forbidden transitions. After excitation,
the long-lived species must travel 35 cm before colliding with an Au
metal detector surface, where an electron is ejected in a
de-excitation process. Two criteria must be met for electron ejection
by a metastable species. First, the vertical electronic energy of the
metastable approaching the surface must exceed the work function of
the metal ($\Phi$Au = 5.1 eV). Second, the radiative lifetime of the
detected metastable eigenstate ($\tau_\text{rad}$) must exceed the
flight time from the point of laser excitation to the SEELEM surface
($\Delta$t=300 $\mu$s).

A sample of acetylene (BOC gases) at a backing pressure of one
atmosphere was pulsed through a 0.5 mm diameter nozzle operating at 10
Hz into a diffusion pumped vacuum chamber at $\sim$5x10-5 torr.  An Nd:YAG
pumped, frequency-doubled dye laser (220 nm) excited the acetylene
molecules in the pulsed jet expansion 2 cm downstream from the nozzle
orifice. UV-LIF was detected perpendicular to the plane defined by the
intersection of the pulsed molecular and laser beams using f/1.2
collection optics, a fluorescence filter (UG-11) to reduce scattered
laser light, and a PMT (Hamamatsu model R375). The fluorescence signal
was averaged by a boxcar integrator and recorded. For SEELEM
detection, the excited molecules in the pulsed expansion passed
through a conical skimmer (3mm diameter) to form a collimated
molecular beam, which traveled into a differentially pumped detector
chamber maintained at $\sim$4x10-7 torr, and collided with a heated (300
C) Au metal surface 35 cm downstream from the point of laser
excitation. The SEELEM detector was identical to that used in the
previously described apparatus with Au foil ($\Phi$ = 5.1 eV) as the
metal surface.  Particle counting techniques, including a multichannel
scalar (Oxford Tennelec Nucleus Inc. MCS-II v2.091) were used to
record laser-excited metastable counts as a function of laser
frequency, along with the simultaneously recorded LIF spectra. Both
SEELEM and LIF signals were averaged over 100 laser shots / data
point.

\section{Results}

\POINT{SEELEM spectrum of $2^2 3^1$ (Oct/Nov 2006, see ``similitude''
  calculations, p.62 of Sep 2006--Jan 2007 notebook.)}  This spectrum
is shown in Figure \ref{fig:spectrum-2231}.  For this band, several
scans were repeated with finer frequency steps.  Figure
\ref{fig:spectrum-2231-q123} shows the first three transitions of the
Q-branch.  Figure \ref{fig:spectrum-2231-q1r0} shows the two
transitions, Q(1) and R(0).  These two transitions have the same upper
state rotational quantum number ($J'=1$), but differ in parity (having
$f$- and $e$-symmetry, respectively).
% Selection rules: Q   -> e-f
%                  P,R -> e-e, f-f

\begin{figure}
  \caption{
    % Simultaneously recorded LIF and SEELEM spectra of the
    % acetylene $V^2_04^2_0K^1_0$ $\tilde{A}^1A_u \leftarrow
    % \tilde{X}^1\Sigma_g$ transition.
    Simultaneously recorded surface electron ejection by laser excited
    metastables (SEELEM, upper trace) and ultraviolet laser-induced
    fluorescence (UV-LIF, lower trace) spectra of the $2^23^1$ $K_a$=1
    sublevel of the $\tilde{A}^1A_u \leftarrow \tilde{X} ^1\Sigma_g^+$
    electronic transition. A delayed, integrated fluorescence signal
    is shown as a dotted trace in the UV-LIF spectrum.}
  \label{fig:spectrum-2231}
  \centering
  \includegraphics[width=8in,angle=90]{spectrum-2231.png}
\end{figure}

\begin{figure}
  \caption{
    % Simultaneously recorded LIF and SEELEM spectra of the
    % acetylene $V^2_04^2_0K^1_0$ $\tilde{A}^1A_u \leftarrow
    % \tilde{X}^1\Sigma_g$ transition.
    Simultaneously recorded surface electron ejection by laser excited
    metastables (SEELEM, upper trace) and ultraviolet laser-induced
    fluorescence (UV-LIF, lower trace) spectra of the $2^23^1$ $K_a$=1
    sublevel of the $\tilde{A}^1A_u \leftarrow \tilde{X} ^1\Sigma_g^+$
    electronic transition. A delayed, integrated fluorescence signal
    is shown as a dotted trace in the UV-LIF spectrum.}
  \label{fig:spectrum-2231-q123}
  \centering
  \includegraphics[width=6in]{spectrum-2231-Q123.png}
\end{figure}

\begin{figure}
  \caption{
    % Simultaneously recorded LIF and SEELEM spectra of the
    % acetylene $V^2_04^2_0K^1_0$ $\tilde{A}^1A_u \leftarrow
    % \tilde{X}^1\Sigma_g$ transition.
    Simultaneously recorded surface electron ejection by laser excited
    metastables (SEELEM, upper trace) and ultraviolet laser-induced
    fluorescence (UV-LIF, lower trace) spectra of the $2^23^1$ $K_a$=1
    sublevel of the $\tilde{A}^1A_u \leftarrow \tilde{X} ^1\Sigma_g^+$
    electronic transition. A delayed, integrated fluorescence signal
    is shown as a dotted trace in the UV-LIF spectrum.}
  \label{fig:spectrum-2231-q1r0}
  \centering
  \includegraphics[width=6in]{spectrum-2231-Q1R0.png}
\end{figure}

\POINT{SEELEM spectrum of $2^1 3^2$ P, Q-branch (See Jan 16A+B,
  p.124--127 of 9/2006--1/2007 notebook, also assignments on p.2 of
  1/2007--3/2007 notebook.)}  This spectrum is shown in Figure
\ref{fig:spectrum-2132}.

\begin{figure}
  \caption{
    % Simultaneously recorded LIF and SEELEM spectra of the
    % acetylene $V^2_04^2_0K^1_0$ $\tilde{A}^1A_u \leftarrow
    % \tilde{X}^1\Sigma_g$ transition.
    Simultaneously recorded surface electron ejection by laser excited
    metastables (SEELEM, upper trace) and ultraviolet laser-induced
    fluorescence (UV-LIF, lower trace) spectra of the $2^13^2$ $K_a$=1
    sublevel of the $\tilde{A}^1A_u \leftarrow \tilde{X} ^1\Sigma_g^+$
    electronic transition. A delayed, integrated fluorescence signal
    is shown as a dotted trace in the UV-LIF spectrum.}
  \label{fig:spectrum-2132}
  \centering
  \includegraphics[width=8in,angle=90]{spectrum-2132.pdf}
\end{figure}


\POINT{SEELEM spectrum of $3^3$ $K=2$ hot band (See Jan 22C, p.31,34
  of 1/2007--3/2007 notebook.)}  This spectrum is shown in Figure
\ref{fig:spectrum-33k2}.

\begin{figure}
  \caption{
    Simultaneously recorded surface electron ejection by laser excited
    metastables (SEELEM, upper trace) and ultraviolet laser-induced
    fluorescence (UV-LIF, lower trace) spectra of the $3^3$ $K_a$=2
    sublevel of the $\tilde{A}^1A_u \leftarrow \tilde{X} ^1\Sigma_g^+$
    electronic transition. A delayed, integrated fluorescence signal
    is shown as a dotted trace in the UV-LIF spectrum.}
  \label{fig:spectrum-33k2}
  \centering
  \includegraphics[width=8in,angle=90]{spectrum-33k2.png}
\end{figure}

\POINT{SEELEM spectrum of $3^2 4^2$ with lone SEELEM peak in band gap
  (December 2006, p.86 of 9/2006--1/2007 notebook.)}  This spectrum is
shown in Figure \ref{fig:spectrum-32b2}.

\begin{figure}
  \caption{
    % Simultaneously recorded LIF and SEELEM spectra of the
    % acetylene $V^2_04^2_0K^1_0$ $\tilde{A}^1A_u \leftarrow
    % \tilde{X}^1\Sigma_g$ transition.
    Simultaneously recorded surface electron ejection by laser excited
    metastables (SEELEM, upper trace) and ultraviolet laser-induced
    fluorescence (UV-LIF, lower trace) spectra of the $3^24^2$ $K_a$=1
    sublevel of the $\tilde{A}^1A_u \leftarrow \tilde{X} ^1\Sigma_g^+$
    electronic transition. A delayed, integrated fluorescence signal
    is shown as a dotted trace in the UV-LIF spectrum.}
  \label{fig:spectrum-32b2}
  \centering
  \includegraphics[width=7.5in,angle=90]{spectrum-32b2.png}
\end{figure}

\section{Analysis: LIF/SEELEM intensity distributions}

\subsection{The $2^23^1$ $K_a$=1 sublevel}

\TODO{Get trace data from MIT.}

\subsection{The $2^13^2$ $K_a$=1 sublevel}

\TODO{Import calibration points from IGOR.}

Figure \ref{fig:2132-qbranch-cog-delay} shows the shifting center of
gravity of a series of rotational lines Q(1-5) in the LIF spectrum of
the $2^13^2$ $K_a$=1 sublevel.

\begin{figure}
  \caption{Shifting center of gravity of a series of rotational lines
    Q(1-5) in the LIF spectrum of the $2^13^2$ $K_a$=1 sublevel.}
  \label{fig:2132-qbranch-cog-delay}
  \centering
  \includegraphics[width=6in]{2132-qbranch-cog-delay.pdf}
\end{figure}

Figure \ref{fig:2132-indiv-cog-delay} shows the shifting center of
gravity of individual transitions Q(1-5) in the LIF spectrum of
the $2^13^2$ $K_a$=1 sublevel.

\begin{figure}
  \caption{Shifting center of gravity of a series of individual
    transitions Q(1-5) in the LIF spectrum of the $2^13^2$ $K_a$=1
    sublevel.}
  \label{fig:2132-indiv-cog-delay}
  \centering
  \includegraphics[width=6in]{2132-indiv-cog-delay.pdf}
\end{figure}

Figure \ref{fig:2132-indiv-var-delay} shows the increasing variance of
individual transitions Q(1-5) in the LIF spectrum of the $2^13^2$
$K_a$=1 sublevel.  \POINT{Huge difference between $J'=1,2$ and
  $J'=3,4,5$ is consistent with coupling to an energetically distant
  doorway with $K_a$=2 via the $F_1$ component.}

\begin{figure}
  \caption{Spectral variance plotted as a function of LIF delay time
    for a series of individual transitions Q(1-5) in the LIF spectrum
    of the $2^13^2$ $K_a$=1 sublevel.}
  \label{fig:2132-indiv-var-delay}
  \centering
  \includegraphics[width=6in]{2132-indiv-var-delay.pdf}
\end{figure}

\subsection{The $3^3$ $K_a$=2 sublevel}

\TODO{Import calibration points from IGOR.}

Figure \ref{fig:33k2-cog-delay} shows the shifting center of gravity
of a series of rotational lines R(1-7) in the LIF spectrum of the
$3^3$ $K_a$=2 sublevel.

\begin{figure}
  \caption{Shifting center of gravity of a series of rotational lines
    R(1-7) in the LIF spectrum of the $3^3$ $K_a$=2 sublevel.}
  \label{fig:33k2-cog-delay}
  \centering
  \includegraphics[width=6in]{33k2-cog-delay.png}
\end{figure}

\subsection{The $3^24^2$ $K_a$=1 sublevel}

\TODO{Adapt from email.}  Consider the R-branch of this transition.  A
look at the long vs. short gates shows a systematic shift to higher
energy in the delayed (long) gate spectrum.  Here is what happens when
the data is examined using all 500 time divisions available
\ref{fig:32b2-cog-delay}.

\begin{figure}
  \caption{Shifting center of gravity of individual rotational
    lines in the LIF spectrum of the $3^24^2$ $K_a$=1 sublevel.}
  \label{fig:32b2-cog-delay}
  \centering
  \includegraphics[width=6in]{32b2-cog-delay.png}
\end{figure}

The systematic shift to higher energies at greater time delays is
evident for all the lines, EXCEPT R(1).  In a somewhat peculiar
manner, R(1) begins to shift upward in energy, but then shifts
downward at later times. I believe this to be evidence of a local
perturbation, probably a T3 level which happens to have a small matrix
element with the $3^2 4^2$ level.  (Since the perturbation occurrs
only at $J'$=2, it most likely involves a change in $N$ of $\pm$1 --
see Ryan and Adya's paper for details.)  Looking at the SEELEM
spectrum near the R(1) transition, you'll notice that it is the
strongest feature in the band.  A local $T_3$ perturber which is
fractionated among $T_1$ and $T_2$ levels would explain this strong,
anamolous feature in the SEELEM spectrum.

Although I did not write out the theory for this in my report, I've
also plotted out the variance of the spectral features as a function
of time \ref{fig:32b2-var-delay}

\begin{figure}
  \caption{Shifting variance of individual rotational lines in the LIF
    spectrum of the $3^24^2$ $K_a$=1 sublevel.}
  \label{fig:32b2-var-delay}
  \centering
  \includegraphics[width=6in]{32b2-variance-delay.png}
\end{figure}

All of the transitions exhibit similar behavior, increasing in
spectral width as our detection method becomes more and more sensitive
to states with smaller amounts of bright state character.  The R(1)
transition may be increasing in variance slightly faster due to the
proposed local interaction, but I hesitate to hang too much weight on
that due to intensity noise.

\section{Discussion}

\TODO{Adapt from email, on the subject of $3^2 4^2$ $K_a$=1.}  The
analysis indicates that the coupling between S1 ($3^2 4^2$) and the
local manifold of T1,2 levels is induced by a non-local mediating
(probably T3) level at higher energy.  If the coupling were
non-mediated, the manifold of T1,2 levels would mix according to their
energy denominators, and we would not observe a systematic shift in
the center of gravity for all transitions in the R-branch.  The
analysis also shows a local perturbation at J'=2.

\TODO{Write this section.}

\section{Conclusion}

The mechanism of doorway-mediated coupling by an energetically distant
$T_3$ level skews the coupling to the local $T_{1,2}$ states appearing
in the SEELEM spectrum, resulting in a center-of-gravity shift between
the LIF and SEELEM spectra.  Additionally, when viewed in successive
time windows, the center of gravity of the LIF spectrum shows a
transition to the limiting behavior exhibited in the SEELEM spectrum.
A simple model can be used to show that strong coupling between the
singlet level and the mediating $T_3$ level causes a gradual shift in
the center of gravity, while weak coupling to the mediating level
induces a more delayed and rapid shift in the center of gravity.

Rotational selection rules for $S_1 \sim T_3$ spin-orbit coupling give
rise to $J$-dependent effects in the LIF/SEELEM spectrum.  As $J$
increases, the $\Delta N$= +1 or -1 components of any distant $T_3$
level approach the singlet at a rate of approximately $2B_T$ (about
2\rcm\ per $J$).  In the spectrum, the result is a shift in the
LIF-SEELEM center of gravity when integrated across an entire branch
of transitions.  The effect not only leads to strong changes in
$T_3$-mediated coupling with rotation, but also ensures "accidental"
$S_1 \sim T_3$ near-degeneracies at relatively low values of $J$.

% \POINT{The LIF/SEELEM spectra for the progression of $2^n3^m$ levels
%   shows the expected trends for strong coupling.}

The universality of triplet perturbations via $F_1$ and $F_3$
components of $T_3$ levels in the spectrum of acetylene ensures that
further LIF/SEELEM spectroscopy of the \astate\ state will be fruitful
and informative.  Primary candidates for further investigations are
(1) levels exhibiting long lifetimes or quantum beats in the LIF
spectrum, (2) levels with unassigned perturbations or splittings, (3)
other $3^2B^2$ polyad members, (4) other $K$-sublevels of the
Franck-Condon active levels studied here.

\bibliography{master}
\bibliographystyle{plain}
\end{document}




President's Report:

The T_3 electronic state is peculiar among the valence states of
acetylene due to its out-of-plane equilibrium geometry. One of the
most interesting unsolved problems in the intersystem crossing of
acetylene is the role of the non-symmetric in- and out-of-plane S_1
bending modes (\nu_4 and \nu_6) in promoting coupling to the nonplanar
T_3 state.  To investigate this issue, we have used an IR-UV-Double
Resonance technique to record simultaneous LIF/SEELEM spectra of the
3^3 4^1 K=0 and 3^3 6^1 K=0 levels.  This method allows us to record
the spectrum of weakly coupled T_1,2 levels without overlap from
adjacent rotational transitions.

[Additionally, we have used a polarization technique in tandem with
IRUVDR to record the spectrum of the rotationless level in 3^3 6^1
K=0.  Due to spin-orbit selection rules, only triplet levels having
K_T = J_T = N_T = 0 may appear in the spectrum -- thus, the local
T_1,2 level density may be obtained from a direct count of SEELEM
transitions.]


That should be K_T = J_T = 0 and N_T = 1...my bad. For a K=0 triplet
level in Hund's case (b),

N            J

  ---------- 3
2 ---------- 2
  ---------- 1


  ---------- 2
1 ---------- 1
  ---------- 0  ***


0 ---------- 1

--Kyle



The total spin-orbit matrix element between a rovibronic level of S_1
and a level of T_3 is the product of an electronic factor, a
vibrational overlap factor, and a rotational factor (arising from
angular momentum coupling laws).

OK, these are kind of hairy, but hold on before you start to panic.
As it turns out, the energy differences between levels of S_1 and T_3
vary much more rapidly than the rotational factors.  This is due to
the spin-orbit selection rule of \Delta_N = 0, +1, -1.

Take a look at this plot: It shows the relative energies of the three
triplet rotational components that are connected to singlet level: F_1
(\Delta_N = -1), F_2 (\Delta_N = 0), and F_3 (\Delta_N = +1).  This
plot is made in the approximation that the singlet and triplet
B-values are nearly equal.  Imagine placing a singlet level at some
energy on the plot - the energy of the singlet could be represented by
a horizontal line.  If the singlet is very close to zero, the F_2
component varies slowly with rotation and the F_1,2 components rapidly
speed away.  Essentially, only the F_2 component matters when J>1.

If the singlet is not located close to zero, EITHER the F_1 or the F_3
component of the triplet is rapidly approaching.  The triplet
component approaches with a rate of about 2 cm-1 per rotational
quantum number. According to Ryan and Bryan's calculations, the
<electronic*vibrational> spin-orbit matrix elements between S_1 and
T_3 max out at about 0.1 cm-1. The rotational factor only reduces this
value.  Compared to the matrix element, an energy difference of 2 cm-1
per J is huge!

I believe that strongly coupled, mediating T_3 levels will make
themselves known at several values of J through this
energy-denominator effect.  I think this is what we observe in our 3^3
K=2 data.  Weakly coupled T_3 levels should appear as perturbations at
a single J, and then immediately disappear, because their matrix
elements are much, much less than 2 cm-1. This means that weak triplet
perturbations should be _common_, and it means that we should be able
to infer their approximate energy from the value of J which is
perturbed in the singlet, i.e. + or - (J*2*B).

This means that further SEELEM spectroscopy of the S_1 state should be
fruitful and informative.  I hope you are all totally stoked!  I'll
discuss the experimental consequences and possibilities in a later
email.

--Kyle
