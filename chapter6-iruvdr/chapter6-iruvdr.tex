\documentclass[12pt]{mitthesis} 

\usepackage{lgrind, braket, amsmath,
  amssymb, bbm, booktabs, subfig, color} 
\usepackage[pdftex]{graphicx}
\usepackage[version=3]{mhchem}

\newcommand{\TODO} [1]{\textcolor{magenta}{\textbf{TODO:} #1}}
\newcommand{\POINT}[1]{\textcolor{magenta}{#1}}

\newcommand{\rcm}{cm$^{-1}$}

\hyphenation{acetylene}
\hyphenation{Hamiltonian}

\begin{document}

\tableofcontents
\clearpage

\subsubsection*{NOTES}
\clearpage

\chapter{IR-UV double resonance SEELEM spectroscopy of the $3^3B^1$
  \emph{ungerade} bending polyad of $S_1$ acetylene}

\section{Introduction}

\POINT{Repeat case for studying the non-symmetric bending modes of
  acetylene.}

\POINT{Argument for mode 6: brings molecule to half-linear geometry,
  which may increase coupling.}

\POINT{State goal: look at n.s. bending motions without DD or Coriolis
  coupling.}


\section{Experiment}

\POINT{Reference previous chapter for most experimental details.}

\POINT{Double resonance notes from AHS, also in his paper.}

\POINT{Say what intermediate states were used for double resonance.}
Transitions ending in the \emph{e}-symmetry branches of $S_1$ $3^36^1$
and $3^34^1$ were recorded using the $J=1-5$ levels of the $S_0$
$\nu_3+\nu_4$ Q branch at 3897.16 \rcm as intermediate states.
Transitions ending in the \emph{f}-symmetry branches were recorded
using various P and R branch transitions.

\POINT{Laser resolution $\approx$ 0.094 \rcm; laser step size
  $\approx$ 0.047 \rcm }

\POINT{Describe polarization method used to record the $J'=0$
  spectrum.  (See p.11 in 4/2007--8/2007 notebook.}

\section{Observations}

\POINT{Present survey spectra for both bands and individual lines for
  $3^36^1$.}

\POINT{Present polarization spectrum for rotationless level of
  $3^36^1$.}

\section{Selection rules for spin-orbit coupling}

\POINT{Discuss spin-orbit selection rules for $\Delta K$, and
  reference tables.  The theory is from Stevens and Brand as well as
  Hougen.}

\TODO{Show character tables and axis labels for \emph{trans}
  acetylene.  Discuss the correspondence between the ($a,b,c$) axis
  system and the ($x,y,z$).  See p.9, 18--19 in 4/2007--8/2007 notebook.}

\begin{table}
  \centering
  \begin{tabular}{llllrl}
    Vib.
    & $^{v}\Gamma_S$ & $^{ev}\Gamma_S$ & $\Gamma_\sigma$ 
    & $\Delta K$ & $^{v}\Gamma_T$ \\
    \toprule

%     $2^1 3^1 4^2$ 
%     & $a_g$ & $^{1}A_u$ & $B_g$ & $0$ & \textcolor{red}{$a_g$} \\
%     & & & $A_g, B_g$ & $\pm 1$ & $a_g, b_g$ \\

%     $2^1 3^1 6^2$
%     & $a_g$ & $^{1}A_u$ & $B_g$ & $0$ & \textcolor{red}{$a_g$} \\
%     & & & $A_g, B_g$ & $\pm 1$ & $a_g, b_g$ \\

%     $2^1 3^1 4^1 6^1$ 
%     & $b_g$ & $^{1}B_u$ & $B_g$ & $0$ & \textcolor{red}{$b_g$} \\
%     &      &                 & $A_g, B_g$ & $\pm1$     & $a_g, b_g$ \\

    $3^3 4^1$ 
    & $a_u$ & $^{1}A_g$ & $B_g$ & $0$ & \textcolor{red}{$a_u$} \\
    & & & $A_g, B_g$ & $\pm1$ & $a_u, b_u$ \\[10pt]

    $3^3 6^1$ 
    & $b_u$ & $^{1}B_g$ & $B_g$ & $0$ & \textcolor{red}{$b_u$} \\
    & & & $A_g, B_g$ & $\pm1$ & $a_u, b_u$ \\

  \end{tabular}\\[5mm]
  
  $C_{2h}$ symmetry, $^{e}\Gamma_T =$ $^{3}B_u$\\[1cm]

  \begin{tabular}{llllrl}
    Vib.
    & $^{v}\Gamma_S$ & $^{ev}\Gamma_S$ & $\Gamma_\sigma$ & $\Delta K$ & $^{v}\Gamma_T$ \\
    \toprule
%     $2^1 3^1 4^2$
%     & $a$ & $^{1}A$ & $B$ & $0$ & \textcolor{red}{$a$} \\
%     & & & $A, B$ & $\pm 1$ & $a, b$ \\

%     $2^1 3^1 6^2$
%     & $a$ & $^{1}A$ & $B$ & $0$ & \textcolor{red}{$a$} \\
%     & & & $A, B$ & $\pm 1$ & $a, b$ \\
    
%     $2^1 3^1 4^1 6^1$
%     & $b$ & $^{1}B$ & $B$ & $0$ &\textcolor{red}{$b$} \\
%     & & & $A, B$ & $\pm1$ & $a, b$ \\

    $3^3 4^1$ 
    & $a$ & $^{1}A$ & $B$ & $0$ & \textcolor{red}{$a$} \\
    & & & $A, B$ & $\pm1$ & $a, b$ \\[10pt]

    $3^3 6^1$ 
    & $b$ & $^{1}B$ & $B$ & $0$ & \textcolor{red}{$b$} \\
    & & & $A, B$ & $\pm1$ & $a, b$ \\

  \end{tabular}\\[5mm]

  $C_{2}$ symmetry, $^{e}\Gamma_T =$ $^{3}B$
\end{table}

\POINT{Show intensity factors for each coupling component and
  forbidden transitions from $3^3B^1$ to $T_3$ levels.  (See p.138,143 of
  1/2007--3/2007 notebook.)}

\TODO{Figure: Rosetta stone for singlet-triplet coupling.  (See p.127
  of 1/2007--3/2007 notebook.)}

\TODO{Use Bryan's program to compute overlap integrals for the two bands.}

\POINT{Show number of modes available in each symmetry as a function
  of energy.}

\section{Analysis: Intensity distributions}

\POINT{Present energy levels and reduced term value plot for $3^36^1$.
  (See p.97 of 4/2007--8/2007 notebook.)} 

\section{Analysis: Parsimonious trees}

\POINT{Use $\Delta_3$ statistic, $\Sigma^2$ statistic, or Fourier
  transform methods? (See p.65 in 11/2007--1/2008 notebook.)}

\section{Analysis: Gated fluorescence and ``smooth transition''
  between SEELEM and LIF spectra}

\section{LIF/SEELEM intensity ratio analysis}

\section{Conclusion}

\end{document}