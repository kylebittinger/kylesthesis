\documentclass[12pt]{mitthesis} 

\usepackage{lgrind, braket, amsmath,
  amssymb, bbm, booktabs, subfig, color} 
\usepackage[pdftex]{graphicx}
\usepackage[version=3]{mhchem}

\newcommand{\TODO} [1]{\textcolor{magenta}{\textbf{TODO:} #1}}
\newcommand{\POINT}[1]{\textcolor{magenta}{#1}}

\hyphenation{acetylene}
\hyphenation{Hamiltonian}

\begin{document}

\tableofcontents
\clearpage

\subsubsection*{NOTES}

\clearpage

\chapter{SEELEM/LIF spectroscopy of \emph{ungerade}
  vibrational levels of $S_1$ acetylene: contrasting singlet-triplet
  dynamical behavior}

\section{Introduction}

\POINT{Show the vibrational modes of \emph{trans} acetylene and their
  symmetries.  Explain selection rules for transitions from the ground
  state.  (See p.33 of1/2007--3/2007 notebook.)}

\POINT{Make the case for studying non-symmetric bending motions in
  acetylene.  The $T_3$ state is bent out-of-plane.  However, several
  observations have been made that mode 6 is more active in promoting
  ISC.  How does this jive with our distant doorway theory?}

\POINT{The out-of-plane bending mode in acetylene $S_1$ is strongly
  mixed with the antisymmetric in-plane bend, so we must study them in
  tandem.}

\POINT{Recent breakthrough in our understanding of these bending
  polyads.  Energies and eigenvectors predicted by AHS and Merer.
  Overlap with $T_3$ levels predicted by Bryan and Ryan.  (See p.40 of
  1/2007--3/2007 notebook.)}

\POINT{Make the case for studying the role of $\nu_3$ in ISC.  Cite
  trends observed on ZAC papers.  We consider three bands in roughly
  the same energy region.}

\section{Experiment}

\POINT{Same info as paper but with a few more details.  Did we use
  other laser dyes for some of the bands?}

\section{Investigation of several $2^13^1B^2$ and $3^2B^2$ polyad members}

\TODO{Make survey figure showing the energy of all investigated bands,
  overlayed on acetylene LIF spectrum.}

\subsection{Results}

\TODO{Figures of spectra for these bands (with assignments)}

\POINT{SEELEM spectrum of $2^13^1B^2$ (Feb 2007)}

\POINT{SEELEM spectrum of $3^2 4^2$ with lone SEELEM peak in band gap
  (December 2006, p.86 of 9/2006--1/2007 notebook.)}

\POINT{Introduce concept of gated fluorescence and show examples.}

\subsection{Analysis: LIF/SEELEM intensity distributions}

\POINT{Make the case for center of gravity metric. Use tools from
  Chapter 2.  Refute arguments against.  (See p.109 of 4/2007--8/2007
  notebook.)}

\POINT{Present error analysis for gated fluorescence
  measurements. (Have an appendix for ``optimum'' gate width? Notebook
  11/2005--3/2006)}

\POINT{Use gated fluorescence and SEELEM intensity comparison to make
  the case for weak coupling. Lifetimes tabulated in notebook
  8/2007--10/2007}

\POINT{Lifetimes do not match intensities -- does this point to
  complicated coupling?  (See p.99 of 8/2007--10/2007 notebook, and
  p.31 of 4/2007--8/2007 notebook.)}

\subsection{Discussion: Competing Coriolis and Darlin-Dennison
  interactions}

\POINT{Adapt analysis from paper.}

\POINT{Contrasting behavior of modes 4 and 6 attributed to differing
  amounts of $b_g$ vibrational character -- they are otherwise
  completely mixed.}

\POINT{Could the amount of $b_g$ character be J-dependent?}

\POINT{Alternating pattern of intensity -- does this point to a $T_3$
  level with $K=0$?}

\section{Investigation of ``pure bending'' vibrational levels:
  $4^16^3$ and $4^3$}

\subsection{Results}

\TODO{Figures of spectra for these bands (with assignments)}

\POINT{SEELEM spectrum of $4^1 6^3$ is very weak (See data from Feb 2,
  p.50 of 1/2007--3/2007 notebook.)}

\POINT{SEELEM spectrum of $4^3$ $K=2$ hot band (Jan 16A) -- need to
  check this in Watson's atlas.}

\subsection{Analysis and discussion}

\POINT{Discuss in terms of global structure of bending polyads.  (See
  AHS and Merer's predictions on p.39 of 1/2007--3/2007 notebook.)}

\POINT{The conclusion is that we probably don't have enough S/N with
  these bands to perform any SEELEM/LIF intensity analysis.  Can we
  get any amount of info from lifetimes or quantum beats?}

\POINT{Uniformity of coupling in both Q and (P,R) branches -- does
  this imply that $K\ne0$?}


\section{Investigation of a near-isoenergetic set of $2^n3^m$ ($n+m=3$)
  vibrational levels}

\subsection{Observations}

\TODO{Figures of spectra for these bands (with assignments)}

\POINT{SEELEM spectrum of $2^2 3^1$ (Oct/Nov 2006, see ``similitude''
  calculations, p.62 of Sep 2006--Jan 2007 notebook.)}

\POINT{SEELEM spectrum of $2^1 3^2$ P, Q-branch (See Jan 16A+B, p.124--127
  of 9/2006--1/2007 notebook, also assignments on p.2 of
  1/2007--3/2007 notebook.)}

\POINT{SEELEM spectrum of $3^3$ $K=2$ hot band (See Jan 22C, p.31,34
  of 1/2007--3/2007 notebook.)}

\subsection{LIF/SEELEM intensity distributions}

\subsection{Analysis}

\POINT{Analyze the remaining bands under the assumption of distant
  doorway coupling via $T_3$.  Include center-of-gravity metrics,
  lifetime/gated fluorescence metrics, and intensity distributions.}

\section{Conclusion}

\POINT{The LIF/SEELEM spectra for the progression of $2^n3^m$ levels
  shows the expected}

\POINT{Darling-Dennison and A-type Coriolis coupling play a major role
  in determining the singlet-triplet interactions for vibrational
  levels involving modes 4 and 6.}

\POINT{Make the case for studying the bending motions without
  complications from Darling-Dennison or A-type Coriolis coupling.}

\end{document}