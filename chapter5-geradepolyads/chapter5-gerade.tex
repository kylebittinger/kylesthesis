% \documentclass[12pt]{mitthesis}
% \usepackage[pdftex]{graphicx} 
% \usepackage{kylesthesis}
% \begin{document}

% \tableofcontents
% \clearpage

% \listoffigures
% \clearpage

% \subsubsection*{NOTES}
% The chapter is basically complete at this point, although I need to
% write a short subsection on band-integrated SEELEM/LIF center of
% gravity and revise the conclusion.  Another 6 small tasks are
% scattered throughout.

% I am looking especially for any suggestions to explain the important
% points more plainly and simply.  It would also help to have any
% confusing passages pointed out, so I can move more slowly through
% those points.  Also, if you find that I am way off on something,
% please let me know about that, too.

% Thank you, triplet readers!

% \clearpage

% \setcounter{chapter}{3}
\chapter{SEELEM/LIF spectroscopy of acetylene: Spectral signatures of
  energetically distant doorway levels}

\section{Introduction}

What are the global characteristics of doorway-mediated intersystem
crossing in acetylene, \ce{C2H2}?  Although we have a detailed
understanding of the spectroscopic patterns arising from a crossing
between near-degenerate $S_1$ and $T_3$ vibrational levels
\cite{humphrey97, altunata00, altunata01, mishra04}, we do not know
quite what to anticipate in the spectrum of an $S_1$ sublevel which is
not near-degenerate ($\lvert \Delta E \rvert < 2$ \rcm) with a $T_3$
doorway.  In this study, we examine the spectra of several $S_1$
vibrational sublevels that fall into the latter category, and produce
a comprehensive description of singlet$\sim$triplet mixing that takes into
account the effects of energetically distant ($\lvert \Delta E \rvert
> 2$ \rcm) $T_3$ doorway sublevels.  The spectra of $S_1$ sublevels
interacting with remote $T_3$ perturbers are far from uneventful, and
provide an unexpected wealth of information as to the identity and
rotationless energy of $T_3$ vibrational sublevels.

To observe the distribution of metastable, nominal $T_{1,2}$
eigenstates in the vicinity of the upper state of an allowed $S_1
\leftarrow S_0$ transition, we take advantage of two complementary,
simultaneously recorded spectroscopic detection channels:
Laser-Induced Fluorescence (LIF) and Surface Electron Ejection by
Laser-Excited Metastables (SEELEM).  Gated LIF detection is sensitive
to eigenstates with short radiative lifetimes ($\tau <$ 10 \microsec),
resulting from large fractional $S_1$ character.  Conversely, SEELEM
detection is sensitive only to long-lived ($\tau >$ 300 \microsec)
states with vertical electronic excitation greater than the work
function of the metal used as the detector surface.  SEELEM detectable
eigenstates must have a small but nonzero amount of fractional $S_1$
character.  As a result, simultaneously recorded LIF and SEELEM
spectra display mutually exclusive sets of eigenstates that arise from
spin-orbit mixed $S_1$ and $T_{3,2,1}$ zero-order basis states.

Intersystem crossing in the \astate\ ($S_1$) state of acetylene is
well-described by a doorway-mediated coupling model.  The model is
supported by observations in Zeeman anticrossing spectroscopy
\cite{dupre91, dupre95a, dupre95b}, LIF Zeeman quantum beat
spectroscopy \cite{ochi87, ochi91, dupre93}, high-resolution LIF
spectroscopy \cite{drabbels94, altunata01}, simultaneous LIF and
SEELEM spectroscopy \cite{humphrey97, altunata00, mishra04},
photodissociation plus H-atom action spectroscopy \cite{yamakita03,
  loffler98, mordaunt98}, and photoelectron spectroscopy
\cite{degroot07}.  Briefly, the model states that matrix elements
between vibrational levels of the $S_1$ and $T_{1,2}$ electronic
states are very nearly zero, and that all mixing between $S_1$ and
$T_{1,2}$ levels is induced indirectly by nonzero $S_1 \sim T_3$ and
$T_3 \sim T_{1,2}$ matrix elements.

% The results of \emph{ab initio} calculations indicate that the
% electronic spin-orbit matrix element is an order of magnitude larger
% for $S_1 \sim T_3$ than for $S_1 \sim T_1$.  This result, in
% combination with smaller vibrational overlap integrals resulting
% from a $\sim$ 10,000 \rcm\ excess of vibrational energy in the
% $T_{1,2}$ electronic states, accounts for the relative magnitude of
% the matrix elements according to the model.

The \emph{trans}-bending mode of $S_1$ acetylene, $\nu_3$, is known to
be an important promoter of $S_1 \sim T_3$ mixing.  In Zeeman
anticrossing experiments, Dupr\'{e} and coworkers observed a rapid
increase in the anticrossing density, as well as the product,
$\rho_{\text{vib}} \braket{H_{st}}$, with energy in $\nu_3$
\cite{dupre91, dupre95b}.  They also observed a single, broad
singlet$\sim$triplet anticrossing in the $3 \nu_3$ $K_a=0$ level,
%with a zero-field matrix element of 0.29 \rcm,
which was in turn perturbed by many smaller couplings \cite{dupre93}.
These observations led the authors to propose that mixing between
$S_1$ and the dense manifold of optically dark states, including
$T_{1,2}$ levels, is controlled by the magnitude of vibrational
overlap factors between $S_1$ levels and particular doorway
vibrational levels.  Subsequent theoretical and experimental work
identified the doorways as vibrational levels of the $T_3$ electronic
state \cite{vacek96, sherrill96, humphrey97, altunata00}.

The $S_1$ $3 \nu_3$ \Ka{1} level has been heavily studied due to the
presence of a near-degenerate vibrational level of $T_3$, which gives
rise to a rotationally assignable $S_1 \sim T_3$ level crossing at $J
\approx 3$.  Spectroscopic patterns in LIF and SEELEM spectra arising
from the effects of the local $T_3$ perturber, are discussed by several
authors \cite{humphrey97, altunata00, altunata01, mishra04}.  Using
vibrational overlap integrals gained from \emph{ab initio}
calculations of the $T_3$ electronic surface, Thom and coauthors were
able to exclude all but several candidate $T_3$ levels as the $3\nu_3$
local perturber \cite{thom07}.
% Overlap between $S_1$ and $T_3$ levels predicted by Bryan and Ryan.
% (See p.40 of 1/2007--3/2007 notebook.)
However, a recent observation of increased average $T_{1,2}$
electronic character at $J'=8-12$ suggests that this local $T_3$
perturber is not the sole, and perhaps not even the primary doorway
for $3 \nu_3$ \Ka{1} \cite{degroot07}.

That an energetically distant $T_3$ doorway, which would be required
to have a correspondingly larger spin-orbit matrix element than the
local perturber, may play a role in spin-orbit mixing between $S_1$ $3
\nu_3$ and the $T_{1,2}$ manifold, is not altogether surprising, given
the energy region in question.  \emph{Ab initio} calculations are in
agreement that a seam of intersection between the electronic surfaces
of $S_1$ and $T_3$ states is energetically nearby \cite{ventura03,
  thom07}.  Such a surface crossing would allow for strong
interactions with several $T_3$ vibrational levels in the same energy
region.  It should also play a role in promoting singlet$\sim$triplet
coupling within other $S_1$ levels in the same energy region.  The
$4\nu_3$ level, 1000 \rcm\ higher in energy, is also strongly
perturbed by the $T_{1,2}$ manifold, although no obvious local $T_3$
doorway has been observed \cite{drabbels94, ochi91}.

We turn our attention to other vibrational levels of $S_1$ in the same
energy region that are not near-degenerate with a mediating $T_3$
level at low $J$.  In the absence of a local $T_3$ perturber, coupling
between $S_1$ levels and the local manifold of $T_{1,2}$ levels is
expected to be mediated by energetically distant $T_3$ levels.
Evidence for such energetically distant, mediating $T_3$ levels is
obtained by comparing simultaneously recorded LIF and SEELEM spectra.

We begin by deriving the energy level spacings and spin-orbit matrix
elements between rovibrational levels of $S_1$ and $T_3$.  Next, we
develop a new analysis technique that is sensitive to the presence of
distant $T_3$ doorway levels.  This new technique takes into account
observed shifts of relative intensities in the frequency-domain LIF
spectrum as a function of delay time, which result from the time
development of an incoherent ensemble of eigenstates having different
radiative lifetimes.  We then apply this new technique to the
simultaneously recorded LIF and SEELEM spectra of four $S_1$
vibrational sublevels, and discuss the properties of admixed,
energetically distant $T_3$ vibrational levels.

\section{Theory: $S_1 \sim T_3$ rotational energy level spacing and
  spin-orbit matrix elements}
\label{theory1}

In the $S_1$ (\astate) electronic state, acetylene is a near-symmetric
prolate top.  Neglecting centrifugal and asymmetry terms, the
rovibrational energies are approximately \cite{watson82}
\begin{equation}
  \label{eq:s1-energy-levels}
  E_{S_1}(v,K,J) = T_v + [A_v - B_v] K^2 + \frac{B_v+C_v}{2} \, J(J+1),
\end{equation}
where $J$ is the total angular momentum, $K$ is the absolute value of
the projection of the orbital angular momentum on the a-axis of the
molecule, $T_v$ is the rotationless energy of the vibrational level
$v$, and $A_v$, $B_v$, and $C_v$ ($\approx B_v$) are the rotational
constants for the vibrational level.  The model of a near-symmetric
prolate top is also appropriate to describe the energy levels of the
$T_3$ electronic state, although \emph{ab initio} calculations predict
that the \emph{trans}-bent equilibrium geometry is non-planar
\cite{ventura03, thom07}.  Adopting a Hund's case ($b$) basis, the
rovibrational energy levels of $T_3$ are approximately
\begin{equation}
  \label{eq:t3-energy-levels}
  E_{T_3}(v,K,N) = T_v + [A_v - B_v] K^2 + \frac{B_v+C_v}{2} \, N(N+1),
\end{equation}
where the pattern-forming rotational quantum number, $N$, is the total
angular momentum exclusive of spin.  At each energy $E_{T_3}(v,K,N)$,
% Work on this more later    vvvv 
three closely spaced spin components are present, due to the three
possible values of the total angular momentum $J$.  The components are
labeled $F_1$ ($J=N+1$), $F_2$ ($J=N$), and $F_3$ ($J=N-1$).  
% Figure
% \ref{fig:triplet-rot-structure} shows an example of the rotational
% energy level structure.  
Since $S=0$ for the $S_1$ electronic state,
only one component $N=J$ is present for all $S_1$ rotational levels.

% \begin{figure}
%   \caption{\TODO{Adapt energy level figure from sketch.  Combination
%       of Herzberg v.3 figs 27 and 31.}}
%   \label{fig:triplet-rot-structure}
%   \centering

%   \vspace{4in}
%   [DIAGRAM OF TRIPLET\\ 
%    ROTATIONAL ENERGY LEVELS]
%   \vspace{4in}
% \end{figure}

The $S_1$ and $T_3$ electronic states are mixed by spin-orbit
interaction.  The total first-order spin-orbit matrix element between
rovibrational states of $S_1$ and $T_3$ is the product of three
factors: an electronic spin-orbit matrix element, a vibrational
overlap factor, and a rotational factor that arises from angular
momentum coupling.  Spin-orbit matrix elements follow the rotational
selection rules $\Delta J = 0$ and $\Delta P = 0$, where $P$ is the
projection of $J$ on the a-axis of the molecule \cite{hougen64}.  In
Hund's case ($b$), the quantum number $P$ is mixed among several
levels with different values of $K$\footnote{The quantum number $K$ is
  the projection of $N$ on the top axis.}, leading to the case ($b$)
selection rule $\Delta K = 0, \pm 1$ \cite{hougen64, stevens73}.

We consider the relative rotational energy differences between an
$S_1$ vibrational sublevel, $\ket{S}$, and a $T_3$ vibrational sublevel,
$\ket{T}$.  According to the selection rule $\Delta J = 0$, a
rotational level of $\ket{S}$, $\ket{S;J=J_S}$, may mix with three
rotational levels of $\ket{T}$: $\ket{T;N=J_S-1}$ (an $F_1$
component), $\ket{T;N=J_S}$ (an $F_2$ component), and
$\ket{T;N=J_S+1}$ (an $F_3$ component).  Since the three triplet
components with nonzero matrix element differ in the rotational
pattern-forming quantum number, $N$, they have different rotational
energies.  Written in terms of $J$, these energies are
\begin{equation}
  \label{eq:f123-energies}
  E_T(J) = T_{0,T} + 
  \begin{cases}
    B_T \, (J+1)(J+2) & F_3 \:\: (\Delta N = +1), \\
    B_T \, J(J+1)     & F_2 \:\: (\Delta N = 0), \\
    B_T \, J(J-1)     & F_1 \:\: (\Delta N = -1),
  \end{cases}
\end{equation}
where $T_{0,T}$ is the rotationless energy of the $T_3$ vibrational
sublevel, $\ket{T}$.  Subtracting the rotational energies of $\ket{S}$,
\begin{equation}
  E_S(J) = T_{0,S} + B_S \, J(J+1),
\end{equation}
from $E_T(J)$ yields the rotational energy differences between
$\ket{S}$ and $\ket{T}$.  In the approximation that the difference in
rotational constants $\Delta B_{ST}$ is small compared to $B_T$, the
rotational energy differences are
\begin{equation}
  \label{eq:components}
  \Delta E_{ST}(J) = \Delta T_0 +
  \begin{cases}
    \:\: 2B_TJ + 2 B_T 
    & F_3 \:\: (\Delta N = +1), \\
    \:\: 0
    & F_2 \:\: (\Delta N = 0), \\
    \:\: - 2B_TJ
    & F_1 \:\: (\Delta N = -1).
  \end{cases}
\end{equation}

\begin{figure}
  \caption{Approximate energy separation between singlet and triplet
    vibrational sublevels, plotted as a function of $J$.  The relative
    energy of the $F_2$ spin component is approximately constant.  The
    relative energies of the $F_1$ and $F_3$ spin components change at
    a rate of approximately 2 \rcm\ per $J'$.  Dotted lines indicate
    the approximate average vibrational level spacing for the $S_1$
    and $T_3$ electronic states in the 45,000 \rcm\ energy region.}
  \label{fig:rotational-energy-differences}
  \centering
  \includegraphics[width=6in]{rotational-energy-differences.pdf}
\end{figure}

The approximate energy differences $\Delta E_{ST}(J)$ are plotted as a
function of $J$ in Figure \ref{fig:rotational-energy-differences}.  A
value of $B_T=1.1$ \rcm\ was chosen for the figure based on the
rotational assignment of a $T_3$ perturber in the $3^3$ \Ka{1}
sublevel of $S_1$ \cite{mishra04}.  The energy of the $F_2$ component,
relative to that of the $S_1$ level, has a slope of approximately
zero, while the relative energies of the $F_1$ and $F_3$ components
have slopes of approximately $-2B_T \approx -2$ \rcm\ and $+2B_T
\approx +2$ \rcm\ per $J$, respectively.  Also included on the plot is
the average vibrational level spacing, $\sim 10$ \rcm\, for both the
$S_1$ and $T_3$ electronic states.  The relative energies of the $F_1$
and $F_2$ components are 10 \rcm\ distant from the rotationless energy
separation when $J \approx 4$.

Because the $F_1$ and $F_3$ components of the triplet have large
slopes in relative energy separation from a singlet level, the
component of a $T_3$ level with which an $S_1$ level interacts is
determined by the rotationless energy difference.  The possibilities
are as follows.
\begin{itemize}
\item When two vibrational sublevels of $S_1$ and $T_3$ are
  near-degenerate, $-4 \lesssim (T_{0,T} - T_{0,S}) \lesssim 2$ \rcm,
  the energy separation between the singlet and the $F_2$ component of
  the triplet remains close to its initial value as $J$ increases,
  while the relative energy separations of the $F_1$ and $F_3$
  components increase at a rate of approximately $2B_T$ per $J$.
  Energy denominators favor mixing between the singlet and the $F_2$
  component of the triplet when $J \ge 1$.
\item When a $T_3$ doorway level lies higher in rotationless energy
  than the singlet level, $ (T_{0,T} - T_{0,S}) \gtrsim 2$ \rcm,
  energy denominators favor mixing between the singlet and the $F_1$
  component of the triplet.  The rotational energy separation between
  the singlet and the $F_1$ component of the triplet decreases at a
  rate of approximately $2B_T$ per $J$ until a level crossing occurs
  at $J \approx (T_{0,T} - T_{0,S}) / 2 B_T$.
\item When a $T_3$ doorway level lies below a singlet level in
  rotationless energy, $(T_{0,T} - T_{0,S}) \lesssim 4$ \rcm, energy
  denominators determine that the primary interaction is via the
  triplet $F_3$ component.  The rotational energy separation between
  the singlet level and the $F_3$ component of the triplet decreases
  at a rate of $2B_T$ per $J$ until a crossing occurs at $J \approx
  (T_{0,S} - T_{0,T}) / 2 B_T - 1$.
\end{itemize}
Taken together, these trends guarantee a level crossing at $J<8$ for
any pair of $S_1$ and $T_3$ sublevels that have a rotationless energy
separation of less than 20 \rcm.  Regardless of the sign of the
rotationless energy difference, one spin component of a $T_3$ level
always approaches the singlet at a rate of $2B_T$ per $J$.  When a
doorway is energetically distant, in other words not near-degenerate
in rotationless energy, energy denominators determine that the
interaction occurs via either the $F_1$ or $F_3$ component, depending
on the sign of $T_{0,T} - T_{0,S}$.

\begin{figure}[p]
  \caption{Rotational factors of spin-orbit matrix elements between
    rovibrational levels of the $S_1$ and $T_3$ electronic states,
    $K_S$=1 ($J \ge K_S$).  The plot includes matrix elements for the
    $F_3$ ($\Delta N = +1$) and $F_1$ ($\Delta N = -1$) components of
    a $T_3$ sublevel with $K_T=0,1,2$.  In all cases, the rotational
    factors vary by less than a factor of 2 as a function of $J$.}
  \label{fig:rotational-factors-1}
  \centering
  \includegraphics[width=6in]{rotational-factors-k1.pdf}
\end{figure}

\begin{figure}[p]
  \caption{Rotational factors of spin-orbit matrix elements between
    rovibrational levels of the $S_1$ and $T_3$ electronic states,
    $K_S$=2.  The plot includes matrix elements for the $F_3$ ($\Delta
    N = +1$) and $F_1$ ($\Delta N = -1$) components of a $T_3$
    sublevel with $K_T=1,2,3$.  The rotational factors are essentially
    the same as those for $K_S=1$.}
  \label{fig:rotational-factors-2}
  \centering
  \includegraphics[width=6in]{rotational-factors-k2.pdf}
\end{figure}

To verify that the magnitude of the spin-orbit matrix elements does
not dominate over energy denominator effects for the $F_1$ and $F_3$
components of a triplet level, the rotational factors for spin-orbit
matrix elements are calculated from general formulas given by Stevens
and Brand \cite{stevens73}.  These factors are plotted as a function
of $J$ in Figure \ref{fig:rotational-factors-1} for a singlet level
with $K_S$=1.  The rotational factors quickly approach their
asymptotic limits as $J$ is increased.  Even at the lowest value of
$J$, the rotational factor never differs from its asymptotic value by
more than a factor of 2.  Figure \ref{fig:rotational-factors-2} shows
the same factors for $K_S$=2, also considered in this study.  The plot
is essentially the same.

The overall magnitude of spin-orbit matrix elements between
rovibrational levels of $S_1$ and $T_3$ is therefore not governed by
$J$-dependent rotational factors, but by vibrational overlap factors.
Franck-Condon overlaps between relatively low-lying vibrational levels
of $S_1$ and $T_3$ are calculated to vary over at least three orders
of magnitude \cite{thom07}.  The resultant wide variation in
spin-orbit matrix element, arising from vibrational overlap factors,
gives rise to two classes of level crossings between $S_1$ and $T_3$
levels: those where $\lvert H_{ST} \rvert \ll 2B_T$, and those where
$\lvert H_{ST} \rvert$ is on the order of $2B_T$.

When $H_{ST}^2 \ll 2B_T$, energy denominator effects cause an $S_1
\sim T_3$ level crossing to be observable at only one $J$ in the
spectrum.  However, once a triplet level with small vibrational
overlap is located at a particular $J$, its energy above or below the
perturbed singlet level may be found according to Equation
\ref{eq:components}.  If the rotationless energy of the triplet
vibrational level is higher than that of the singlet, the triplet must
have $T_{0,T} \approx T_{0,S} + 2B_TJ$.  If the rotationless energy of
the triplet vibrational level is lower than that of the singlet, the
triplet must have $T_{0,T} \approx T_{0,S} - 2B_TJ - 2B_T$.

When $\lvert H_{ST} \rvert$ is on the order of $2B_T$, significant
mixing between the $S_1$ and $T_3$ levels may occur at several
consecutive values of $J$.  Such $T_3$ levels, having $\lvert H_{ST}
\rvert$ on the order of $2B_T$, and lying more than $4B_T$ away in
rotationless energy, are the energetically distant doorway states with
which we are most concerned.  By first mixing with the $T_3$ level,
the $S_1$ level is permitted to mix with the local manifold of
$T_{1,2}$ levels.  Although a level crossing with the mediating $T_3$
level may occur at too high $J$ to be observed\footnote{We are able to
  measure $J' < 9$ in pulsed jet experiments.}, some $T_3$-mediated
$S_1 \sim T_{1,2}$ mixing is observable in the spectrum of the $S_1$
level.  Mixing between the $S_1$ level and nearby $T_{1,2}$ levels is
influenced by the relative energy of the doorway level: mixing is
increased with $T_{1,2}$ levels that lie closer in energy to the
doorway level.  In the following section, we explain this effect and
examine a method by which it may be observed.



























\section{Theory: Signatures of doorway-mediated intersystem crossing
  in delayed, incoherent fluorescence measurements}

An energetically distant $T_3$ doorway level imprints its presence on
the LIF and SEELEM spectra of the $S_1$ levels with which it
interacts.  When an $S_1$ level mixes with an energetically distant
$T_3$ doorway level, the relative energy of the doorway level
influences the local mixing between the $S_1$ level and nearby
$T_{1,2}$ levels.  This leads to a local energy dependence of the
average fractional $S_1$ character in nominal $T_{1,2}$ eigenstates.
This quantity may be evaluated using perturbation theory, after first
pre-diagonalizing the interaction between $S_1$ and $T_3$.  Following
Chapter 2, Equation 2.4, we find the energy dependence of the
fractional $S_1$ character, $a_{S_1}^2$, of a nominal $T_{1,2}$
eigenstate to be
\begin{equation}
  \label{eq:ave-s1-char}
    a_{S_1}^2(E) = 
    \alpha^2 (1-\alpha^2) H_{T_3T_{1,2}}^2
    \biggl \lbrace 
    \frac{1}{E - E_{S_1}} 
    + \frac{1}{E - E_{T_3}} 
    \biggr \rbrace^2,
\end{equation}
where $\alpha$ is the mixing angle between the $S_1$ level and the
doorway state, $H_{T_3T_{1,2}}$ is the $T_3 \sim T_{1,2}$ matrix
element, $E_{S_1}$ is the energy of the nominal $S_1$ eigenstate and
$E_{T_3}$ is the energy of the nominal $T_3$ eigenstate.

The energy dependence of average fractional $S_1$ character in the
ensemble of nominal $T_{1,2}$ eigenstates causes their
intensity-weighted center of gravity to be slightly shifted, relative
to the energy of the nominal $S_1$ eigenstate in the LIF spectrum.
The intensity-weighted center of gravity is defined as
\begin{equation}
  \label{eq:cog-def}
  E_{\text{ave}} = \int E \times I(E) \; dE,
\end{equation}
where $I(E)$ is a unit-normalized intensity distribution.  Since the
total intensity of fluorescence from an eigenstate is proportional to
its fractional $S_1$ character (see Chapter 3), the direction of the
center of gravity offset may be found by combining Equations
\ref{eq:ave-s1-char} and \ref{eq:cog-def}.  Using an energy scale
where $\alpha^2 (1-\alpha^2) H_{T_3T_{1,2}}^2 = 1$, and setting the
zero of energy to $E_{S_1}$, we obtain
\begin{equation}
  \label{eq:ave-s1-cog}
  \begin{split}
    E_{\text{ave}} \: 
    &\propto \: \int_{-1}^{+1} E \times a_{S_1}^2 \; dE,\\
    &\propto \: \frac{2 E_{T_3}}{E_{T_3}^2 - 1} + \log
    \left[
      \frac{E_{T_3}+1}{E_{T_3}-1}
    \right].
  \end{split}
\end{equation}
The above expression for the center of gravity is correct only in its
sign, since it cannot be normalized by integrating Equation
\ref{eq:ave-s1-char} over an energy range that includes $E_{S_1}$.
The sign of Equation \ref{eq:ave-s1-cog} is preserved because the
normalization factor is always positive.  When the relative energy
$E_{T_3} - E_{S_1}$ is positive, the center of gravity in Equation
\ref{eq:ave-s1-cog} is positive relative to $E_{S_1}$.  When the
relative energy, $E_{T_3} - E_{S_1}$, is negative, the center of
gravity is negative.  We have thus established that the center of
gravity offset for an ensemble of nominal $T_{1,2}$ eigenstates is
biased toward a $T_3$ doorway state, relative to $E_{S_1}$.
% \TODO{Plot the center of gravity by diagonalizing a 3x3
%   Hamiltonian, instead of using perturbation theory?}

In the remainder of this section, we examine the dependence of the
intensity-weighted center-of-gravity on the time delay of a gated
fluorescence spectrum for an incoherent ensemble of simultaneously
excited, mixed $S_1 \sim T_{1,2}$ eigenstates.  Because the different
eigenstates in the ensemble have different fluorescence lifetimes,
their relative fluorescence intensities change as a function of time
delay after excitation.  Of particular concern is the relative
intensity of the nominal $S_1$ eigenstate relative to that of the
remaining, nominal $T_{1,2}$ eigenstates.  Changes in relative
eigenstate intensities lead to changes in the overall center of
gravity, when the metric is computed from the total fluorescence
intensity within a delayed time window $t$ to $t+dt$ after excitation.
Using a simple model, we show how the dependence of center of gravity
on time delay reveals the energy of the doorway state, relative to
that of the nominal $S_1$ eigenstate.



\subsection{Characteristics of an incoherent, high-resolution
  fluorescence spectrum after a time delay}

For a pure $S_1$ rovibrational level, $\ket{s}$, the time-dependent
fluorescence intensity is
\begin{equation}
  I_s(t) = \frac{1}{\tau_s} \;
           \exp \left[
             -\frac{t}{ \tau_s} 
           \right],
\end{equation}
normalized such that $\int_0^{\infty} I_s(t) \, dt = 1$.  The quantity
$\tau_s$ is the radiative lifetime of the zero-order $S_1$ level.  In
acetylene, this quantity is largely independent of vibrational and
rotational levels within $S_1$, and is determined to be approximately
270 ns \cite{ochi91, stephenson84}.

We consider the case where $\ket{s}$ is admixed, through an
energetically distant $T_3$ doorway, into $T_{1,2}$ triplet levels to
create a set of singlet$\sim$triplet mixed eigenstates, $\lbrace
\ket{m} \rbrace$, each having some fractional $S_1$ character,
$a_m^2$.  If the lifetime of a pure triplet level is much longer than
$\tau_s$, the lifetime of a mixed eigenstate $\ket{m}$ is
\begin{equation}
  \label{eq:tau-m}
  \tau_m = \tau_s / a_m^2,
\end{equation}
and its time-dependent fluorescence intensity is
\begin{equation}
  \label{eq:int-m}
  I_m(t) = \frac{a_m^4}{\tau_s} \;
           \exp \left[
             -\frac{a_m^2 \, t}{\tau_s} 
           \right].
\end{equation}
The total integrated fluorescence intensity for a mixed state is
$\int_0^{\infty} I_m(t) \, dt = a_m^2$, relative to unit intensity for
a pure $S_1$ state.  This reflects the $a_m^2$ times smaller
probability for excitation of a mixed eigenstate, $\ket{m}$.

The relative fluorescence intensity among an incoherent ensemble of
simultaneously excited, singlet$\sim$triplet mixed eigenstates is
examined after a selectable time delay $t$.  The upper limit of time
delay that can be selected in our LIF experiments is determined by the
field-of-view of the fluorescence detection optics.  In this study,
the field-of-view of the fluorescence detection optics is about 5 mm.
The maximum viewing time is therefore about 5 \microsec\ $\approx
18\tau_s$ for molecules travelling in the molecular beam with average
velocity $\approx 10^3$ m/s.  At a time delay longer than $18\tau_s$,
no molecules remain to be seen within the fluorescence field of view.

At a chosen value of time delay, the relative fluorescence intensity
among an ensemble of mixed states has a single maximum with respect to
fractional $S_1$ character, $a_m^2$.  States with the largest amount
of fractional $S_1$ character fluoresce quickly.  Although
fluorescence from these states begins at $t=0$ with the greatest
intensity, their relative intensity decreases rapidly as $t>0$ with
respect to longer-lived states.  Eigenstates having the smallest
amount of fractional bright state character, conversely, have lower
intensity at $t=0$.  However, they fluoresce slowly, and as a
consequence their relative intensity increases with time delay,
relative to the fluorescence from other eigenstates in the ensemble.

At a given time delay, $t$, the fractional $S_1$ character resulting
in the greatest relative intensity is found by setting the derivative
of eigenstate intensity (Equation \ref{eq:int-m}) with respect to
fractional $S_1$ character, $a_m^2$, equal to zero.  We find that the
derivative is
\begin{equation}
 \frac{ \partial I_m }{ \partial a_m^2 } =
   \frac{a_m^2}{\tau_s}
   \exp \left[
     -\frac{a_m^2 \, t}{\tau_s} 
   \right]
   \left (
     2 - \frac{a_m^2 \, t}{\tau_s}
   \right ).
\end{equation}
When $t \leq 2\tau_s$, the fluorescence intensity equation increases
monotonically with fractional $S_1$ character, and is maximized
when the fractional $S_1$ character is 1.  When $t > 2 \tau_s$,
the fluorescence intensity equation is at a maximum when the
fractional $S_1$ character is
\begin{equation}
  \label{eq:am-max}
  a_m^2 = \frac{2 \tau_s}{t}.
\end{equation}

\begin{figure}
  \caption{The intensity of fluorescence from a mixed eigenstate at
    time $R = t/\tau_s$, plotted as a function of bright state
    character.}
  \label{fig:int-at-rc}
  \centering
  \includegraphics[width=7.5in,angle=90]{intensity-at-delay.pdf}
\end{figure}

Figure \ref{fig:int-at-rc} shows the dependence of fluorescence
intensity on fractional $S_1$ character at several selectable values
of time delay.  When $t = 0$, the fluorescence intensity is monotonic
and greatest for eigenstates that contain the largest amount of
fractional $S_1$ character.  At a delay of $t = 5 \tau_s$, the
fluorescence intensity equation is ``tuned'' to states with a
fractional $S_1$ character of 40\%.  Molecules in eigenstates
containing too much fractional $S_1$ character are discriminated
against, because they have already fluoresced with high probability by
a time delay of $5 \tau_s$.  Molecules in eigenstates with fractional
$S_1$ character $\ll$ 40\% have a low probability of fluorescing in
the time window under consideration, and are also discriminated
against at this delay time.  Figure \ref{fig:int-at-rc} also shows the
dependence of fluorescence intensity on fractional $S_1$ character for
$t=18\tau_s$, the maximum fluorescence time delay used in this study.
At this time delay, the fluorescence intensity is at a maximum when
the fractional $S_1$ character is approximately 22\%.

The SEELEM detector used in this study detects metastable acetylene
molecules after a flight time of $\tau_f =$ 309
\microsec.  Considering only the SEELEM electron ejection probability
resulting from $S_1$ electronic character, the total SEELEM detection
probability is (Chapter 2, equation 28):
\begin{equation}
  \label{eq:seelem-prob-s}
  P_{\text{SEELEM}} \propto 
    a_m^4 \; 
    \exp \left [ 
      -\frac{a_m^2 \tau_f}{\tau_s} 
    \right ].
\end{equation}
This is a good approximation to the total SEELEM detection
probability, including electron ejection probability resulting from
fractional $T_3$ electronic character, when a $T_3$ doorway state is
energetically distant\footnote{The term ``energetically distant'' has
  a slightly different meaning in this context. The magnitude of $T_3$
  electronic character will be approximately proportional to the
  magnitude of $S_1$ electronic character if the energy denominator
  $E_m - E_{T_3}$ is approximately constant with respect to the energy
  denominator $E_m - E_{S_1}$.} (see Chapter 2, section 2.3.2).  The
SEELEM detection probability, Equation \ref{eq:seelem-prob-s}, has the
same functional form as the delayed LIF intensity, Equation
\ref{eq:int-m}.  At a flight time of 309 \microsec\, the detection
probability is at a maximum for eigenstates with 0.17\% fractional
$S_1$ character.  In this approximation, the SEELEM detector may be
viewed as an extreme discriminator for detecting states with small
fractional $S_1$ character.  The dependence of SEELEM detection
probability on fractional $S_1$ character is also included in Figure
\ref{fig:int-at-rc}.




\subsection{Dependence of the intensity-weighted center of gravity on
  time delay}

Our next step is to examine how the LIF intensity-weighted center of
gravity metric,
\begin{equation}
  \label{eq:ch5-cog-def}
  E_{\text{ave}} = \int E \times I(E) \; dE,
\end{equation}
changes as the time delay is increased.  We model the interaction
between a rovibrational level of $S_1$ and the local ensemble of
$T_{1,2}$ rovibrational levels by first constructing a model system
containing only one $T_{1,2}$ basis state.


Consider a Hamiltonian containing one $S_1$ level, $\ket{s}$, and one
$T_{1,2}$ level, $\ket{t}$.  Let us define the two mixed states
$\ket{1}$ and $\ket{2}$ as
\begin{equation}
  \begin{split}
    \ket{1} &=  (1 - a^2)^{1/2} \ket{s} + a \ket{t}\\
    \ket{2} &= -a \ket{s} + (1 - a^2)^{1/2} \ket{t},
  \end{split}
\end{equation}
where $a$ is the mixing angle between $\ket{s}$ and $\ket{t}$, $0 \leq
\lvert a \rvert \leq 0.5$.  The fractional $S_1$ character of
$\ket{2}$ is $a^2$, and that of $\ket{1}$ is $(1 - a^2)$.  The
normalized fluorescence intensities of the mixed states are
\begin{equation}
  \begin{split}
    I_1(t) &= \frac{(1 - a^2)^2}{\tau_s} \; \exp 
          \left[
            - \frac{(1 - a^2) t}{\tau_s}
          \right]\\
    I_2(t) &= \frac{a^4}{\tau_s} \; \exp 
          \left[
            - \frac{a^2 t}{\tau_s}
          \right].
  \end{split}
\end{equation}
The ratio of the fluorescence intensities, $I_1(t)/I_2(t)$, has the
following time dependence:
\begin{equation}
  \frac{I_1(t)}{I_2(t)} = 
  \left(
    \frac{1 - a^2}{a^2}
  \right)^2
  \exp
  \left[
    - \frac{(1 - 2a^2) t}{\tau_s}
  \right].
\end{equation}  
The intensity ratio at $t=0$ is determined by the prefactor $( (1 -
a^2)/a^2 )^2$.  In the limit of long time delay, $I_1(t)/I_2(t)$
always approaches zero, because the lifetime of $\ket{2}$ is always
longer than that of $\ket{1}$, by definition.  

The intensity ratio $I_1/I_2$ is plotted as a function of time in
Figure \ref{fig:ratio-devel}, using several values of the mixing
fraction, $a^2$.  At small values of the mixing fraction, $0.001
\leq a^2 \leq 0.2$, the intensity ratio changes rapidly at a time
delay between $5\tau_s$ and $15\tau_s$.  When the mixing fraction
approaches its limiting value of 0.5, the prefactor $( (1 -
a^2)/a^2 )^2$ causes the intensities $I_1$ and $I_2$ to be
of comparable magnitude at $t=0$.  As the time delay is increased, the
intensity ratio changes slowly, because the states $\ket{1}$ and
$\ket{2}$ have similar lifetimes.

\begin{figure}
  \caption{Dependence of the intensity ratio $I_1/I_2$ on time delay
    for a model two-state system.}
  \label{fig:ratio-devel}

  \centering
  \includegraphics[width=6in]{ratio-development.pdf}
\end{figure}

The intensity-weighted center of gravity of the model system may be
written as a sum of two terms,
\begin{equation}
  E_{\text{ave}}(t) = E_1 \times I_1(t) \; + \; E_2 \times I_2(t),
\end{equation}
where $E_1$ and $E_2$ are the energies of $\ket{1}$ and $\ket{2}$,
respectively.  The intensity-weighted center of gravity is plotted as
a function of time delay in Figure \ref{fig:cog-devel}.  The main
features of the plot are similar to those of Figure
\ref{fig:ratio-devel}.  At small values of the mixing fraction, $0.001
\leq a^2 \leq 0.2$, the intensity-weighted center of gravity changes
rapidly from $E_1$ to $E_2$ at a time delay between $5\tau_s$ and
$15\tau_s$.  When the mixing fraction is nearly 0.5, the initial
center of gravity at $t=0$ is midway between $E_1$ and $E_2$, because
$\ket{1}$ and $\ket{2}$ have similar intensities at $t=0$.  In this
case of near-50:50 mixing, the center of gravity changes slowly as a
function of time.  The quantity still approaches $E_2$ in the limit of
long time delay, because the lifetime of $\ket{2}$ is always longer
than that of $\ket{1}$.  The total magnitude of change in center of
gravity is decreased, due to the greater relative intensity of
$\ket{2}$ at $t=0$.

\begin{figure}
  \caption{Time development of the center of gravity for a model
    system containing two basis states.}
  \label{fig:cog-devel}
  \centering
  \includegraphics[width=6in]{cog-development.pdf}
\end{figure}

A characteristic feature in the plot of center of gravity vs. time is
the presence and location of a rapid (within several $\tau_s$) shift
in center of gravity from $E_1$ to $E_2$.  If a rapid shift is
observed, its location may be used to infer the mixing fraction,
$a^2$.  The location of the shift is determined by the point of
inflection for the center of gravity function, where $I_1=I_2$.  The
dependence of mixing fraction on the location of a center of gravity
shift is plotted in Figure \ref{fig:cog-inflection-point}.  Rapid
center of gravity shifts, located between $t=5.5\tau_s$ and
$t=14\tau_s$, are the result of mixing fractions between 0.1 and 0.001,
respectively.

\begin{figure}
  \caption{The location of a rapid shift in center of gravity (within
    several units of $\tau_s = 270$ ns) may be used to infer the $S_1
    \sim T_3$ mixing fraction.  The characteristic time of the shift
    is defined by the value of delay time where the relative
    intensities of the nominal $S_1$ eigenstate and the nominal
    $T_{1,2}$ eigenstates are equal.  The mixing fraction is plotted
    here as a function of the location of a rapid shift in center of
    gravity, defined as the delay time where $I_1 = I_2$.}
  \label{fig:cog-inflection-point}

  \centering
  \includegraphics[width=6in]{cog-inflection-point.pdf}
\end{figure}

The presence of a rapid shift in center of gravity vs. time indicates
that the intensity ratio $I_1/I_2$ changes rapidly.  For this to
occur, the fluorescence decay rates of $\ket{1}$ and $\ket{2}$ must be
appreciably different.  For a system containing many $T_{1,2}$ levels,
the qualitative features of the center of gravity function will be
determined by the subset of nominal $T_{1,2}$ eigenstates with the
largest fractional $S_1$ character.  These states will have the
largest intensities, and therefore the greatest impact on the
behavior of the center of gravity vs. time.  Therefore, the presence
of a rapid shift in center of gravity vs. time indicates that no
$T_{1,2}$ level is mixed to more than 20\% with the $S_1$ level.

We have examined the dependence of the intensity-weighted center of
gravity on time delay in the LIF spectrum of an incoherent ensemble
of simultaneously excited, singlet$\sim$triplet mixed molecular
eigenstates.  A model system was constructed, containing one $S_1$
level and one $T_{1,2}$ level.  The initial ($t=0$) position of the
center of gravity for the model system is influenced by the total $S_1
\sim T_{1,2}$ mixing fraction.  As the time delay is increased, the
center of gravity shifts from its initial position to a final
position, which is determined solely by the energy and fractional
$S_1$ character of nominal $T_{1,2}$ eigenstates.  Small $S_1 \sim
T_{1,2}$ mixing fractions lead to a rapid, characteristic shift in the
center of gravity with time delay.  Large mixing fractions lead to a
gradual change in the center of gravity with time delay.  In this
case, the change in center of gravity not only occurs more slowly, but
the total magnitude of the shift is decreased, due to the larger
relative intensity of nominal triplet states at $t=0$.

The behavior of the LIF center of gravity with time delay will be used
as a primary tool to examine the spectra of various vibrational
sublevels in acetylene \astate.  To analyze the local $S_1 \sim
T_{1,2}$ mixing around a single rotationally-resolved $S_1 \leftarrow
S_0$ transition, relative intensities are compared over a very small
energy range, about 1 \rcm.  This has the effect of minimizing
relative intensity errors, which arise in LIF experiments mostly from
slowly drifting laser power and baseline effects.  Unlike measurements
of fluorescence lifetime, the center of gravity metric is not biased
by molecules leaving the LIF detection area.  Laser excitation of
molecules is instantaneous with respect to the position of a molecule
in a molecular beam, and has no affect on molecular velocities.  When
the excitation laser is perpendicular to the jet axis, the molecular
velocity along the jet axis has no bearing on the excitation
probability.  Although the total signal is decreased as molecules exit
the LIF detection region, relative excitation probabilities are
conserved among the subset of molecules that remain in the observation
region.  The center of gravity is determined only by the
\emph{relative} fluorescence among the set of molecules which remain
within the field-of-view of the LIF detection optics.


























\section{Experiment}

In the SEELEM experiment, a molecular beam of acetylene is excited by
a $\sim$5 ns FWHM pulsed tunable laser into spin-rotation-vibration
eigenstates of metastable electronic states via weak, nominally
forbidden transitions. After excitation, the long-lived species must
travel 35 cm before colliding with an Au metal detector surface, where
an electron is ejected in the de-excitation process. Two criteria must
be met for electron ejection by a metastable species. First, the
vertical electronic energy of the metastable state of the particle
approaching the surface must exceed the work function of the metal
($\Phi_{\text{Au}}$ = 5.1 eV). Second, the radiative lifetime of the
detected metastable eigenstate ($\tau_\text{rad}$) must exceed the
flight time from the point of laser excitation to the SEELEM surface
($\Delta$t=300 $\mu$s).

A sample of acetylene (BOC gases) at a backing pressure of one
atmosphere is pulsed through a 0.5 mm diameter nozzle (Jordan Valve),
operating at 10 Hz, into a diffusion pumped vacuum chamber at
$\sim$5\e{-5} Torr.  An Nd:YAG pumped (Spectra Physics GCR-270),
frequency-doubled dye laser (Lambda Physik FL3002, 220 nm) excites the
acetylene molecules in the pulsed jet expansion 2 cm downstream from
the nozzle orifice.  %\TODO{Include laser details here.}

UV-LIF is detected in a direction perpendicular to the plane defined by the
intersection of the pulsed molecular and laser beams using f/1.2
collection optics, a fluorescence filter (UG-11) to reduce scattered
laser light, and a PMT (Hamamatsu model R375).  The time-varying
UV-LIF signal was averaged on a digital oscilloscope (LeCroy) 
%\TODO{get  model} 
at each laser frequency.  The resulting oscilloscope trace,
%containing 500 evenly spaced voltage levels over 
recorded over a 5 \microsec\ timespan (time bins of 10 ns), was
transferred to a PC for analysis.

Simultaneous SEELEM detection took place in a separate, differentially
pumped vacuum chamber.  Following laser excitation, molecules in the
pulsed expansion passed through a conical skimmer (3mm diameter),
forming a collimated molecular beam in the SEELEM detection chamber.
The detector chamber was diffusion pumped (Varian 600) to maintain a
pressure of $\sim$4\e{-7} torr during operation.  The SEELEM detector
surface, a 2.5 cm diameter region of heated (300\degrees\ C) Au foil,
was located 35 cm downstream from the point of laser excitation.  The
SEELEM detector was identical to that used in the previously described
apparatus with Au foil ($\Phi_{\text{Au}}$ = 5.1 eV) as the detector
surface.

Electron signals from the SEELEM detector were collected using pulse
counting techniques.  Electrons ejected from the metal surface were
detected by a nearby electron multiplier (ETP, SGE Instruments, Model
14831H).  The collection plate of the multiplier was biased at +100V
to attract electrons.  The electron multiplier signal was sent to a
discriminator (EG\&G/Ortec Model 9301), and then to a PC-operated
multichannel scalar (Oxford Tennelec Nucleus Inc. MCS-II v2.091),
where the total number of electrons was summed.  The typical signal
level for SEELEM detection of acetylene was 2-20 counts per laser
pulse.

Both SEELEM and LIF signals were averaged over 100 laser shots at each
laser frequency.  The frequency increment step size was typically
$\sim$0.015 \rcm\ in the doubled output, although several detailed
spectra were recorded with a 4$\times$ smaller frequency step size.





























\section{SEELEM/LIF spectroscopy of four $S_1$ vibrational levels
  located at higher energy than the minimum of the $S_1 \sim T_3$ seam
  of intersection}


The choice of $S_1$ vibrational levels in this study was guided by two
considerations: the energetic accessibility of $T_3$ vibrational
levels \cite{cui97, thom07} and absence of predissociation effects.
The minimum of the $S_1 \sim T_3$ electronic seam of intersection,
located near $3^3$ \Ka{1}, $T_0 \simeq 45300$ \rcm, provides a lower
energy bound for the Franck-Condon accessibility of $T_3$ vibrational
levels from the $S_1$ electronic surface \cite{cui97}.  Above this
energy, it is possible for levels of $S_1$ to have large vibrational
overlaps with levels of $T_3$.  The first dissociation limit, located
near $3^4$ \Ka{1}, $T_0 \simeq 46300$ \rcm, provides an upper energy
bound for this study \cite{mordaunt98}.  Below this dissociation
limit, predissociation pathways are not energetically accessible, and
do not complicate the study of $S_1 \sim T_3$ interactions.

We consider four $S_1$ sublevels in the energy region of 45300$-$46300
\rcm, observed in the $\tilde{A}^1A_u \leftarrow \tilde{X}
^1\Sigma_g^+$ spectrum of acetylene.  The sublevels are listed with
their total (rotationless) vibronic energy, $T_{v0}$, in Table
\ref{table:termvals}.  Included in the table is an order-of-magnitude
estimate of the SEELEM:LIF intensity ratio observed in the spectrum.
To determine this quantity, the relative LIF intensities of $2^23^1$
\Ka{1}, $3^24^2$ \Ka{1}, and $2^13^2$ \Ka{1} were estimated from a low
resolution jet spectrum reported by Merer and coworkers
\cite{merer03}.  The LIF intensity of the $3^3$ \Ka{2} subband was
compared to that of $3^3$ \Ka{1} in our own experiments.  The absolute
SEELEM intensity was then divided by the estimated LIF intensity, and
normalized to the largest SEELEM:LIF ratio.


%%%%%%%%%%%%%%%%%%%%%%%%%%%%%%%%%%%%%%%%%%%%%%%%%%%%%%
%%
%% INSERT LEVEL TABLE HERE
%%
%%%%%%%%%%%%%%%%%%%%%%%%%%%%%%%%%%%%%%%%%%%%%%%%%%%%%%

\begin{table}
  \caption{The four acetylene $S_1$ vibrational levels considered in
    this study, listed here, have total vibronic energies in the
    critical region above the minimum of the $S_1 \sim T_3$ electronic
    seam of intersection ($\simeq 45300$ \rcm) and below the first
    dissociation barrier ($\simeq 46300$ \rcm) \cite{mordaunt98}.
    % The asymptote is 46074, barrier is actually about 500 cm-1
    An order-of-magnitude estimate of the observed SEELEM:LIF 
    intensity ratio is included.}
  \label{table:termvals}

  \centering
  \begin{tabular}{rlrr}
    \\
    Subband & & $T_0$ (\rcm ) & SEELEM:LIF\\
    \midrule
    $2^23^1$ \Ka{1} & $\leftarrow$ $0_0$ & 46009 & $10^{-2}$ \\
    $3^24^2$ \Ka{1} & $\leftarrow$ $0_0$ & 45812 & $10^{-1}$ \\
    $2^13^2$ \Ka{1} & $\leftarrow$ $0_0$ & 45677 & $10^{-1}$ \\
      $3^3$ \Ka{2} & $\leftarrow$ $4_1$ & 45352 & 1 \\
  \end{tabular}
\end{table}

%%%%%%%%%%%%%%%%%%%%%%%%%%%%%%%%%%%%%%%%%%%%%%%%%%%%%%
%%
%% END OF LEVEL TABLE
%%
%%%%%%%%%%%%%%%%%%%%%%%%%%%%%%%%%%%%%%%%%%%%%%%%%%%%%%



We observe that the highest energy state in our study has the lowest
SEELEM:LIF intensity ratio.  Rather than total energy, the SEELEM:LIF
intensity ratio is determined by the number of quanta in mode 3.  This
observation is in agreement with previous estimates of SEELEM:LIF
intensity ratios in acetylene \astate\ \cite{humphrey97}.
Measurements of the Zeeman anticrossing density show the same
exponential dependence on the number of quanta in $\nu_3$
\cite{dupre91}.  Because the intensity ratio is determined by
vibrational basis state character and not by total energy, it must be
a reflection of the vibrational overlap factors contained in the
spin-orbit matrix elements, rather than due to the slowly increasing
density of $T_{1,2}$ states \cite{dupre91, dupre95b}.

The presence of local $T_3$ perturbers is ruled out, because we
observe no telltale, systematic splittings that would have indicated a
crossing with a triplet $F_2$ spin component, as in $3^3$ \Ka{1}
\cite{mishra04}.  Evidence of energetically distant $T_3$ doorway
levels is found, however, in both the SEELEM and LIF spectra of these
sublevels.  To characterize the spectral signatures of energetically
distant doorway levels, we discuss the SEELEM/LIF spectrum of each
sublevel individually.


\subsection{The $3^24^2$ \Ka{1} sublevel: evidence for an
  energetically distant $T_3$ doorway level plus a localized $T_3$
  level crossing}

% Spectrum: Dec05n 2006, p.86 of 9/2006--1/2007 notebook.

%%%%%%%%%%%%%%%%%%%%%%%%%%%%%%%%%%%%%%%%%%%%%%%%%%%%%%
%%
%% INSERT 3^2 4^2 FIGURES HERE
%%
%%%%%%%%%%%%%%%%%%%%%%%%%%%%%%%%%%%%%%%%%%%%%%%%%%%%%%


\begin{figure}
  \caption{Simultaneously recorded SEELEM (upper trace) and LIF (lower
    trace) spectra of the $3^24^2$ \Ka{1} sublevel of the \astate\
    state of \ce{C2H2}.  The LIF spectrum is integrated in two time
    regions: an early time window ($0.5\tau_s-2\tau_s$, solid trace)
    and a delayed time window ($10\tau_s-18\tau_s$, dashed trace).
    The peak positions are blueshifted in the delayed fluorescence
    spectrum for all transitions, with the exception of Q(2).
    Interactions with a remote perturber level of slightly lower
    energy than the nominal $S_1$ level are apparent in the delayed
    fluorescence spectrum of the Q(2) and R(1) transitions.}
  \label{fig:spectrum-32b2}
  \centering
  \includegraphics[width=7in,angle=90]{spectrum-3242-p3r4.pdf}
\end{figure}

\begin{figure}
  \caption{Dependence of the intensity-weighted center of gravity on
    delay for a series of individually resolved transitions, Q(1$-$3)
    (top), and R(0$-$3) (bottom), in the LIF spectrum of $3^24^2$
    \Ka{1}.  Local perturbations in the Q(2) and R(1) transitions
    cause the behavior of the center of gravity to be different from 
    those of the other transitions as the delay time is increased.  
    % \TODO{Include, for
    %   each delayed center of gravity plot, a table showing the
    %   integration regions used.}
  }
  \label{fig:32b2-cog-delay}
  \centering
  \vspace{5mm}
  \includegraphics[width=6in]{32b2-q123-cog-delay.pdf}
  \includegraphics[width=6in]{32b2-r0123-cog-delay.pdf}
\end{figure}


%%%%%%%%%%%%%%%%%%%%%%%%%%%%%%%%%%%%%%%%%%%%%%%%%%%%%%
%%
%% END OF 3^2 4^2 FIGURES
%%
%%%%%%%%%%%%%%%%%%%%%%%%%%%%%%%%%%%%%%%%%%%%%%%%%%%%%%

The $3^24^2$ level is the upper member of the $3^2B^2$ polyad of the
acetylene \astate\ state, where $B \equiv v_4 + v_6$.  The $\nu_4$ and
$\nu_6$ vibrations are strongly mixed by a-type Coriolis and
Darling-Dennison anharmonic resonances \cite{merer08}.  As a result,
the nominal $3^24^2$ level is expected to contain essentially a 50:50
mixture of mode 4 ($4^2$) and mode 6 ($6^2$) basis character
\cite{merer08, virgo07}.

The simultaneously recorded SEELEM/LIF spectrum of the $3^24^2$ \Ka{1}
$\leftarrow$ $0_0$ subband of the \AtoX\ electronic spectrum is shown
in Figure \ref{fig:spectrum-32b2}. To highlight the dependence of the
LIF spectrum on delay time, two integration regions are used.  An
early-LIF fluorescence spectrum is generated by gating the
fluorescence signal between $0.5\tau_s$ and $2\tau_s$.  A second,
delayed-LIF spectrum is generated by integrating over the time window
$10\tau_s-18\tau_s$.  The line positions of several transitions are
clearly blueshifted in the delayed-LIF spectrum relative to the
early-LIF.  Line offsets in the R(0), R(1), and R(4) transitions are
approximately 0.08 \rcm.

To more closely examine the dependence of the LIF spectrum on time
delay, the intensity-weighted center of gravity (Equation
\ref{eq:ch5-cog-def}) is plotted for each transition as a function of
delay time.  The plot for Q-branch transitions was generated from
another dataset, not shown here, sampled at approximately $4 \times$
more closely spaced energy intervals, resulting in increased signal to
noise.  With the exception of the R(1) and Q(2) transitions, the
center of gravity increases by 0.04-0.12 \rcm\ over the delay range
$0.5\tau_s-18\tau_s$.  Such a consistent increase in center of gravity
with delay time indicates the presence of a non-local $T_3$ doorway
level that lies higher in energy, according to Equation
\ref{eq:ave-s1-cog}.

% The interaction is observed in both parities of the singlet at
% $J'=1$ and $3$.  A doorway level with $K_T=0$ would The doorway
% level cannot be assigned as $K_T=0$.  \emph{e}/\emph{f} symmetry
% selection rules the interaction is present in both parities of the
% singlet, .  Two possibilities remain for $K_T$

A line splitting is present in the LIF spectrum of two transitions,
Q(2) and R(1).  In the delayed fluorescence spectrum, the extra lines
appear with larger relative intensity than the main component of the
transition.  Their increased relative intensity in the delayed
fluorescence spectrum indicates that the extra lines arise from levels
with longer zero-order fluorescence lifetimes than that of the nominal
singlet level.  The extra component of R(1) is observable only in the
delayed-LIF spectrum, where its relative intensity increase causes it
to emerge from under the red wing of the main line.  The extra line in
the R(1) transition, with upper state quantum number $J'=2$, is
located at an energy of -0.33 \rcm\ below the main component at
45814.87 \rcm.  The $J'=2$ perturber of opposite parity appears in
Q(2).  It is located at an energy of -0.17 \rcm\ relative to the
strong, nominally singlet LIF transition at 45810.08 \rcm.

The presence of two local perturbations at the same $J'$ (but opposite
parities), arising from levels with a long zero-order lifetime, and
not appearing in transitions with adjacent values of $J'$, can be
explained by the presence of a level crossing involving the $F_1$ or
$F_3$ component of a $T_3$ level that has a small spin-orbit matrix
element with the $S_1$ level.  According to Equation
\ref{eq:f123-energies}, the relative energy of a $T_3$ level in this
case is $\pm 2B \simeq \pm 2$ \rcm\ at $J'=1$ and $3$.  

At $J'=2$, the singlet and its triplet perturber are not 50:50 mixed,
thus the matrix element must be appreciably smaller than half the
energy separation.  This places an upper bound on the matrix elements
of about $0.01$ \rcm.  At $J'=1$ or $3$, the increase in energy
denominator from 0.2 \rcm\ to 2 \rcm\ would make the intensity of the
perturber $100$ times weaker than for $J'=2$, precluding its
observation in LIF.

A lone SEELEM peak is observed in the band gap at 45811.6 \rcm, at
approximately 2 \rcm\ higher frequency than the Q(3) transition at
45809.5 \rcm.  It is possible that this peak arises from the same
weakly perturbing $T_3$ vibrational level.  Unfortunately, it is
impossible to search for the other parity component of this perturber
in the spectrum, because the appropriate energy region is overlapped
by an R-branch transition.  However, if we assign the transition in
the band gap as $J'=3$, it must also be assigned as the $F_3$
component of the triplet level, $N_T=J'+1=4$, in order to be located
at $+2$ \rcm\ instead of $-2$ \rcm.  As a consequence, the triplet
level must have a rotationless vibronic energy of $45811 - 6B \simeq
45804$ \rcm.

We have examined a singlet sublevel, $3^24^2$ \Ka{1}, which shows
evidence of interaction with an energetically distant $T_3$ doorway
level at higher energy.  An extra line with a small matrix element was
observed at $J'=2$ in both parity components of the singlet level, and
was assigned as a level of $T_3$ based on its zero-order lifetime.  An
assignment of $\Delta N=+1$ was suggested for the perturbing level
based on the observation of a weak transition in the SEELEM spectrum,
which was assigned as $J'=3$, $N_T=4$ because the observed frequency
is $+2$ \rcm\ relative to the singlet Q(3) transition.  The assignment
of $N_T$ determines that the rotationless energy of the perturbing
triplet level is $-6$ \rcm\ relative to the singlet.

% If the assignment of $N_T$ is not made from the observed SEELEM
% transition at 45811.6 \rcm, the extra line in Q(3) may belong to
% either the $F_1$ ($\Delta N = -1$) or $F_3$ ($\Delta N = +1$)
% component of the perturbing $T_3$ level.  In this circumstance, the
% additional possibility for assignment of $N_T$ means that the
% perturbing $T_3$ level may also be located at an energy of $+4$ \rcm\
% relative to the singlet.

Local, weak perturbations by $T_3$ levels are common in the spectrum
of \astate\ acetylene, as we will see in the following examples.  In
cases where $N_T$ or $K_T$ of the perturber can be assigned, the
relative energies of weak transitions into locally perturbing
vibrational levels can be determined exactly.  In cases where $N_T$ or
$K_T$ can be assigned for energetically distant $T_3$ doorway levels
($\lvert H_{ST} \rvert$ is on the order of $2B_T$), the relative
rotationless energy is determined to be either positive or negative.
This can then be checked against the dependence of the
intensity-weighted LIF center of gravity on time delay.

\subsection{The $2^13^2$ \Ka{1} sublevel: assignment of $K_a$ for an
  energetically distant $T_3$ doorway level}

% Spectrum: Jan 16A+B, p.124--127 of 9/2006--1/2007 notebook, also
% assignments on p.2 of 1/2007--3/2007 notebook.

%%%%%%%%%%%%%%%%%%%%%%%%%%%%%%%%%%%%%%%%%%%%%%%%%%%%%%
%%
%% INSERT 2^1 3^2 FIGURES HERE
%%
%%%%%%%%%%%%%%%%%%%%%%%%%%%%%%%%%%%%%%%%%%%%%%%%%%%%%%

\begin{figure}
  \caption{Simultaneously recorded SEELEM (upper trace) and LIF (lower
    trace) spectra of the $2^13^2$ \Ka{1} sublevel of the \astate\
    state of \ce{C2H2}.  The LIF spectrum is integrated in two time
    regions: an early time window ($0.5\tau_s-2\tau_s$, solid trace)
    and a delayed time window ($10\tau_s-18\tau_s$, dashed trace).
    The Q(1) and Q(2) transitions are redshifted in the delayed
    fluorescence spectrum, in contrast to the Q(3,4,5,6) transitions.
    The upper states of the P(2) and Q(1) transitions, with the same
    $J'$ but opposite parity, show similar redshifts in the delayed
    fluorescence spectrum.}
  \label{fig:spectrum-2132}
  \centering
  \vspace{1cm}
  \includegraphics[width=7in,angle=90]{spectrum-2132-q6q1.pdf}
\end{figure}

\begin{figure}
  \caption{Dependence of the intensity-weighted center of gravity on
    delay for a series of individually resolved rotational
    transitions, Q(1$-$6), in the LIF spectrum of $2^13^2$ \Ka{1}.
    The center of gravity of the Q(1) and Q(2) transitions is
    identical to the peak positions in the SEELEM spectrum at
    delay $>$ $15\tau_s$.}
  \label{fig:2132-q123456-cog-delay}
  \centering
  \vspace{1cm}
  \includegraphics[width=6in]{2132-q123456-cog-delay.pdf}
\end{figure}

%%%%%%%%%%%%%%%%%%%%%%%%%%%%%%%%%%%%%%%%%%%%%%%%%%%%%%
%%
%% END OF 2^1 3^2 FIGURES
%%
%%%%%%%%%%%%%%%%%%%%%%%%%%%%%%%%%%%%%%%%%%%%%%%%%%%%%%

We turn next to an $S_1$ sublevel which also contains two quanta of
the $\nu_3$ vibration, $2^13^2$ \Ka{1}.  The overall SEELEM:LIF
intensity ratio observed in the spectrum of this sublevel is similar
to that of $3^24^2$ \Ka{1}, indicating a similar overall mixing angle
with $T_3$ doorway levels.

The SEELEM/LIF spectrum of the $2^13^2$ \Ka{1} $\leftarrow$ $0_0$
subband of the \AtoX\ transition is shown in Figure
\ref{fig:spectrum-2132}.  The LIF spectrum is integrated in early and
delayed time windows, using time limits of $0.5\tau_s-2\tau_s$ and
$10\tau_s-18\tau_s$.  The spectrum contains six Q-branch transitions
(upper states of \emph{f}-symmetry, $J'=1-6$) and one P-branch
transition (upper state of \emph{e}-symmetry, $J'=1$).  The delayed
LIF peak position is redshifted by -0.08 \rcm\ relative to the early
LIF peak position for two transitions terminating in \emph{f}-symmetry
states, Q(1) and Q(2).  The sole \emph{e}-symmetry state observed in
the spectrum, via the P(2) transition, is also redshifted by -0.12
\rcm\ in the delayed fluorescence spectrum.  The remaining
transitions, Q(3,4,5,6), have the same peak position in the early and
delayed-LIF spectra.  The Q(3) transition appears anomalously weak.
This issue is addressed later in this section.

The dependence of the intensity-weighted center of gravity for all
transitions in the Q-branch is shown in Figure
\ref{fig:2132-q123456-cog-delay}.  The center of gravity for the Q(1)
and Q(2) transitions changes from the zero-delay position by
approximately -0.07 \rcm\ at a time delay of $15\tau_s$, in accord
with the observation of delay-dependent peak positions in the LIF
spectrum.  The centers of gravity for the Q(3,4,5,6) transitions do
not shift by more than 0.02 \rcm\ from their initial positions.

We showed in the preceding section that a monotonic shift in center of
gravity with delay indicates the presence of an energetically distant
$T_3$ doorway level.  It is puzzling that, in the present case, such
an interaction abruptly ceases at $J' \geq 3$.  However, this behavior
can be explained by the presence of a \emph{second} $T_3$ doorway
level, located at higher energy than the singlet level.  An
interaction with a second $T_3$ doorway level could cause the delayed
center of gravity to behave differently for the Q(3,4,5,6)
transitions, if the interaction with the second doorway level were to
\emph{begin} at $J'=3$.

The existence of such an interaction leads to an assignment of $K_T$
for the second doorway level.  The spin-orbit selection rule, $\Delta K
= 0, \pm 1$, restricts possible assignments to $K_T=$0, 1, and 2.  A
level with $K_T=2$ has rotational levels that begin with $N_T=2$.  An
$S_1 \sim T_3$ interaction with the $F_1$ component of such a level
follows the selection rule $\Delta N = -1$, and would first turn on at
$J'=N_S \geq 3$, as observed in the spectrum.  All other combinations
of $\Delta K$ and $\Delta N$ lead to $S_1 \sim T_3$ interactions
starting at $J' \geq 1$ or $2$.  Thus, we can assign $K_T=2$ for the
second $T_3$ doorway level.

For the $F_1$ component of a triplet level, the energy of the triplet
relative to the singlet has a negative slope with respect to $J'$ (see
Figure \ref{fig:rotational-energy-differences}).  At $J'=3$, where the
interaction with the second doorway level begins, the $F_1$ component
is $6B \simeq 6$ \rcm\ lower in energy than the $F_2$ component.  As
$J'$ increases, the relative energy of the $F_1$ component decreases
by $2B$ per $J'$.  Since the interaction must occur through an $F_1$
component in order to turn on at $J'=3$, and since the relative energy
of the $F_1$ component is appreciably lower than that of the other two
components, we must conclude from the assignment of $K_T$ that the
second $T_3$ doorway level lies at higher energy than the singlet
level.
% This is, in fact, what is deduced from in the dependence of
% intensity-weighted center of gravity on delay time.

The assignment of $K_T=2$ for the second doorway state has additional
consequences.  The $K_T=2$ doorway level, higher in energy than the
singlet level, must be accompanied by a $K_T=1$ doorway level at lower
energy.  % A diagram of the relevant energy level structure is given in
% Figure \ref{fig:double-doorway-k-levels}.
The energy separation
between the $K=1$ and $2$ sublevels of the same vibrational level is
$3(A_T-B_T)$, where $A_T$ is the a-axis rotational constant for the
triplet level in question.  According to \emph{ab initio}
calculations, the $A_T$ rotational constants for $T_3$ vibrational
levels vary between 10 and 30 \rcm \cite{thom07}.  This places the
$K_T=1$ sublevel about 30$-$90 \rcm\ lower in energy than the second
$K_T=2$ doorway level.

% \begin{figure}
%   \caption{\TODO:{Create figure from sketch.  It will basically just
%       show a singlet $K=1$ vibrational level between the $K=1$ and
%       $K=2$ sublevels of a triplet.}}
%   \label{fig:double-doorway-k-levels}
%   \centering

%   \vspace{4in}
%   [DIAGRAM OF RELATIVE\\ 
%    DOORWAY ENERGY LEVELS]
%   \vspace{4in}

% \end{figure}

Spin-orbit matrix elements between $S_1$ and $T_3$ are limited by
vibrational overlap factors to a magnitude of approximately $\lesssim
1$ \rcm.  Therefore, the rotationless energy of the second ($K_T=2$)
doorway cannot be more than about $20$ \rcm\ above the singlet
sublevel, for appreciable mixing to occur.  It follows that the
rotationless energy of the $K_T=1$ doorway must be lower than the
singlet level.  Furthermore, the vibrational overlap factors for two
sublevels of the same vibrational level are identical, and the
relative spin-orbit matrix elements are determined only by rotational
factors.  Using the rotational factors calculated in Section
\ref{theory1}, we find that the $K_T=1$ sublevel of the same
vibrational level as the second doorway has matrix elements about 5
times larger than those of the $K_T=2$ sublevel, and have opposite
sign \cite{stevens73}.  Because the energy denominators between the
singlet sublevel and the $K_T=1,2$ sublevels also have opposite phase,
the spin orbit matrix elements connecting these two sublevels to the
singlet must have the same phase.

However, we observe that the overall fluorescence lifetime is at a
minimum in the LIF spectrum at $J'=3$.  This is indicated by the low
intensity of the Q(3) transition in the delayed-LIF spectrum.  A short
fluorescence lifetime indicates less mixing between the $S_1$
rotational level and the doorway state, resulting from a smaller total
$S_1 \sim T_3$ matrix element at $J'=3$.  For the interaction with the
second doorway level to cause a decrease in the total $S_1 \sim T_3$
matrix element, the mixing amplitude of the second doorway level must
interfere \emph{destructively} with that of the first doorway level.
Since the mixing amplitudes of the $K_T=1$ and $K_T=2$ sublevels of
the second doorway interfere constructively, we must conclude that the
first doorway belongs to another $T_3$ vibrational level.

Analysis of the LIF spectrum vs. time delay has led to the
determination of the relative energy and $K$-assignments of two
energetically distant $T_3$ doorway sublevels in the $2^13^2$ \Ka{1}
$\leftarrow$ $0_0$ spectrum of acetylene \astate.  Such a conclusion
is further supported by the general appearance of the SEELEM spectrum
of this vibronic transition, which is strikingly similar to the
delayed fluorescence spectrum.  Peak positions in the delayed
fluorescence spectrum are matched in the SEELEM spectrum, even for
weak lines, for instance at 45675.9 \rcm\ and 45671.9 \rcm.  
% Agreement
% between the SEELEM spectrum and delayed fluorescence spectrum is
% expected in the absence of a level crossing between the singlet level
% and a $T_3$ doorway \cite{altunata01}.


\subsection{The $2^23^1$ \Ka{1} sublevel: a local $T_3$
  perturbation in the presence of small $S_1 \sim T_3$ matrix
  elements}

% Spectrum: Nov06d, see ``similitude'' calculations, p.62 of Sep
% 2006--Jan 2007 notebook.

%%%%%%%%%%%%%%%%%%%%%%%%%%%%%%%%%%%%%%%%%%%%%%%%%%%%%%
%%
%% INSERT 2^2 3^1 FIGURES HERE
%%
%%%%%%%%%%%%%%%%%%%%%%%%%%%%%%%%%%%%%%%%%%%%%%%%%%%%%%

\begin{figure}
  \caption{Simultaneously recorded SEELEM (upper trace) and LIF (lower
    trace) spectra of the $2^23^1$ \Ka{1} sublevel of the \astate\
    state of \ce{C2H2}.  The LIF spectrum is integrated in two time
    regions: an early time window ($0.5\tau_s-2\tau_s$, solid trace)
    and a delayed time window ($8\tau_s-12\tau_s$, dashed trace).  The
    upper states of the Q(1) and R(0) transitions, which have the same
    quantum number $J'=1$ but opposite parities, are shifted in
    opposite directions in the delayed fluorescence spectrum.}
  \label{fig:spectrum-2231}
  \centering
  \vspace{1cm}
  \includegraphics[width=7in,angle=90]{spectrum-2231-q4r3.pdf}
\end{figure}

\begin{figure}
  \caption{Dependence of the intensity-weighted center of gravity on
    delay for a series of rovibronic transitions, Q(1$-$4) (top), and
    R(0$-$3) (bottom), in the LIF spectrum of $2^23^1$ \Ka{1}.  As
    delay increases, the center of gravity for the Q(1) transition
    rapidly increases to its final value, where it matches the peak of
    the SEELEM distribution at 46007.87$+$0.03 \rcm.  For the R(0)
    transition, the center of gravity decreases to 46010.35$-$0.3
    \rcm\ at a delay of 18$\tau_s$.}
  \label{fig:2231-cog-delay}
  \centering
  \vspace{5mm}
  \includegraphics[width=6in]{2231-q1234-cog-delay.pdf}
  \includegraphics[width=6in]{2231-r0123-cog-delay.pdf}
\end{figure}

%%%%%%%%%%%%%%%%%%%%%%%%%%%%%%%%%%%%%%%%%%%%%%%%%%%%%%
%%
%% END OF 2^2 3^1 FIGURES
%%
%%%%%%%%%%%%%%%%%%%%%%%%%%%%%%%%%%%%%%%%%%%%%%%%%%%%%%

Although the $2^23^1$ \Ka{1} sublevel has the highest energy of the
four sublevels discussed in this study, it interacts most weakly with
the local manifold of $T_{1,2}$ levels.  This results from having only
one quantum of the $\nu_3$ vibration, which controls the overall
magnitude of $S_1 \sim T_3$ doorway matrix elements.

The spectrum of the $2^23^1$ \Ka{1} $\leftarrow$ $0_0$ subband is
shown in Figure \ref{fig:spectrum-2231}.  In addition to the overall
SEELEM:LIF intensity ratio, the small magnitude of $S_1 \sim T_3$
doorway matrix elements is indicated by the width of the SEELEM
intensity envelope surrounding each singlet transition.  This topic is
discussed at length in Chapter 2.  Briefly, the width of the SEELEM
spectrum is a measure of the energy range over which the singlet
level, through interaction with $T_3$ doorways, is able to lend
approximately 0.25\% fractional singlet character to eigenstates
within the local manifold of $T_{1,2}$ levels.  In the case of
$2^23^1$ \Ka{1}, the width of the SEELEM envelope surrounding each
singlet transition is on the order of 0.4 \rcm, much narrower than the
average spacing between rotational lines.  In the next section, we
will contrast this with the observed width of SEELEM intensity
envelopes in a strongly interacting subband.

% Using a formula
% derived in Chapter 2, the FWHM of the SEELEM spectrum can be related
% to the $S_1 \sim T_3$ doorway matrix element, $H_{S_1,T_3}$, and the
% average $T_3 \sim T_{1,2}$ matrix element, $\braket{H_{T_3,T_{1,2}}}$:
% \begin{equation}
%   \Delta E_{FWHM} = \frac{2\sqrt{2} \; \tau_s}
%                        {e \; \tau_{\text{flight}}}
%   \left \lvert
%     \frac{H_{S_1,T_3} \braket{H_{T_3,T_{1,2}}}}{\Delta E_{S_1,T_3}}
%   \right \rvert,
% \end{equation}
% where $\Delta E_{S_1,T_3}$ is the energy difference between the
% singlet level and the $T_3$ doorway, and $\tau_{\text{flight}}$ is the
% flight time in the SEELEM apparatus, about 310 \microsec.  Solving for
% the quantity $\lvert H_{S_1,T_3} \braket{H_{T_3,T_{1,2}}} / \Delta
% E_{S_1,T_3} \rvert$ and using the approximate FWHM of 0.4 \rcm, we
% find that
% \begin{equation}
% \end{equation}

The dependence of the intensity-weighted center of gravity for each
transition terminating in $2^23^1$ \Ka{1} is shown in Figure
\ref{fig:2231-cog-delay}.  With the exception of the Q(1) and R(0)
transitions, the center of gravity does not shift by more than 0.02
\rcm\ from its initial position.  For a singlet level with such a
small spin-orbit matrix element with a doorway state, any small change
in center of gravity, which might arise from the effects of an
energetically distant $T_3$ level, is not expected to appear until
delay times in excess of $15\tau_s$.

% \NOTE{Bob suggests interference effects here, but the basis for this
%   is not clear to me.} 
The anomalous behaviour of the center of gravity for the Q(1) and R(0)
transitions is once again evidence of a weak, local $T_3$ perturbation
at $J'=1$.  The upper states of the Q(1) and R(0) transitions have the
same $J'$, but opposite parity ($f$- and $e$-symmetry, respectively).
The energy difference between the \emph{e}- and \emph{f}-symmetry
components is $\Delta E_{e-f}=+0.13$ \rcm.  The perturbation in the
\emph{e}-symmetry singlet transition, R(1), results in a shift to
lower frequency, with a magnitude of approximately $-0.03$ \rcm, while
the perturbation shifts the \emph{f}-symmetry singlet level to higher
frequency, with a magnitude of approximately $+0.03$ \rcm.  The
splitting between the asymmetry components of the $T_3$ perturber is
therefore less than $0.13-0.03-0.03=0.07$ \rcm\ at $J'=1$.  An
asymmetry splitting of this magnitude would be unusually small for a
$T_3$ level with $K_T=1$.  As a result, we conclude that the perturber
is likely a level with $K_T=2$.  Only one component of a $K_T=2$ level
may interact with a singlet level at $J'=1$, and that is the $F_3$
component, where $N_T=J+1$.  An interaction with the $F_1$ or $F_2$
components would require $N_T$ to be $0$ or $1$; these levels cannot
exist when $K_T=2$.  Because the interaction occurs via the $F_3$
component of the triplet level at $J'=1$, the rotationless energy of
the triplet level must be located approximately $4B \simeq 4$ \rcm\
below the singlet level.

Again, the assignment of the $K_a$ quantum number for a $T_3$ level
observed in the spectrum has allowed us to infer its energy relative
to the singlet.  At $J'=2$, the nearest component of this weak $T_3$
perturber lies $2B \simeq 2$ \rcm\ above the singlet level.  The
resultant increase in squared energy denominator makes the
extra line $(2.0/0.03)^2 \simeq 4500$ times weaker at $J'=2$, thus it
is not observed.

%
% Selection rules: Q   -> e-f
%                  P,R -> e-e, f-f
% 











\subsection{The $3^3$ \Ka{2} sublevel: spectral patterns in the
  presence of large $S_1 \sim T_3$ matrix elements}

% Spectrum:  Jan 22C, p.31,34 of 1/2007--3/2007 notebook.

%%%%%%%%%%%%%%%%%%%%%%%%%%%%%%%%%%%%%%%%%%%%%%%%%%%%%%
%%
%% INSERT 3^3 K^2 FIGURES HERE
%%
%%%%%%%%%%%%%%%%%%%%%%%%%%%%%%%%%%%%%%%%%%%%%%%%%%%%%%

\begin{figure}
  \caption{Simultaneously recorded SEELEM (upper trace) and LIF (lower
    trace) spectra of the $3^3$ \Ka{2} sublevel of the \astate\ state
    of \ce{C2H2}.  The LIF spectrum is integrated in two time regions:
    an early time window ($0.5\tau_s-2\tau_s$, solid trace) and a
    delayed time window ($10\tau_s-18\tau_s$, dashed trace).  The
    individual transitions each split into at least two strongly mixed
    components.  Although the energy splitting between the components
    is on the order of the experimental resolution, the splitting is
    discernible because the nominal singlet and triplet components
    have different relative intensities in the early and delayed-LIF
    spectra.  One splitting in the R(4) transition is barely resolved
    in this spectrum.}
  \label{fig:spectrum-33k2}
  \centering
  \includegraphics[width=7in,angle=90]{spectrum-33k2-r1r7.pdf}
\end{figure}

\begin{figure}
  \caption{Dependence of the intensity-weighted center of gravity on
    delay for a series of individually resolved transitions, R(1$-$7),
    in the LIF spectrum of the $3^3$ \Ka{2} sublevel.  The individual
    transitions have an overall bias toward lower energies at long
    delay times, indicating an interaction with a $T_3$ doorway level
    at lower energy than the $S_1$ level.}
  \label{fig:33k2-cog-delay}
  \centering
  \includegraphics[width=6in]{33k2-r123456-cog-delay.pdf}
\end{figure}

%%%%%%%%%%%%%%%%%%%%%%%%%%%%%%%%%%%%%%%%%%%%%%%%%%%%%%
%%
%% END OF 3^3 K^2 FIGURES
%%
%%%%%%%%%%%%%%%%%%%%%%%%%%%%%%%%%%%%%%%%%%%%%%%%%%%%%%

The $3^3$ \Ka{2} sublevel is the higher-energy sibling of $3^3$
\Ka{1}, which has been studied in great detail due to a perturbation
by a $T_3$ doorway level with matrix element $\simeq 0.1$ \rcm.  The
$T_3$ perturber observed in $3^3$ \Ka{1} has been assigned as the
$F_2$ component of $K_T=1$.  Because
%\begin{inparaenum}[\itshape a\upshape)]
%\item 
its energy relative to $S_1$ tunes slowly with $J'$,
%\item 
it interacts with both parities of the singlet, and
%\item 
the perturbation is present at $J'=1$
%\end{inparaenum}
\cite{mishra04}.  It has been suggested that the $K_T=0$ sublevel of
this perturber is responsible for the large Zeeman anticrossing
observed in $3^3$ \Ka{0} \cite{thom07, dupre93}.  To account for this,
the $A$-rotational constant of the perturbing $T_3$ level would have
to closely match the $A$-rotational constant for the singlet level.
However, it is unlikely that an energetic near match in the $K=0$ and
$1$ sublevels, separated by $1A \simeq 15$ \rcm, will extend to $K=2$,
which would be $3A \simeq 45$ \rcm\ higher in energy.  In the absence
of a local $T_3$ perturbation, the $3^3$ \Ka{2} sublevel provides an
excellent opportunity to examine a singlet sublevel which is known to
be capable of large vibrational overlap with $T_3$ levels.

The $3^3$ \Ka{2} sublevel is not accessible from the ground state of
acetylene due to $K' - \ell" = \pm 1$ selection rules.  However, the
spectrum of the $3^3$ \Ka{2} $\leftarrow$ $4_1$ hot band transition
was observed in our apparatus without heating the nozzle.  The
simultaneously recorded SEELEM/LIF spectrum of the R-branch of this
transition is shown in Figure \ref{fig:spectrum-33k2}.  Another
interloping band\footnote{This band is currently unassigned because it
  is too broadly overlapped by other strong bands.  The lines
  referenced here possibly belong to a hot band with a Q-head near
  44736 \rcm.  Another possible assignment is to the second highest
  $K=2$ sublevel of the $B^4$ polyad \cite{merer-private}.} is present
in the spectrum with low intensity, giving rise to the weak lines at
44740.3, 44742.2, and 44743.6 \rcm.  No intensity alternation is
present in the rotational series, consistent with expectations for a
band that originates from the $4_1$ vibrational level of the ground
electronic state.

Evidence of strong mixing with the local manifold of $T_{1,2}$ levels
is present in both the SEELEM and LIF detection channels.  The SEELEM
intensity envelope surrounding each singlet transition exceeds the
spacing between adjacent rotational lines, about 1.5 \rcm\ at low $J'$.
This is at least 4 times the width observed for the transitions in
$2^23^1$ \Ka{1}, which contains only one quantum of excitation in
$\nu_3$.  The increased width of the local SEELEM intensity envelope
has the effect of creating an envelope of SEELEM intensity that
extends across the entire spectrum.  This effect is also observed in
the $3^3$ \Ka{1} $\leftarrow$ $0_0$ spectrum (see reference
\cite{humphrey97}, for example).  

However, we observe no large, systematic splittings at adjacent
rotational levels of $3^3$ \Ka{2}, which would result from a level
crossing with the $F_2$ component of a near-degenerate $T_3$ doorway
vibrational level, as in $3^3$ \Ka{1} \cite{mishra04}.  Instead,
narrow line splittings on the order of 0.05 \rcm\ are evident in the
LIF spectrum of each transition, even in the early time-gated LIF
spectrum.  Although most of the line splittings are slightly smaller
than the laser resolution, the individual components have different
fractional $S_1$ characters, and hence different lifetimes.  This
produces different relative intensities in the early and delayed-LIF
spectra, which permits the splittings to be determined by comparing
the spectra.  Two of three strong components are barely resolved in
the early-time LIF spectrum of the R(4) transition.  In contrast to this
strongly $S_1 \sim T_3$ mixed sublevel, the weakly mixed $2^23^1$
\Ka{1} sublevel contains no such splittings.

The dependence of the intensity-weighted center of gravity for each
transition in $3^3$ \Ka{2} is shown in Figure
\ref{fig:33k2-cog-delay}.  The transitions display an overall bias
toward lower energy, indicating the presence of a $T_3$ doorway level
to lower energy.  The $F_3$ components of such a doorway level would
approach the singlet level from below at a rate of $2B$ per $J'$,
while the $F_2$ components would remain energetically distant and the
$F_1$ components would rapidly tune away.  That no crossing is
observed by at least $J'=6$ means that the doorway must lie at least
$6\times2B + 2B \simeq 14$ \rcm\ lower in energy than the singlet
level.

The center of gravity generally decreases as a function of time delay
for the transitions in the spectrum.  The dependence is most
pronounced for the $J'=2$ rotational level, and the magnitude of shift
is diminished as $J'$ is increased.  However, if the $F_3$ component
of the doorway level is rapidly approaching the singlet, the overall
mixing angle is expected to increase in magnitude.  Why does the
magnitude of the change in center of gravity not also increase?  This
paradox is explained by the large fractional singlet character in
nominal $T_{1,2}$ levels, which is induced by strong mixing with the
doorway level.  As several states of nominal $T_{1,2}$ electronic
character borrow more intensity, their lifetime decreases.  As the
nominal singlet level lends out more fractional character, its
lifetime increases.  The result is a lack of contrast between the
intensities of the nominal singlet and nominal triplet eigenstates as
a function of delay time, resulting in smaller changes in the
intensity-weighted center of gravity.

% However, evidence for the approach toward $S_1$ of energetically
% distant $T_3$ doorway levels is still available from the spectrum.
% The most reliable indicator is obtained from a comparison of
% band-integrated center of gravity of the early-LIF spectrum vs. the
% SEELEM spectrum.  We discuss this in the following section.

% \subsection{Evidence for energetically distant $T_3$ doorway levels in
%   band-integrated center of gravity}

% %%%%%%%%%%%%%%%%%%%%%%%%%%%%%%%%%%%%%%%%%%%%%%%%%%%%%%
% %%
% %% INSERT CENTER OF GRAVITY TABLE HERE
% %%
% %%%%%%%%%%%%%%%%%%%%%%%%%%%%%%%%%%%%%%%%%%%%%%%%%%%%%%

% \begin{table}[p]
%   \caption{Band-integrated center of gravity measurements from
%     simultaneously recorded SEELEM/LIF spectra of \astate\ acetylene.
%     The SEELEM$-$LIF center of gravity is offset to lower energy for
%     Q-branch measurements and offset to higher energy for R-branch measurements, 
%     indicating an overall increase in relative SEELEM:LIF intensity
%     with $J'$.  Such $J'$-dependent behavior is predicted in the
%     presence of energetically distant $T_3$ doorway levels.}
%   \label{table:integrated-cog-shifts}

%   \centering
%         \begin{tabular}{lllll}
%         \\[1cm]
%         \multicolumn{3}{l}{Q-branch measurements} \\
%         \toprule
%         Level & Integration region & \multicolumn{2}{l}{Center of gravity} & Offset \\
%         \cmidrule{3-5}
%         & & LIF & SEELEM & SEELEM$-$LIF \\
%         \midrule
%         $\nu'_2+2\nu'_3$ & 45671.75$-$76.30 & 45674.60 & 45674.09 & $-0.51$ \\
%         $2\nu'_2+\nu'_3$ & 46004.50$-$08.50 & 46007.29 & 46007.23 & $-0.06$ \\
%         $2\nu'_3+2\nu'_4$ & 45807.30$-$11.00 & 45810.09 & 45809.72 & $-0.37$ \\
%         \bottomrule
%         \end{tabular}

%         \begin{tabular}{lllll}
%         \\[1cm]
%         \multicolumn{3}{l}{R-branch measurements} \\
%         \toprule
%         Level & Integration region & \multicolumn{2}{l}{Center of gravity} & Offset \\
%         \cmidrule{3-5}
%         & & LIF & SEELEM & SEELEM$-$LIF \\
%         \midrule
%         $2\nu'_2+\nu'_3$ & 46009.50$-$16.50 & 46012.11 & 46012.49 & $+0.38$ \\
%         $3\nu'_3 \:\; K'\!=\!2$ & 44738.50$-$47.50 & 44741.79 & 44742.67 & $+0.88$ \\
%         $2\nu'_3+2\nu'_4$ & 45812.00$-$20.50 & 45814.67 & 45815.57 & $+0.89$ \\
%         \bottomrule
%         \end{tabular}
% \end{table}

%%%%%%%%%%%%%%%%%%%%%%%%%%%%%%%%%%%%%%%%%%%%%%%%%%%%%%
%%
%% END CENTER OF GRAVITY TABLE
%%
%%%%%%%%%%%%%%%%%%%%%%%%%%%%%%%%%%%%%%%%%%%%%%%%%%%%%%

% \TODO{Write this section.  The shift in integrated SEELEM center of
%   gravity, relative to LIF, is an effect of a $J$-dependent SEELEM:LIF
%   intensity ratio.  Since one component, either $F_1$ or $F_3$, of the
%   nearest doorway always approaches the singlet in energy as a
%   function of $J$, we expect the SEELEM:LIF intensity ratio to always
%   increase with $J$.  If the quantum number $J'$ is increasing with
%   energy (R-branches), the c.o.g. offset should be positive.  If the
%   quantum number $J'$ is decreasing with energy (Q-branches), the
%   c.o.g. offset should be negative.  This is shown in Table
%   \ref{table:integrated-cog-shifts}.}





























\section{Conclusion}

The mechanism of doorway-mediated interaction by an energetically
distant $T_3$ level skews the SEELEM spectrum of nearby, nominal $T_{1,2}$
eigenstates, resulting in a center of gravity shift between the LIF
and SEELEM spectra.  Additionally, when viewed in successive time
windows, the center of gravity of the LIF spectrum exhibits evolution
toward the limiting behavior exhibited in the SEELEM spectrum.  A
simple model can be used to show that strong mixing between the
singlet level and the mediating $T_3$ level causes a gradual shift in
the center of gravity, while weak mixing with the doorway level
induces a more delayed and rapid shift in the center of gravity, with
respect to the $S_1$ lifetime.

Rotational selection rules for $S_1 \sim T_3$ spin-orbit interaction
give rise to $J$-dependent effects in the LIF/SEELEM spectrum.  As $J$
increases, the $F_1$ or $F_3$ component of every distant $T_3$ level
approaches the singlet at a rate $d\Delta E / dJ$ of approximately
$2B_T$ ($\sim$2 \rcm\ per $J$).  In the spectrum, the result is a
shift in the LIF-SEELEM center of gravity when integrated across an
entire branch of transitions.  The effect not only leads to strong
$J$-dependent changes in the patterns of $T_3$-mediated coupling, but
also ensures detection and assignment of $S_1 \sim T_3$ level
crossings at relatively low values of $J$.

For every $S_1$ vibrational level, one spin component of a $T_3$
doorway level is rapidly approaching with $J$.  Because of this,
further LIF/SEELEM spectroscopy of the acetylene \AtoX\ transition
will be fruitful and informative.  We highlight some candidates for
future investigation.
\begin{enumerate}
\item Levels which exhibit long lifetimes or quantum beats in the LIF
  spectrum
\item levels with unassigned perturbations or splittings,
\item other $3^2B^2$ polyad members
\item other $K$-sublevels of the Franck-Condon active levels studied
  here.  
\end{enumerate}
% \TODO{Specify candidate bands specifically, give energies.  This
%   section needs some work, overall.}

The appearance of population quantum beats in the spectrum indicates a
splitting on the order of 80 MHz or less.  Quantum beat waveforms
follow a well-defined analytical expression, and an analysis of
quantum beats determines both the matrix element and zero-order energy
spacing of the levels involved.  The matrix elements gained from an
analysis of zero-field quantum beats in the acetylene spectrum can
serve as probes of the magnitude of local matrix elements between the
nominal $S_1$ bright state and neighboring $T_{1,2}$ dark states.
Furthermore, the study of Zeeman quantum beats at low magnetic fields
can provide a method for distinguishing between triplet perturbations
and perturbations from other singlet levels, such as $S_1$ levels
which are localized in the \emph{cis} geometry of the \astate\ state,
because singlet states do not tune in a magnetic field.

% \bibliography{master} 
% \bibliographystyle{plain}
% \end{document}
% LocalWords:  redshifted blueshifted sublevel pdf png intersystem sublevels ns
% LocalWords:  Zeeman anticrossing photodissociation initio Dupr vib perturber
% LocalWords:  rovibrational prolate rotationless Hund's vvvv Condon prefactor
% LocalWords:  collimated timespan Hamamatsu predissociation perturbers polyad
% LocalWords:  anharmonic subband dataset vibronic cis rovibronic
