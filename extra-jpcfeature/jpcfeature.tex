\documentclass[12pt]{mitthesis} 
\usepackage{amsmath, amssymb, braket, color} 
\usepackage[pdftex]{graphicx}

\newcommand{\TODO}[1]{\textcolor{magenta}{\textbf{TODO:} #1}}
\newcommand{\POINT}[1]{\textcolor{magenta}{\emph{#1}}}
\newcommand{\rcm}{cm$^{-1}$}
\newcommand{\tnt}{$S_1$ $3\nu_3$}

\hyphenation{acetylene}
\hyphenation{Hamiltonian}

\begin{document}

\chapter{The Spectral Signatures of Doorway-Mediated Intersystem
  Crossing}

\section{Introduction}

The $3\nu_3$ $K=1$ level of $S_1$ acetylene has been heavily studied,
in part because it plays host to several interesting energy
resonances.  Firstly, a singlet vibrational level involving 4 quanta
of the non-symmetric bending modes $\nu_4$ and $\nu_6$ is near
degenerate with $3\nu_3$ at $J=0$.  However, this singlet perturber
quickly detunes from the \emph{trans}-bending level due to a
relatively large difference in rotational constant.  Secondly, a
triplet vibrational level of the $T_3$ electronic state crosses the
$S_1$ $3\nu_3$ level near $J=3$.  The matrix element between $S_1$
$3\nu_3$ and the local triplet perturber is approximately $0.1$ \rcm.

Recently, DeGroot \emph{et al.} have observed in photoelectron kinetic
energy experiments that the coupling between $S_1$ $3\nu_3$ and the
manifold of $T_{1,2}$ states \emph{increases} with rotational quantum
number.  Over the same range of $J$, the local $T_3$ doorway level
tunes \emph{away} from $S_1$ $3\nu_3$.  If this local $T_3$ doorway
level were the sole mediator for intersystem crossing between $S_1$
$3\nu_3$ and the manifold of $T_{1,2}$ levels, one would expect the
$S_1 \sim T_{1,2}$ coupling to decrease with the $S_1 \sim T_3$ energy
difference.  The authors address this issue by proposing that an
unobserved, energetically distant $T_3$ level interacts more strongly
with the $S_1$ $3\nu_3$ level.  An interference effect between the
local and distant doorways, then, could cause the $S_1 \sim T_{1,2}$
coupling to decrease near the local $S_1 \sim T_3$ crossing.

This hypothesis is supported by recent \emph{ab initio} calculations
of $S_1 \sim T_3$ vibrational overlap integrals.  To a very good
approximation, the spin-orbit matrix element between two vibrational
levels of different electronic surfaces is given by the product of
electronic spin-orbit matrix element and the vibrational overlap
integral.  Cui and Morokuma have computed an $S_1 \sim T_3$ electronic
spin-orbit matrix element of 13.7 \rcm.  Thom \emph{et al.}  reported
nonzero overlap integrals for six vibrational levels lying within
$100$ \rcm of \tnt.  The integrals range almost three orders of
magnitude.  The largest vibrational overlap integral, $0.12$, leads to
a vibronic spin-orbit matrix element of 1.6 \rcm, ten times larger
than the matrix element observed for the local $T_3$ perturber in
\tnt.

A similar $T_3$ ``distant doorway'' interaction has also been proposed
to explain the difference in singlet $\sim$ triplet coupling that is
observed for the $\nu_2 + \nu_3 + 2\nu_6$ and $\nu_2 + \nu_3 + 2\nu_4$
levels.  Strong anharmonic and Coriolis interactions cause these
levels to be almost 50:50 mixed in zero-order character.  However,
interference between these matrix elements causes only one of the
levels, $\nu_2 + \nu_3 + 2\nu_4$, to mix with the $\nu_2 + \nu_3 +
\nu_4 + \nu_6$ level.  This quality is proposed to control the
interaction with mediating but energetically distant $T_3$ levels.

We propose that such ``distant doorway'' interactions are in fact the
usual case for intersystem crossing in $S_1$ acetylene.  To
investigate the global characteristics of singlet$\sim$triplet
coupling, we have recorded simultaneous Laser Induced Fluorescence
(LIF) and Surface Electron Ejection by Laser Excited Metastables
(SEELEM) spectra of a series of vibrational levels in $S_1$ acetylene.
These levels lie in the energy region surrounding the $3\nu_3$ level
but below the first dissociation limit of the molecule.  In none of
these levels do we observe a local $T_3$ doorway state.  With our
choice of vibrational levels, we address the role of $\nu_2$ in the
coupling between the $S_1$ and $T_3$ electronic states.

The spectral signatures of a local doorway level in LIF and SEELEM
have been well developed.  The related question we address in our
current analysis is: what are the spectral patterns, in LIF and
SEELEM, of an intersystem crossing process mediated by nonlocal
triplet doorway levels?  What information about the coupling is
available from the spectrum, and how reliable is it?  Obviously, the
most valuable information would be the relative energy of a mediating
doorway level, or the quantity $H_{S_1 T_3}/\Delta E$.

In our analysis, we develop a new method to relate the LIF and SEELEM
spectra, invesigating the distribution of LIF spectral intensity using
a time-delayed gate.  This technique maps the development of the
spectrum from the characteristics of the bright state to the final
distribution of metastable states observed in the SEELEM spectrum.  In
our development of this method, we demonstrate that the SEELEM
spectrum is an extreme case of discrimination against large amounts of
bright state character.

\section{Experiment}

The experimental apparatus has been described previously.  Briefly, a
molecular beam of 100\% acetylene is formed from a 0.3 mm pulsed
nozzle (Jordan Valve). \TODO{Finish experimental section.} 



\end{document}