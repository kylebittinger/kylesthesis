\documentclass[12pt]{article}
\usepackage{amsmath, amssymb}

\setlength\parskip{4mm}

\begin{document}

\section{A simple Bayesian model for LIF}


Traditional least-squares fitting offers answers to questions like,
``What is the best fit to this data?''  I would like to move to
questions like, ``What is the best model?''  More specifically, ``How
many spectral lines are in this data?''  To answer these questions, we
need to know something about the probability of making a measurement
given certain parameters in a model.

We begin by addressing the sources of error in our measurements:
\begin{enumerate}
\item Those that vary shot-to-shot and depend on signal strength (e.g.
  laser intensity fluctuations leading to varying amounts of
  fluorescence)

\item Those that vary shot-to-shot but are independent of signal
  strength (e.g. scattered light and voltage noise leading to baseline
  fluctuations in LIF; dark counts in SEELEM)

\item Those that are inherent to experiments that count rare events.
  (Poisson statistics in SEELEM, not present in LIF)
\end{enumerate}
At the end of each section, I'll write down the things we can do to
actually characterize these noise sources for our experiment.

\subsection{Shot-to-shot laser intensity fluctuation}

Our goal is to derive an expression for measurement error in
signal-averaged traces. 

First, consider a trace where no laser light enters the apparatus, but
the rest of the experiment is otherwise running.  The trace is
averaged over a number of trials $N$, as it would be while taking a
spectrum.  This is essentially a control experiment; the trace
consists of whatever voltage noise or offset is present when the
molecules do not interact with the light field.  Several varieties of
noise arise in a photomultiplier tube under these conditions: shot
noise, $1/f$ noise, and stray photon counts.

(Discuss references for detector/PMT noise.) 

In any case, this noise can be measured in the laboratory before an
experiment.  Results from a typical set of trials are shown in figure
1 (TODO).  We parameterize the voltage noise at each time $t$ in the
trace as a gaussian distribution.  This model assumes that the signal
at each time is independent of its neighbors, and that successive
trials are independent of those following and preceeding them.  The
latter assumption is, in general, good, while the former assumption is
generally not true for time-series data.  Nevertheless, this simple
model of noise in time traces has been shown to work surprisingly
well, for example in NMR data (cite Bretthorst), so we adopt it as a
starting point.

The probability of measuring a voltage $s_0$ at time $t$ is thus
\begin{equation}
  p(s_{0} \vert \hat{s}_{0}, \sigma_{0}) 
  = \frac{1}{\sqrt{2 \pi} \; \sigma_{0}} \,
  exp 
  \left[
    -\frac{(s_{0}-\hat{s}_{0})^2}{4 \; \sigma_{0}^2}
  \right] ,
\end{equation}
where $\hat{s}_{0}$ is the mean value of the measurement at time $t$,
and $\sigma^2_{0}$ is the variance in this measurement.  As seen in
the figure, this distribution parameterizes the measurements well.  If
the effects of other triggered noise sources in the laboratory are
negligable, then we can remove the functional dependence of $\sigma_0$
on $t$, and the null model $M_0$ reduces to one parameter $\sigma_0$.

Now, consider a trace where the laser light is allowed to enter the
apparatus, but is tuned off resonance.  We label this model $M_1$.  A
signal arising from scattered laser light is present, along with
associated noise. Measurements from such a set of trials are shown in
figure 2 (TODO).

It is readily apparent from the histograms in figure 2 that the
variance of measured signal from scattered laser light is very nearly
proportional to its mean value.  This linear dependence arises from
the liner response of the photomultiplier tube to the intensity of
scattered laser light.  We parameterize the noise arising from
fluctuating laser intensity in terms of the voltage noise in $M_0$:
\begin{equation}
  \sigma_1(\hat{s}_1)^2 = \sigma_0^2 + \beta \hat{s}_1
\end{equation}
The probability of measuring a signal $s_1$ under model $M_1$ retains
the form of a gaussian distribution, only now a functional dependence
of $\sigma_1^2$ on $\hat{s}_1$ is included:
\begin{equation}
  p(s_1 \vert \hat{s}_1, \sigma_1) 
  = \frac{1}{\sqrt{2 \pi} \; \sigma_1(\hat{s}_1)} \,
  exp 
  \left[
    -\frac{(s_1-\hat{s}_1)^2}{4 \; \sigma_1(\hat{s}_1)^2}
  \right] .
\end{equation}

To consider the effects of signal averaging, we evaluate the
probability of $N$ successive measurements,
\begin{equation}
  P(D \vert \hat{s}_1, \sigma_1) 
  = \prod_{N} \frac{1}{\sqrt{2 \pi} \; \sigma_1(\hat{s}_1)} \,
  exp 
  \left[
    -\frac{(s_1-\hat{s}_1)^2}{4 \; \sigma_1(\hat{s}_1)^2}
  \right] .
\end{equation}

STOPPED HERE

If we define a normalized gaussian function over x as
Then the probability of having some laser intensity $I$ on any given
laser shot is
\begin{equation}
  p(I) = gauss(I; I_{ave},sigma)
       = gauss(I; I_{ave}, (a*I_{ave}))
\end{equation}
where $I_{ave}$ is the average laser intensity.

The probability of measuring a signal S after 100 shots at this laser
frequency can be parameterized as
\begin{equation}
  p(S) = gauss(S; S_ave, (b*S_ave))
\end{equation}
So, the point is that this can be described with two parameters, and
that the deviation will be somewhat proportional to the average signal
level.

To measure this experimentally: we could tune the laser to a strong
LIF/SEELEM line, and take repeated measurements at the same frequency.
Plot a histogram if the intensity to verify an approximately gaussian
distribution.

To measure the (supposedly linear) dependence of signal stdev on
signal strength, we could repeat this measurement on several different
spectral lines.

\subsubsection{LIF Baseline Noise}

Baseline fluctuations in the experiment come from two sources: voltage noise
and scattered light (i.e. that which does not come from electrons traveling
through the PMT and that which does).

Because of the scattered light, the baseline fluctuations will depend weakly
on average laser intensity. However, we are going to ignore this for now,
because we expect this effect to change slowly over the course of the
experiment.

We're going to assume that we can parameterize the baseline fluctuations
with two parameters, and that we take enough measurements that their
distribution is approximately normal.

In this case, the probability of measuring a signal with baseline at S0 is:
\begin{equation}
  p(S_0) = gauss(S_0; ave(S_0),\sigma_{S_0})
\end{equation}
We can measure this in an experiment by tuning our laser to a frequency
where there is no spectral line, collecting data, and plotting a histogram
of the baseline signal intensity.

\subsubsection{SEELEM dark counts}

My opinion on this is that we should ignore this most of the time. We
probably have dark counts on the order of 1 per 100 shots or 1 per 1000
shots. This is much lower than our other noise sources.

When we have equipment errors that generate a lrage number of dark counts, I
think we should not use these data points in our analysis; it is very
unlikely that they will help us get closer to the right answer.

Regardless, dark counts should have an constant rate over time (independent
of when the laser fires), so we can measure their occurrence by looking in
the third SEELEM time window, which we measure during every experiment.

\subsection{Poisson Statistics}

There is an intrinsic probability for counting a certain number of
independent events from a process with constant rate. The probability
ditribution is the Poisson Distribution. It has one parameter: the
"expected" number of counts, which is just the rate multiplied by the time
of observation. (And it is also the mean of the distribution.)
\begin{equation}
  poisson (n; n_ave) = (e^(-n_ave)) * (n_ave^(n)) / n!
\end{equation}
And this is the probability of measuring n counts given $n_{ave}$.
\begin{equation}
  p(n) = poisson(n; n_ave)
\end{equation}
It goes without saying that we don't need to measure this property in the
experiment because it is determined a priori.

So, now we can calculate the probability of a certain signal in the
experiment for a given value of the "true" signal:

  Model parameters:
  $S_ave$  = proposed signal at some frequency and laser intensity
  $S0_ave$ = proposed baseline signal

  Noise parameters:
  $b$        = proportionality constant for signal deviation with signal
             intensity
  $S0_stdev$ = stdev of baseline noise

\end{document}