\documentclass[12pt]{article}
\usepackage{fullpage, amsmath, amssymb}
\usepackage[pdftex]{graphicx}
\usepackage[version=3]{mhchem}
\hyphenation{density}
\setlength{\parskip}{3mm}

\begin{document}

\section*{Acetylene Zeeman Anticrossing Experiments}

In a Zeeman anticrossing experiment, the laser is locked on an optical
transition in the molecule (we'll call this ``the singlet level'').
As the magnetic field is tuned, triplet-mixed levels tune through
resonance with the singlet.  Relative to the singlet level at
$E=0$, the energy of a mixed triplet level is (linear Zeeman effect)
\begin{equation}
E = g \, M_s \, \mu_B \, (B-B_0),
\end{equation}
where $g$ is the mixed-triplet level's $g$-factor, $M_s$ is the
lab-fixed projection of spin, $B_0$ is the field required to bring the
level's energy to 0, and $\mu_B$ is the Bohr magneton (constant).  An
excited electron which is not coupled to the molecular axis (Hund's
case \emph{b}) has a $g$-factor of $2.0023$.

When the singlet level and one of the weakly interacting background
levels are at exactly the same energy, they become 50:50 mixed.  When
this happens under the experimental conditions, half the molecules
decay by collision instead of fluorescence, and the fluorescence
intensity decreases by half.  The \emph{width} of the resonance is
proportional to the matrix element between the bright state and the
interacting level:
\begin{equation}
\Delta B = \frac{4\,V}{g\,\mu_B}
\end{equation}

What were the results and conclusions of these experiments?

\subsection*{Paper 1: Anomalous behavior of the anticrossing density}

Dupr\'{e}, Jost, Lombardi, Green Abramson, and Field.
\emph{Chem. Phys.}  \textbf{152}, 293 (1991).\\
Measured anticrossing spectra of $\nu_3=1-3$ rotationless levels,
field strength $0-8T$.

\begin{description}
\item[Observation: Rapid increase in density of mixed-triplet levels
  with $\nu_3$ excitation.]  Observed density of states is larger than
  number of $T_{1,2}$ levels, therefore the triplet levels must mix
  with and contaminate the denser manifold of $S_0$ levels.

  However, increasing the $T_{1,2} \sim S_0$ coupling by itself does
  not make more mixed-triplet levels accessible to cross the $S_1$
  level.  The $S_1$ level can only undergo triplet-mediated mixing
  with $S_0$ levels over an energy range on the order of the $S_1 \sim
  T$ matrix element.

  Therefore, coupling between $S_1$ and a subset of the triplet levels
  $T$ must increase to make the dense manifold of contaminated
  $T_{1,2} \sim S_0$ levels observable in the experiment.

\item[Observation: Rapid increase in anticrossing linewidth.]  Since
  the linewidth is proportional to the $S_1 \sim T$ matrix element,
  this supports the conclusion that the $S_1 \sim T$ coupling is
  increasing with excitation in $\nu_3$.

\item[Excluded Mechanism: Direct $S_1 \sim S_0$ internal conversion.]
  No mechanism for sudden increase in vibrational overlap integrals
  between relatively unmixed $S_1$ levels and essentially fully mixed
  $S_0$ levels over this energy range.

\item[Excluded Mechanism: $S_0$ dissociation limit increases total
  density of states.]  The increase in state density would be
  accompanied by a decrease in vibrational overlap integrals.

\item[Proposed Mechanism: $S_1 \sim T_3$ curve crossing.]  The
  observed $S_1 \sim T$ coupling could be induced by a sparse
  collection $T_3$ levels, mostly distant.  Question proposed by
  authors: too many broad anticrossings in $3\nu_3$ spectrum?

\item[Proposed Mechanism: $T_2$ linear isomerization barrier.]  The
  vibrational overlap between $T_2$ levels in the \emph{cis}-well and
  $S_1$ levels in the \emph{trans}-well would increase in the region
  of the linear turning point.
\end{description}

\subsection*{Paper 2: Characterization of a large singlet-triplet
  coupling}

Dupr\'{e} and Green. \emph{Chem. Phys. Lett.} \textbf{212}, 555
(1993).\\
Measured LIF spectrum of rotationless $3\nu_3$ level at 0T and near 7.075T

Ryan provides a useful summary of this paper in his article ``Studies
of Intersystem Crossing Dynamics in Acetylene,'' \emph{JCP}
\textbf{126}, 184307 (2007).  

A $T_3$ level with energy $\Delta E = \pm 6.67$ cm$^{-1}$ interacts
with the $J=K=0$ level of $3\nu_3$.  This strong interaction is very
well characterized by the Zeeman anticrossing data.  Many weaker
interactions with nominal $S_0$ levels can be seen perturbing the
strong $S_1 \sim T_3$ anticrossing.  The zero-field $S_1 \sim T_3$
matrix element is $0.29$ cm$^{-1}$ if $K_{T_3}=0$ or $0.58$ if
$K_{T_3}=1$.

\subsection*{Paper 3: Quantum beat spectroscopic studies}

Dupr\'{e}, Green, and Field.  \emph{Chem. Phys.} \textbf{196}, 211
(1995).\\
A number ($\sim 60$) of anticrossings from $\nu_3=0-2$ are analyzed
individually.

\begin{description}
\item[Observation: Large variation in measured $g$ factors.]
  Mixing between triplet states with the selection rule $\Delta M_S =
  \pm 1$ causes the $g$ factors to vary from the expected value of
  $2$.  The triplet$\sim$triplet interactions must be strong relative
  to the mediated interactions between $S_1$ and $T_{1,2}$.

  If the $S_0 \sim T_{1,2}$ were strong, the $g$ factors would all be
  close to zero.

\item[Observation: All quantum beats change with field strength.]
  Direct ``internal conversion'' interactions between $S_1$ and $S_0$
  would give rise to quantum beats which would not tune with magnetic
  field strength ($M_S=0$ for both states).

\item[Proposed order of interaction strengths: $T_1 \sim T_2 \gg S_1
  \sim T_{1,2} \gg S_0 \sim T_{1,2} \gg S_1 \sim S_0$.] The first
  inequality is supported by the large variation in measured $g$
  factors.  The second inequality allows the dense number of $S_0$
  levels to be sampled in the experiment, as observed in paper \#1.
  The third inequality states that the internal conversion process
  from $S_1$ to $S_0$ is triplet-mediated, as supported by the second
  observation in this paper.

  The ordering of interaction strengths not only proceeds from the
  observations, but also makes sense according to our chemical
  intuition.
\end{description}

\subsection*{Paper 4: Study by Fourier transform}

Dupr\'{e}. \emph{Chem. Phys.} \textbf{196}, 239 (1995).\\
The product $\rho_{vib} \cdot \langle V \rangle$ is derived from the
Fourier xform of anticrossing spectra for $\nu_3=0-4$.

\begin{description}
\item[Observation: The product $\rho_{vib} \cdot \langle V \rangle$
  increases exponentially from $0\nu_3$ to $4\nu_3$.]  See figure 1
  for a great example of how the spectra change from $0\nu_3$ to
  $4\nu_3$ (ignore frame E, which is the singlet perturber in
  $3\nu_3$).  The density of states does not increase appreciably over
  this energy range, so the effect must arise from an increasing
  matrix element.
\end{description}
\end{document}