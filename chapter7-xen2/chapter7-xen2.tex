\documentclass[12pt]{mitthesis} 
\usepackage[pdftex]{graphicx}
\usepackage{kylesthesis}
\begin{document}

\tableofcontents
\clearpage

\subsubsection*{NOTES}
\clearpage

\chapter{Collisional excitation of molecular triplet states by atomic
  metastables}

\section{Introduction}


Although molecular triplet states are hard to populate optically, they
are easily generated in electronic-energy exchanging collisions.  In
fact, this property is one of the primary resons for studying triplets
in the first place.  We should be able to generate large numbers of
triplet molecules by first creating large numbers of metastable atoms,
and then letting collisions do the work.

The process of mercury photosentization is well-known in organic
chemistry \cite{brown89, brown88, crabtree92, cvetanovic64,
  phillips74, strausz70}.  The common technique is to place mercury
and reactants in a heated reaction vessel, and irradiate with 257 nm
light from a mercury resonance lamp \cite{brown87}.

The first experimental detection of acetylene triplet states was
accomplished using the process of mercury photosensitization.  Later,
Kanamori and coworkers have used mercury photosensitization to study
the \emph{cis}-$T_2$ $\leftarrow$ \emph{cis}-$T_1$ absorption spectrum
of acetylene.

The main drawback to using a traditional mercury photosensitization
technique with acetylene is polymer formation.  This process is well
known to photochemists \cite{shida58, leroy44}, and to any unfortunate
spectroscopists who have used the technique.

The rate of polymer formation could be kept to a minimum if the number
density of mercury atoms could be carefully controlled.  In addition,
it would be nice if the particular atomic metastable state could be
selected based on the desired electronic energy.  In addition, we
would like to carry out experiments in a molecular beam, so that
rotational and vibrational cooling could concentrate the population of
triplet molecules to a relatively small number of rovibronic states.

We have developed, using a technique of optical pumping via two-photon
transitions, a method for generating large numbers of metastable atoms
in the early stages of a supersonic expansion, without polymer
formation.  We demonstrate the general principles of the technique for
the \ce{Xe}* + \ce{N2} system, and then report progress on
\ce{Hg}* + \ce{C2H2}.  We first discuss the details of two-photon
transitions in atoms, and weigh the various options for optical
pumping.

\section{Theory: Optical pumping of atomic metastables via two-photon
  transitions}

Our goal is to populate metastable excited states of mercury and xenon
atoms.  The lowest-energy excited state term of both atoms is
$^{1,3}P$, arising from a $6s6p$ configuration in mercury and a
$5p^56s$ configuration in xenon.  The $^{1,3}P$ term contains two
metastable levels, $^3P_0$ and $ ^3P_1$.

We examine the level structure of mercury first.  Excitation of one
electron into a $6p$ orbital gives rise to four energy levels: $^3P_0$,
$^3P_1$, $^3P_2$, and $^1P_1$.  The matrix of spin-orbit interaction
among these low-lying levels is
\begin{equation}
  \text{TODO: Insert condon and Shortley s.o. matrix here. Get Condon
    book from MIT.}
\end{equation}
The $^3P_1$ level is contaminated with singlet character to first
order; the $^3P_0$ acquires a small amount of singlet character to
second order.  The $^3P_2$ level has no spin-orbit interaction pathway
to the singlet, and is essentailly a pure triplet state.  The
lifetimes of the $^3P_0$ and $^3P_2$ levels are so excessively long
that experimental measurement is difficult.  Theoretical predictions
for the lifetimes of the $^3P_0$ and $^3P_2$ levels are on the order
of 1.0 and 0.5 s, respectively \cite{mishra01}.

The metastable triplet levels may be populated by radiative decay from
from singlet$\sim$triplet mixed levels at higher energy.  We examine a
series of potential two-photon excitation schemes, which result in
population of either $^3P_0$ and $^3P_1$, or both.  Many of the
high-lying atomic levels which radiate directly to $^3P_0$ and $^3P_1$
have been carefully studied, and accurate branching ratios are
available \cite{benck89}.  \TODO{Get Hg* folder from MIT.}  Two
especially promising candidates are $6 \; ^3D_2$ and $7 \; ^3S_1$.
The $6 \; ^3D_2$ level has a natural lifetime of 9 ns and a branching
ratio of 61:17:22 among the $^3P_1$, $^3P_2$, and $^1P_1$ levels
\cite{benck89}.  Thus, it is suitable to populate the higher-energy
metastable level, $^3P_2$.  The $7 \; ^3S_1$ level radiates only
within the $6 \; ^3P$ manifold, with a branching ratio of 17:45:39
among the $^3P_0$, $^3P_1$, and $^3P_2$ levels \cite{benck89}.  Both
metastable levels may be populated via radiative decay from $7 \;
^3S_1$.

Since we will consider some two-photon absorption (TPA) schemes that
include photons of unequal frequency, we must use a general formula
for two-photon transition probability.  Starting from the atomic
ground state, $\ket{i}$, the two-photon transition probability to a
final state $\ket{f}$ is given by
\begin{equation}
  \label{eq:tpa-prob}
  \begin{split}
    P_{f \leftarrow i} = &\: \: \frac{2 \pi}{\hbar^4}
    \left \lvert
      \sum_a
      \frac{
        \braket{f|\hat{\epsilon}_1 \cdot \bar{D}|a}\braket{a|\hat{\epsilon}_2 \cdot \bar{D}|f}
      }{
        \omega_{ai} - \omega_2 + i \Gamma_a / 2
      } + \frac{
        \braket{f|\hat{\epsilon}_2 \cdot \bar{D}|a}\braket{a|\hat{\epsilon}_1 \cdot \bar{D}|f}
      }{
        \omega_{ai} - \omega_1 + i \Gamma_a / 2
      }
    \right \rvert ^2\\[2mm]
    & \: \: \: \: \: \: \times 
      \frac{1}{\pi} 
      \frac{
        \Gamma_f / 2
      }{
        (\omega_1 + \omega_2 - \omega_{fi})^2+(\Gamma_f/2)^2
      } \frac{
        \omega_1 \omega_2
      }{
        4 \epsilon_0^2 c^4 k_1 k_2
      } \bar{I}_1 \bar{I}_2,\\
  \end{split}
\end{equation}
where the index $a$ runs over all intermediate states \cite{bonin84,
  grynberg77}.  In the formula above, $\omega_n = E_n / \hbar$ is the
energy of $\ket{n}$ in units of s$^{-1}$, $\omega_{nm} = (E_n -
E_m)/\hbar$ is the energy difference between $\ket{n}$ and $\ket{m}$
in units of s$^{-1}$, $\Gamma_n$ is the natural width of $\ket{n}$,
$\hat{\epsilon}_{1,2}$ is the unit polarization vector for each
photon, $k_{1,2}$ is the wave-vector magnitude of each photon, and
$\bar{I}_{1,2}$ is the intensity of each beam.

The following two-photon optical pumping schemes were investigated: 
\renewcommand{\theenumi}{(\alph{enumi})}
\renewcommand{\labelenumi}{\theenumi}
\begin{enumerate}
  \item one color TPA to $7 \; ^3S_1$,
  \item two color TPA to $7 \; ^3S_1$, with one laser tuned to the
    $7 \; ^3S_1 \rightarrow 6 \; ^3P_0$ downward transition,
  \item two color TPA to $7 \; ^3S_1$, using the  3nd harmonic Nd:YAG
    laser output at 355 nm,
  \item one color TPA to $6 \; ^3D_2$, and
  \item two color TPA to $6 \; ^3D_2$, with one laser tuned to the
    $6 \; ^3D_2 \rightarrow 6 \; ^3P_2$ downward transition.
\end{enumerate}
Figure \TODO{fig:hg-tpa-levels} shows the various optical pumping
schemes on an energy level diagram for the mercury atom.  For each
scheme, the total transition probability for TPA was calculated using
formula \ref{eq:tpa-prob}, yielding a final transition probability in
units of $\text{s}^{-1}(\text{W cm}^2)^{-2}$.  The calculated
transition probabilites for each scheme are displayed in Figure
\TODO{fig:hg-tpa-prob}.

\begin{figure}
  \caption{Diagram of possible two-photon excitation schemes for
    population of the $6 \; ^3P_0$ and $6 \; ^3P_2$ metastable excited
    states of Hg.  Excitation to the $7 \; ^3S_1$ level is followed by
    spontaneous decay to both metastable levels, while excitation to
    the $6 \; ^3D_2$ is followed spontaneous decay to only one of the
    metastable levels, $6 \; ^3P_2$.  The various schemes, labeled
    (a)-(e), are described in the text.  A dashed arrow is used to
    indicate two-photon excitation schemes that utilize photons of two
    different frequencies.}
  \label{fig:tpa-levels}
  \centering
  \includegraphics[height=8in,trim=4mm 0 0 0]{Hg-opticalpumpingschemes.pdf}
\end{figure}

% \begin{figure}
%   \caption{...}
%   \label{fig:tpa-prob}
% \end{figure}


One color TPA to $7 \; ^3S_1$, is rigorously forbidden, due to
interference between the quantum mechanical pathways leading from the
initial to final state.  Figure \TODO{fig:hg-forbidden} shows the
possible pathways, and their phases according to Formula
\ref{eq:tpa-prob}.

\POINT{Discuss selection rule against $J=1 \leftarrow 0$ for equal
  frequency photons.}


\POINT{Show relative transition probabilities for various one- and
  two-color pumping schemes}

\section{Theory: A collision model for electronic excitation transfer}

\section{Experiments: \ce{Xe}* + \ce{N2}}

\begin{figure}
  \caption{Laser-Induced Fluorescence (LIF) spectrum of the one color,
    two photon transition \ce{Xe} $6\;^3D_2 \leftarrow \leftarrow
    6\;^1S_0$, recorded under static cell conditions.  The LIF signal
    results from spontaneous emission to the metastable $6\;^3P_2$
    state at 823 nm.}
  \label{fig:xe3d2-cell}
  \centering
  \includegraphics[width=6in]{Xe3D2-cell.pdf}
\end{figure}

\begin{figure}
  \caption{Time dependence of \ce{N2} $B \rightarrow A$ emission,
    induced by collisions with metastable Xe* ($^3P_2$).  The static
    cell contained a 50:50 mixture of Xe and \ce{N2}, at a total
    pressure of ???  mTorr.  Metastable Xe* ($^3P_2$) was populated by
    spontaneous emission, following the $6\;^3D_2 \leftarrow
    \leftarrow 6\;^1S_0$ two-photon transition.}
  \label{fig:xen2-firstlight}
  \centering
  \includegraphics[width=7in,trim=2cm 0 1in 0]{XeN2-firstlight.pdf}
\end{figure}

\begin{figure}
  \caption{(Top) Laser-Induced Fluorescence (LIF) spectrum of the one
    color, two photon transition \ce{Xe} $6\;^3D_2 \leftarrow
    \leftarrow 6\;^1S_0$, recorded in a supersonic expansion.
    (Bottom) Time dependence of \ce{Xe} $6\;^3D_2 \rightarrow
    6\;^3P_2$ emission (solid trace), compared to a magnified signal
    resulting from scattered laser light (dashed trace). The
    fluorescence signal results from spontaneous emission to the
    metastable $6\;^3P_2$ state at 823 nm ($\tau = 28$ ns).
    \TODO{Change top plot to LIF intensity.}}
  \label{fig:xe-beam}
  \centering
  \includegraphics[width=6in]{Xe-beamlif-060406.pdf}
  \includegraphics[width=6in]{Xe-beamtrc-060406.pdf}
\end{figure}

\begin{figure}
  \caption{Time dependence of \ce{Xe} $6\;^3D_2 \rightarrow 6\;^3P_2$
    (top plot) and \ce{N2} $B \rightarrow A$ (bottom plot) emission,
    following the two-photon excitation of Xe ($^3D_2$) in a
    supersonic expansion.  The xenon and nitrogen emission signals
    occur on vastly different timescales.  Xenon fluorescence (823 nm,
    $\tau=28$ ns) is emitted when the optically excited state decays
    spontaneously to the metastable $6\;^3P_2$ state.  Nitrogen
    molecules are electronically excited during the expansion process
    in collisions with metastable xenon atoms.  Near-resonant
    vibrational levels of the nitrogen $B$ state decay spontaneously
    to the metastable $A$ state ($\tau=20$ \microsec), accompanied by
    fluorescence in the visible region of the spectrum.}
  \label{fig:xen2-traces}
  \centering
  \includegraphics[width=7.7in,angle=90,trim=0 0 1in 1cm ]{XeN2-traces.pdf}
\end{figure}



\section{Experiments: \ce{Hg}* + \ce{C2H2}}

\POINT{Discuss direct excitation to Hg*($^3P_2$).}

\begin{figure}
  \caption{Laser-Induced Fluorescence (LIF) spectrum of the one color,
    two photon transition \ce{Hg} $6\;^3D_2 \leftarrow \leftarrow
    6\;^1S_0$, recorded under static cell conditions.  The LIF signal
    results from spontaneous emission to the metastable $6\;^3P_2$
    state at 365 nm.}
  \label{fig:hg3d2-cell}
  \centering
  \includegraphics[width=6in]{Hg3D2-cell.pdf}
\end{figure}

\TODO{Figure: Direct excitation to Hg*($^3P_2$). (See p.47 of
  11/2007--1/2008 notebook.)}

\section{Results}

\section{Conclusions}

\bibliography{master} 
\bibliographystyle{plain}
\end{document}