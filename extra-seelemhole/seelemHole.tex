\documentclass{article}
\usepackage{amsmath, amssymb, braket}
\begin{document}

\section{The SEELEM intensity expression}

The intensity of a SEELEM-active eigenstate is formulated as a product
of sequential events that must happen for a metastable molecule to
eject an electron from the detector surface. Each factor in the
intensity expression can be expressed in terms of matrix elements and
energy differences between zero-oder $S_1$, $T_3$, and $T_{1,2}$
states. We derive, to a proportionality constant, the functional form
of this expression. We will show that the formulas become surprisingly
simple in regions where a few specific approximations are appropriate.

\subsection{Excitation probability factor}

The first event that must happen in order to detect an electron
ejected by a laser-excited metastable is laser excitation of the
molecule. The excitation probability is proportional to the amount of
zero-order bright state character squared:
\begin{equation}
P_{excit.} \propto C_{S_1}^2
\end{equation}


\subsection{Survival probability factor}

If an eigenstate has too much $S_1$ character, its SEELEM detectivity
will be small, because it will not live long enough to reach the
detector surface. This factor is exponential in its functional form,
and it acts as a sort of ``switch,'' filtering out SEELEM-active
states that exceed a certain threshold.
\begin{equation}
P_{surv.} \propto exp \left( -\frac{200 \mu s}{270 ns} \times C_{S_1}^2 \right)
\end{equation}

\subsection{Electron ejection factor}

There is some contraversy surrounding the exact form of the electron
ejection probability. In early SEELEM experiments, it was given
initially as
\begin{equation}
P_{eject.} \propto (C_{S1} + C_{T3})^2 .
\end{equation}
Later, it was revised by Altunata et. al. to
\begin{equation}
P_{eject.} \propto (C_{S1} + \beta \times C_{T3})^2 ,
\end{equation}
where $\beta$ is a proportionality factor (less than unity) that
expresses the relative excitation efficiencies of $S_1$ and $T_3$
basis states.

For many reasons, this has been rejected as the dominant form for the
electron ejection probability. We use the form
\begin{equation}
P_{eject.} \propto (C_{S1}^2 + C_{T3}^2)
\end{equation}
as a starting point, and in the next section we examine its effects on
the statistical properties of SEELEM intensity.

\section{Properties of the LEM intensity expression}

We wish to find relationships between statistical properties of the
SEELEM intensity distribution and the functional form for electron
ejection probability. To this end, we begin by examining the form of
the intensity expression \emph{before electron ejection}. We label
this the LEM (Laser-Excited Metastable) intensity expression, since it
includes factors for excitation and survival probability:
\begin{equation}
I_{LEM} \propto P_{excit.} \times P_{surv.}
\end{equation}

Since our goal is to investigate the functional form of the LEM
intensity and not its absolute magnitude, we henceforth drop an
absolute proportionality factor from the intensity equations. Relative
intensities are maintained in the presence of this global factor,
which is to be implicit in all further expressions that involve
absolute intensities.

Since the electron ejection probability will be similar in form to the
excitation probability, the properties of the LEM intensity
distribution will serve as an upper/lower bound to the properties of
the full SEELEM intensity expression.

\subsection{Approximations for $C_{S_1}$}

We evaluate the fractional $S_1$ character in a dark state by
pre-diagonalizing the $S_1 \sim T_3$ interaction and using
second-order perturbation theory. Defining the $S_1 \sim T_3$ mixing
amplitude as $\alpha = H_{ST} / \Delta E_{ST}$ and the $T_3 \sim$ dark
state spin-orbit matrix element as $H_{Td} =
\bra{T_d}H^{SO}\ket{T_3}$, we have
\begin{equation}
C_{S_1} = \alpha (1 - \alpha^2)^{1/2} \times H_{Td} \times \biggl\lbrace
  \frac{1}{\Delta E_{d1}} - \frac{1}{\Delta E_{d2}} \biggr\rbrace ,
\end{equation}
where $\Delta E_{d1,d2}$ represent the energy difference between the
dark state and the mixed $S_1$ or $T_3$ state, respectively.

To justify the use of perturbation theory for the $T_3 \sim T_d$
interaction, the energy difference between the two zero-order states
must not be less than twice the matrix element:
\begin{equation}
\Delta E_{Td} > 2 \times H_{Td} .
\end{equation}
If this condition is not satisfied, we muct check our work by direct
diagonalization of the effective Hamiltonian.

In the case that the $S_1 \sim T_3$ mixing amplitude is small (less
than about $0.25$), we can make the approximation that
\begin{equation}
(1 - \alpha^2) \approx 1 .
\end{equation}
This will diminish the result by 1 percent when the value of
$\alpha$ is [such and such].

In the case that the dark state of interest lies close in energy to
the pre-diagonalized $S_1$ state but much farther away from the
prediagonalized $T_3$ state, we have the condition that
\begin{equation}
\Delta E_{d1} \ll \Delta E_{d2} ,
\end{equation}
allowing the following factor to be simplified in the intensity
expression:
\begin{equation}
  \biggl\lbrace 
  \frac{1}{\Delta E_{d1}} - \frac{1}{\Delta E_{d2}} 
  \biggr\rbrace
  \approx \frac{1}{\Delta E_{d1}} .
\end{equation}
If this approximation cannot be made, all is not lost, because the
above expression can be rewritten in terms of $\Delta E_{d1}$ and
$\Delta E_{ST}$,
\begin{equation}
  \Delta E_{d2} = \Delta E_{ST} + 2 \alpha^2 \Delta E_{ST} + \Delta
  E_{d1} ,
\end{equation}
eliminating $\Delta E_{d2}$ as a parameter wherever second-order
perturbation theory applies.

In the presence of these two approximations, the excitation
probability becomes
\begin{equation}
  P_{excit.} \propto
  \biggl(
  \frac{H_{ST} \, H_{Td}}{\Delta E_{ST} \, \Delta E_{d1} }
  \biggr)^2
\end{equation}

\subsection{Peak intensity and FWHM}

The LEM intensity will have two maxima, where the survival probability
and excitation probability counteract each other exactly. The
derivative of the LEM intensity is easily written without any
approximations or perturbation theory, using the functional
relationships between the two contributing factors. The function is
maximized at the non-trivial zeros of the derivative, where
\begin{equation}
  \frac{\partial P_{excit.}}{\partial \Delta E_{d1}} \times P_{surv.} +
  \frac{\partial P_{surv.}}{\partial \Delta E_{d1}} \times P_{excit.}
  = 0 .
\end{equation}

The derivative of the survival probability may be written in terms of
the derivative of the excitation probability:
\begin{equation}
  \frac{\partial P_{surv.}}{\partial \Delta E_{d1}} = 
  - R_m \times P_{surv.} \times 
  \frac{\partial P_{excit.}}{\partial \Delta E_{d1}}
\end{equation}
This yields maxima where
\begin{equation}
  P_{excit.} = \frac{1}{R_m} .
\end{equation}
Consequently, the peak LEM intensities at the maxima are constant:
\begin{equation}
  (I_{LEM})_{max} = \frac{1}{e \times R_m}
\end{equation}

Up to this point, we have used no perturbation theory, nor have we
used any energy-dependent approximations which may break down under
certain conditions. The constant value of the LEM intensity maximum
comes as a result of it functional form alone.

This is a rather surprising result, but its shock is mitigated in part
when we examine the FWHM of the LEM intensity distribution using
perturbation theory. far from the center of the intensity
distribution, the survival probability is near unity, and we can
approximate the LEM intensity by the excitation probability
alone. Making the approximations outlined in the preceeding section,
it can be seen that the FWHM of $I_{LEM}$ varies linearly with the
coupling matrix elements.
\begin{equation}
FWHM = 2 \sqrt{2 e R_m} \biggl\lvert
\frac{H_{ST} \, H_{Td}}{\Delta E_{ST}} \biggr\rvert
\end{equation}
Under typical experimental conditions, the leading numerical factor is
about 174.

The expression for total LEM intensity is slightly more complicated,
but again varies almost linearly with the matrix elements.

\subsection{Possibility and width of a ``SEELEM hole'' in the
  spectrum}

We wish to investigate the possibility that, within a spectrum, some
dark states near the bright state energy may be systematically turned
off by the survival probability factor. This would have the effect of
producing a ``hole'' in the SEELEM spectrum, located in the immediate
vicinity of the bright state.

The HWHM of this hole is derived by setting the survival probability
to $0.5$:
\begin{equation}
\frac{1}{2} = exp \left[ -\frac{200 \mu s}{270 ns} \times C_{S_1}^2 \right],
\end{equation}
and finding the energy separation between the dark state and the
bright state which satisfies this condition. 

The amount of $S_1$ character is strongly energy dependent, so we
solve for this quantity:
\begin{equation}
log \frac{1}{2} = - \frac{200 * 10^{-6} s}{270 * 10^{-9} s} \times
C_{S_1}^2.
\end{equation}
Evaluating the numerical parts of this expression for typical
experimental conditions, we find that the amount of $S_1$ character in
a dark state on the cusp of the SEELEM hole is
\begin{equation}
C_{S_1}^2 \approx 10^{-3}
\end{equation}

Since we are investigating the energy immediately surrounding the mixed $S_1$
state, we make the approximation that $\Delta E_{d1} \gg \Delta
E_{d2}$, so
\begin{equation}
C_{S_1} \approx \frac{ \alpha (1 - \alpha^2)^{1/2} \times H_{Td}}{\Delta E_{d1}}.
\end{equation}

Substituting this into the original equation, we solve for the energy
difference between the mixed $S_1$ state and the dark state:
\begin{equation}
\Delta E_{d1} = \sqrt{\alpha^2 (1 - \alpha^2) \times H_{Td}^2 \times 10^{-3}}
\end{equation}
Furthermore, we make the apprximation that $(1 - \alpha^2) \approx 1$.
Although this condition may not always be true, the quantity is
certainly always less than 1, so in the worst case this gives an upper
bound for the width of the hole.

We rewrite the quantity in terms of the matrix element between the
zero-order $S_1$ and $T_3$ states, and factor out the approximately
constant parts to obtain the FWHM of the hole:
\begin{equation}
FWHM \approx 0.06 \times \biggl\lvert \frac{ H_{ST} \times H_{Td} }{ \Delta
  E_{ST} } \biggr\rvert .
\end{equation}

Coincidentally, the numerical factor in this expression is
approximately our laser linewidth in $cm^{-1}$. If we consider this
expression in units of reciporical centemeters, the SEELEM hole
will be resolvable in our experiments only if the factor in the
absolute value bars is greater than 1. Since spin-orbit matrix
elements between rovibronic states in acetylene are typically on the
order of 0.1, the SEELEM hole will not be resolvable unless the
zero-order $S_1 \sim T_3$ energy difference is less than approximately
$0.01 \, cm^{-1}$.

Since the vast majority of cases will not meet this requirement, we
conclude that such a SEELEM hole would not be resolvable in acetylene
unless the $S_1$ and $T_3$ rovibronic states are at an exact crossing
point. If this situation were to occur, such a resonance would almost
certainly last for only one value of $J$.

\end{document}