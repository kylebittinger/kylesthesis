\chapter{Conclusions}

% Intersystem crossing is a cornerstone of the dynamics of metastable
% molecules.  When this process is mediated by a special class of
% vibrational states, 

% In the limit of small $T_3$ level density, the tier model simplifies
% to a ``doorway-mediated'' model.  In most cases, one vibrational level
% of $T_3$, called the doorway state, dominates the $S_1 \sim T_3$
% mixing.  In this thesis, the doorway model for intersystem crossing is
% examined and characterized for cases where a doorway state is
% non-degenerate with a perturbed singlet level.

A theory of the spectral signatures of doorway-mediated intersystem
crossing is developed.  The problem is formulated under three distinct
frameworks, and the relationships between the frameworks are
described.  Four model Hamiltonians are discussed: the local doorway
model, the distant doorway model, the local doorway plus direct
coupling model, and the double doorway model.  Using the theory
developed in this thesis, the relative statistical properties of LIF
and SEELEM intensity distributions are derived for each class of model
Hamiltonian.

% In Chapter 2, we develop several frameworks for the problem of
% doorway-mediated intersystem crossing.  Using these frameworks, 

%  derive LIF intensity
% expressions for each, and compare the results.  We turn to evaluating
% terms in the SEELEM intensity for doorway-mediated systems.  Using
% these intensity expressions, we examine the spectral signatures of
% DMISC for several model Hamiltonians.  The models examined are: a
% local doorway, a distant doorway, a local doorway plus direct
% coupling, and a double (local + distant) doorway.

The parameters that define a local doorway model Hamiltonian are
determined uniquely by the absorption spectrum.  This discrete
spectral inversion process is carried out by a recursive application
of the standard Lawrance-Knight-Lehmann deconvolution method.  The
statistical moments of the spectrum are shown to be related, by simple
equations, to the important parameters of the model Hamiltonian.  The
technique of ``doorway deconvolution'' is demonstrated using examples
from \ce{NO2} and acetylene spectra.

% After a brief review of the continuous and discrete forms of spectral
% deconvolution, we demonstrate how to use the discrete deconvolution
% method recursively to recover the Spectroscopic Channel Basis of Ziv
% and Rhodes.  We provide formulas to simplify the computation of
% doorway state energy and matrix element, using moments of the spectral
% intensity.  By comparison with the discrete method, we show that the
% continuous method produces exact results for the zero-order density
% function, within the spectral resolution, when used with Lorentzian
% line shapes.  
% %To investigate errors in the deconvolution results, we
% %apply the process to a set of simulated spectra.  
% The usefulness of
% the technique is then demonstrated by application to a spectrum of the
% acetylene molecule.  The doorway deconvolution results are shown to be
% consistent with other investigations of the system.

% We have shown how to extend established methods of spectral
% deconvolution to uniquely determine the parameters of a
% doorway-coupling effective Hamiltonian.  The most important parameters
% of the effective Hamiltonian, the doorway state energy and
% bright$\sim$doorway matrix element, were related to simple moments of
% the spectral intensity distribution.  By making a correspondence to
% the work of Ziv and Rhodes on continuous deconvolution, we have shown
% how the parameters that define an effective Hamiltonian can be derived
% for any number of sequential doorway states.

% This technique was applied to the spectrum of the $3\nu_3$ $K=1$
% vibrational level of $S_1$ acetylene, where a single, local, $T_3$
% doorway level mediates coupling to the $T_{1,2}$ manifold.  The
% deperturbed $T_3$ energies and matrix elements were shown to be
% consistent with past studies.  A comparison was made to the $S$
% parameter of Altunata and Field, which is equal to the product
% $E_{\ell} \times H_{s\ell}^2$.

% In most cases, the $T_3$ vibrational levels responsible for mediating
% $S_1 \sim T_{1,2}$ mixing are not observed in the spectrum.  Relating
% these spectra to the doorway model requires new theory, new types of
% experiments, and new analysis techniques.

The experimental method of simultaneous SEELEM/LIF spectroscopy is
used to investigate singlet$\sim$triplet mixing in the acetylene
molecule.  New experimental results are reported for a number of $S_1$
vibrational levels, looking beyond the frequently studied $v_3'=0-4$
levels.  New methods are developed to analyze the spectra, which allow
the detailed structure of the spectrum to be related back to the
doorway model.

% We turn our attention to other vibrational levels of $S_1$ in the same
% energy region that are not near-degenerate with a mediating $T_3$
% level at low $J$.  In the absence of a local $T_3$ perturber, coupling
% between $S_1$ levels and the local manifold of $T_{1,2}$ levels is
% expected to be mediated by energetically distant $T_3$ levels.
% Evidence for such energetically distant, mediating $T_3$ levels is
% obtained by comparing simultaneously recorded LIF and SEELEM spectra.

% We begin by deriving the energy level spacings and spin-orbit matrix
% elements between rovibrational levels of $S_1$ and $T_3$.  Next, we
% develop a new analysis technique that is sensitive to the presence of
% distant $T_3$ doorway levels.  This new technique takes into account
% the shifts of relative intensities in the frequency-domain LIF
% spectrum as a function of delay time, which result from the time
% development of an incoherent ensemble of eigenstates having different
% radiative lifetimes.  We then apply this new technique to the
% simultaneously recorded LIF and SEELEM spectra of four $S_1$
% vibrational sublevels, and discuss the properties of admixed,
% energetically distant $T_3$ vibrational levels.

% The mechanism of doorway-mediated interaction by an energetically
% distant $T_3$ level skews SEELEM spectrum of nearby, nominal $T_{1,2}$
% eigenstates, resulting in a center of gravity shift between the LIF
% and SEELEM spectra.  Additionally, when viewed in successive time
% windows, the center of gravity of the LIF spectrum exhibits evolution
% toward the limiting behavior exhibited in the SEELEM spectrum.  A
% simple model can be used to show that strong mixing between the
% singlet level and the mediating $T_3$ level causes a gradual shift in
% the center of gravity, while weak mixing with the doorway level
% induces a more delayed and rapid shift in the center of gravity, with
% respect to the $S_1$ lifetime.

% Rotational selection rules for $S_1 \sim T_3$ spin-orbit interaction
% give rise to $J$-dependent effects in the LIF/SEELEM spectrum.  As $J$
% increases, the $F_1$ or $F_3$ component of any distant $T_3$ level
% approaches the singlet at a rate $d\Delta E / dJ$ of approximately
% $2B_T$ ($\sim$2 \rcm\ per $J$).  In the spectrum, the result is a
% shift in the LIF-SEELEM center of gravity when integrated across an
% entire branch of transitions.  The effect not only leads to strong
% $J$-dependent changes in the patterns of $T_3$-mediated coupling, but
% also ensures detection and assignment of $S_1 \sim T_3$ level
% crossings at relatively low values of $J$.

% For every $S_1$ vibrational level, one spin component of a $T_3$
% doorway level is rapidly approaching with $J$.  Because of this,
% further LIF/SEELEM spectroscopy of the acetylene \AtoX\ transition
% will be fruitful and informative.  We highlight some candidates for
% future investigation.
% \begin{enumerate}
% \item Levels which exhibit long lifetimes or quantum beats in the LIF
%   spectrum
% \item levels with unassigned perturbations or splittings,
% \item other $3^2B^2$ polyad members
% \item other $K$-sublevels of the Franck-Condon active levels studied
%   here.  
% \end{enumerate}
% % \TODO{Specify candidate bands specifically, give energies.  This
% %   section needs some work, overall.}

% The appearance of population quantum beats in the spectrum indicates a
% splitting on the order of 80 MHz or less.  Quantum beat waveforms
% follow a well-defined analytical expression, and an analysis of
% quantum beats determines both the matrix element and zero-order energy
% spacing of the levels involved.  The matrix elements gained from an
% analysis of zero-field quantum beats in the acetylene spectrum can
% serve as a probe of the magnitude of local matrix elements between the
% nominal $S_1$ bright state and neighboring $T_{1,2}$ dark states.
% Furthermore, the study of Zeeman quantum beats at low magnetic fields
% can provide a method for distinguishing between triplet perturbations
% and perturbations from other singlet levels, such as $S_1$ levels
% which are localized in the \emph{cis} geometry of the \astate\ state,
% because singlet states do not tune in a magnetic field.

The impact of torsional ($\nu_4'$) and antisymmetric in-plane bending
($\nu_6'$) vibrations on singlet-triplet mixing is investigated by
IR-UV double resonance LIF/SEELEM spectroscopy.  The effects of
Darling-Dennison resonance and \emph{a}-type Coriolis interaction,
which plagued earlier studies, are excluded by an appropriate choice
of vibrational sublevels.  The roles of the $\nu_4'$ and $\nu_6'$ modes in
promoting vibrational overlap with $T_3$ levels are discussed.

The understanding of ``distant doorway'' interactions in $S_1$
acetylene paints an optimistic picture for future study, using the
standard experimental techniques of LIF/SEELEM spectroscopy.
Specifically, the crucial finding is that one spin component of a
distant doorway level rapidly approaches a perturbed singlet level as
$J'$ is increased.  Not only does this explain many rotational effects
at low values of $J'$, but it also means that more perturbations are
likely to be observable in the range $5 \lesssim J' \lesssim 15$.
% Moreover, the observation of $T_3$ perturbers allows the
% experimentalist to estimate the rotationless energy of the parent
% $T_3$ vibrational sublevel.  A primary candidate for the observation
% of this phenomenon is the $3^24^2$ \Ka{1} sublevel, where strong
% perturbations are observed in the LIF spectrum at $J' \approx 7$.

% Chapter 5 presents the results of IR-UV double resonance SEELEM
% experiments, which reveal similar singlet-triplet coupling
% characteristics for

% To investigate the effects of $\nu_4$ and $\nu_6$ near
% the crucial half-linear and twisted geometries, we select the
% particular combination levels $3^34^1$ and $3^36^1$ of $S_1$
% acetylene.

% The interference effect is due to two separate, strong interactions
% between the $\nu_4$ and $\nu_6$ vibrations.  The first interaction is
% a strong Darling-Dennison resonance, which connects vibrations by
% exchanging two quanta of $\nu_4$ for two quanta of $\nu_6$, or vice
% versa.  Levels with only one quantum of modes 4 or 6, such as those
% studied by Mizoguchi, are immune to this effect.  The second
% interaction is the $a$-axis Coriolis coupling, which exchanges one
% quantum of mode 4 for one quantum of mode 6.  The matrix element for
% $a$-axis Coriolis coupling includes a factor of $K$, thus sublevels
% with $K=0$ are immune to this effect.

Methods are developed for the production of metastable molecules in a
molecular beam.  Two-photon transitions are used to generate
metastable atoms, and the molecules are excited via collisional
excitation transfer from the atoms.  The selection rules and
transition probabilities for two-photon excitation of the atoms are
discussed in detail for mercury.  SEELEM and LIF spectroscopy are used
to demonstrate the production of metastable acetylene and nitrogen
molecules in a molecular beam by the method of collisional electronic
excitation transfer.

Building on the success of our experiments with \ce{Xe}* + \ce{N2}, we
propose a new experiment to study collisional intersystem crossing in
the \ce{N2}* + \ce{Xe} system.  The proposal is motivated by the
ability to selectively excite many metastable rovibronic transitions
in \ce{N2}, and by our ability to selectively detect the various
products in the current apparatus.

% The fundamental patterns of triplet perturbations in $S_1$ acetylene
% are no longer a mystery.  

% We describe a method to generate large numbers of metastable atoms in
% the early stages of a supersonic expansion, without polymer formation.
% The metastable atoms are populated by a technique of optical pumping
% via two-photon transitions.  We first discuss the details of
% two-photon transitions in atoms, and weigh the various options for
% optical pumping.  A simple curve-crossing model is described for
% atom*-molecule collisional excitation transfer.  We demonstrate the
% general principles of the technique for the \ce{Xe}* + \ce{N2} system,
% and then report progress on \ce{Hg}* + \ce{C2H2}.




% LocalWords:  intersystem Lehmann Lawrance DMISC deperturbed Altunata polyad
% LocalWords:  perturber rovibrational sublevels Condon Zeeman cis Mizoguchi
% LocalWords:  versa Ottinger translationally sublevel
